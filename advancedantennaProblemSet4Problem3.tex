%
% Copyright � 2015 Peeter Joot.  All Rights Reserved.
% Licenced as described in the file LICENSE under the root directory of this GIT repository.
%
\makeoproblem{Binomial array.}{advancedantenna:problemSet4:3}{2015 ps4, p3}{
\index{binomial array}
%
Five antenna elements are placed symmetrically along the z-axis. The distance between
the elements is \( k d = 5 \pi / 4 \).  For a binomial array, find
%
\makesubproblem{}{advancedantenna:problemSet4:3a}
the excitation coefficients (currents)
\makesubproblem{}{advancedantenna:problemSet4:3b}
an expression for the array factor
\makesubproblem{}{advancedantenna:problemSet4:3c}
the normalized power pattern (for the array factor)
\makesubproblem{}{advancedantenna:problemSet4:3d}
the angles in degrees where the nulls (if any) occur.

Plot the power array factor with a tool like Matlab to verify your predictions.
} % makeoproblem
%
\makeanswer{advancedantenna:problemSet4:3}{
\makeSubAnswer{}{advancedantenna:problemSet4:3a}
%
The array geometry assuming is illustrated in \cref{fig:ps4p3:ps4p3Fig3}.
%
\imageFigure{../figures/ece1229-antenna/ps4p3Fig3}{Five element array on z-axis.}{fig:ps4p3:ps4p3Fig3}{0.2}
%
With \( \Br_m = m d \zcap, m \in [-2,2] \), the array factor is
%
\begin{equation}\label{eqn:advancedantennaProblemSet4Problem3:20}
\begin{aligned}
\textrm{AF} 
&= \sum_{m =-2}^2 I_m e^{ j k \Br_m \cdot \rcap } \\
&= \sum_{m=-2}^2 I_m e^{ j k d m \cos\theta }.
\end{aligned}
\end{equation}
%
Recall that the binomial expansion for \( N = 4 \) is
%
\begin{equation}\label{eqn:advancedantennaProblemSet4Problem3:40}
(z + 1)^4
 =
1 + 4 z + 6 z^2 + 4 z^3 + z^4,
\end{equation}
%
so the currents are
%
\begin{equation}\label{eqn:advancedantennaProblemSet4Problem3:100}
\begin{aligned}
I_{-2} &= 1 \\
I_{-1} = 4 \\
I_{0} = 6 \\
I_{1} = 4 \\
I_{2} = 1.
\end{aligned}
\end{equation}
%
\makeSubAnswer{}{advancedantenna:problemSet4:3b}
%
With \( z = e^{ j k d \cos\theta } \), we can assume a binomial representation of the form
%
\begin{equation}\label{eqn:advancedantennaProblemSet4Problem3:60}
\begin{aligned}
\textrm{AF}
&=
\sum_{m =-2}^2 \binom{4}{m+2} e^{j k d m \cos\theta }
\\ &=
\binom{4}{2} +
2 \sum_{m=1}^2 \binom{4}{m+2} \cos\lr{ k d m \cos\theta }
\\ &=
6 + 2 \lr{ 4 \cos\lr{ k d \cos\theta } + \cos\lr{ 2 k d \cos\theta } }.
\end{aligned}
\end{equation}
%
\makeSubAnswer{}{advancedantenna:problemSet4:3c}
%
Normalizing so that \( \textrm{AF} = 1 \) at \( \Omega = k d \cos\theta \rightarrow 0 \), gives
%
\begin{equation}\label{eqn:advancedantennaProblemSet4Problem3:80}
\begin{aligned}
%\boxedEquation{eqn:advancedantennaProblemSet4Problem3:80}{
\textrm{AF} &= \inv{8} \lr{ 3 + 4 \cos u + \cos 2 u } \\ &= \cos^4 (u/2),
%}
\end{aligned}
\end{equation}
%
where \( u = 5 \pi \cos\theta/4 \).  The substitution \( u = -j \ln z \), puts the array factor in explicit polynomial form
%
\begin{equation}\label{eqn:advancedantennaProblemSet4Problem3:120}
\textrm{AF}(z) = \inv{16 z^2} \lr{ 1 + z }^4.
\end{equation}
%
The leading \( 1/z^2 \) factor, which introduces negative power polynomial terms but does not change the roots, is because the expansion \cref{eqn:advancedantennaProblemSet4Problem3:60} effectively factored out a pure phase term.  This does not impact the power array factor \( \Abs{\textrm{AF}}^2 \).  Had the array elements not been placed symmetrically about the origin, instead being located at \( \Br_m = m d \zcap, m \in \setlr{0,1,2,3,5} \), this factor would have been eliminated.  A leading \( z^N \) factor in \( \textrm{AF}(z) \) is seen to be associated with the location of the origin of the coordinate system.
%
\makeSubAnswer{}{advancedantenna:problemSet4:3d}
%
Solutions for the nulls are found for integer \( N \) solutions of
%
\begin{equation}\label{eqn:advancedantennaProblemSet4Problem3:140}
\frac{5 \pi}{4} \cos\theta = \pi( 1 + 2 N ),
\end{equation}
%
Two solutions in the visible range can be found
%
\begin{equation}\label{eqn:advancedantennaProblemSet4Problem3:160}
\theta = \cos^{-1} \lr{ \pm 4/5 },
\end{equation}
%
or
%
\begin{equation}\label{eqn:advancedantennaProblemSet4Problem3:180}
\theta \in \setlr{ \ang{143.1}, \ang{36.9} }.
\end{equation}
%
The power array factor is plotted in \cref{fig:ps4p3Linear:ps4p3LinearFig1}, and \cref{fig:ps4p3Binomial3D:ps4p3Binomial3DFig2}.
\mathImageTwoFigures{../figures/ece1229-antenna/ps4p3LinearFig1}{../figures/ece1229-antenna/ps4p3LogScale50Fig1}{Polar plot of 5 element binomial power array factor.}{fig:ps4p3Linear:ps4p3LinearFig1}{scale=0.2}{ps4:p3.jl}
%\imageFigure{../figures/ece1229-antenna/ps4p3LinearFig1}{CAPTION: ps4p3LinearFig1}{fig:ps4p3Linear:ps4p3LinearFig1}{0.3}
%\imageFigure{../figures/ece1229-antenna/ps4p3LogScale50Fig1}{CAPTION: ps4p3LogScale50Fig1}{fig:ps4p3LogScale50:ps4p3LogScale50Fig1}{0.3}
%\mathImageFigure{../figures/ece1229-antenna/ps4p3Binomial2DFig1}{Polar plot of 5 element binomial power array factor.}{fig:ps4p3Binomial2D:ps4p3Binomial2DFig1}{0.3}{ps4:problem3BinomialArray.nb}
\mathImageFigure{../figures/ece1229-antenna/ps4p3Binomial3DFig2}{Spherical polar plot of 5 element binomial power array factor.}{fig:ps4p3Binomial3D:ps4p3Binomial3DFig2}{0.3}{ps4:problem3BinomialArray.nb}
}
