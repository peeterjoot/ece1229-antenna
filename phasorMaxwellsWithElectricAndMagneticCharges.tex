%
% Copyright � 2015 Peeter Joot.  All Rights Reserved.
% Licenced as described in the file LICENSE under the root directory of this GIT repository.
%
%\input{../blogpost.tex}
%\renewcommand{\basename}{phasorMaxwellsWithElectricAndMagneticCharges}
%\renewcommand{\dirname}{notes/ece1229/}
%%\newcommand{\dateintitle}{}
%%\newcommand{\keywords}{}
%
%\input{../peeter_prologue_print2.tex}
%
%\usepackage{ece1229}
%
%\beginArtNoToc
%
%\generatetitle{Phasor form of (extended) Maxwell's equations in Geometric Algebra}
\index{Geometric Algebra}
\index{Maxwell's equations}
\index{Maxwell's equation}
\index{phasor}
%\chapter{Phasor form of (extended) Maxwell's equations in Geometric Algebra}
%\label{chap:phasorMaxwellsWithElectricAndMagneticCharges}
%
Separate examinations of the phasor form of Maxwell's equation (with electric charges and current densities), and the Dual Maxwell's equation (i.e. allowing magnetic charges and currents) were just performed.  Here the structure of these equations with both electric and magnetic charges and currents will be examined.
%
\section{Space time split.}
\index{space time split}
The vector curl and divergence form of Maxwell's equations are
%
\begin{subequations}
\begin{dmath}\label{eqn:phasorMaxwellsWithElectricAndMagneticCharges:20}
\spacegrad \cross \bcE = -\PD{t}{\bcB} -\bcM
\end{dmath}
\begin{dmath}\label{eqn:phasorMaxwellsWithElectricAndMagneticCharges:40}
\spacegrad \cross \bcH = \bcJ + \PD{t}{\bcD}
\end{dmath}
\begin{dmath}\label{eqn:phasorMaxwellsWithElectricAndMagneticCharges:60}
\spacegrad \cdot \bcD = \rho
\end{dmath}
\begin{dmath}\label{eqn:phasorMaxwellsWithElectricAndMagneticCharges:80}
\spacegrad \cdot \bcB = \rho_\txtm.
\end{dmath}
\end{subequations}
%
In phasor form these are
%
\begin{subequations}
\begin{dmath}\label{eqn:phasorMaxwellsWithElectricAndMagneticCharges:100}
\spacegrad \cross \BE = - j k c \BB -\BM
\end{dmath}
\begin{dmath}\label{eqn:phasorMaxwellsWithElectricAndMagneticCharges:120}
\spacegrad \cross \BH = \BJ + j k c \BD
\end{dmath}
\begin{dmath}\label{eqn:phasorMaxwellsWithElectricAndMagneticCharges:140}
\spacegrad \cdot \BD = \rho
\end{dmath}
\begin{dmath}\label{eqn:phasorMaxwellsWithElectricAndMagneticCharges:160}
\spacegrad \cdot \BB = \rho_\txtm.
\end{dmath}
\end{subequations}
%
Switching to \( \BE = \BD/\epsilon_0, \BB = \mu_0 \BH\) fields (even though these aren't the primary fields in engineering), gives
%
\begin{subequations}
\begin{dmath}\label{eqn:phasorMaxwellsWithElectricAndMagneticCharges:180}
\spacegrad \cross \BE = - j k (c \BB) -\BM
\end{dmath}
\begin{dmath}\label{eqn:phasorMaxwellsWithElectricAndMagneticCharges:200}
\spacegrad \cross (c \BB) = \frac{\BJ}{\epsilon_0 c} + j k \BE
\end{dmath}
\begin{dmath}\label{eqn:phasorMaxwellsWithElectricAndMagneticCharges:220}
\spacegrad \cdot \BE = \rho/\epsilon_0
\end{dmath}
\begin{dmath}\label{eqn:phasorMaxwellsWithElectricAndMagneticCharges:240}
\spacegrad \cdot (c \BB) = c \rho_\txtm.
\end{dmath}
\end{subequations}
%
Finally, using
%
\begin{dmath}\label{eqn:phasorMaxwellsWithElectricAndMagneticCharges:260}
\Bf \Bg = \Bf \cdot \Bg + I \Bf \cross \Bg,
\end{dmath}
%
the divergence and curl contributions of each of the fields can be grouped
\begin{subequations}
\label{eqn:phasorMaxwellsWithElectricAndMagneticCharges:280}
\begin{dmath}\label{eqn:phasorMaxwellsWithElectricAndMagneticCharges:300}
\spacegrad \BE = \rho/\epsilon_0 - \lr{ j k (c \BB) +\BM} I
\end{dmath}
\begin{dmath}\label{eqn:phasorMaxwellsWithElectricAndMagneticCharges:320}
\spacegrad (c \BB I) = c \rho_\txtm I - \lr{ \frac{\BJ}{\epsilon_0 c} + j k \BE },
\end{dmath}
\end{subequations}
%
or
%
\begin{equation}\label{eqn:phasorMaxwellsWithElectricAndMagneticCharges:340}
\spacegrad \lr{ \BE + c \BB I }
=
\rho/\epsilon_0 - \lr{ j k (c \BB) +\BM} I
+
c \rho_\txtm I - \lr{ \frac{\BJ}{\epsilon_0 c} + j k \BE }.
\end{equation}
%
Regrouping gives Maxwell's equation including both electric and magnetic sources
\boxedEquation{eqn:phasorMaxwellsWithElectricAndMagneticCharges:360}{
%\begin{dmath}\label{eqn:phasorMaxwellsWithElectricAndMagneticCharges:360}
\lr{ \spacegrad + j k } \lr{ \BE + c \BB I }
=
\inv{\epsilon_0 c} \lr{ c \rho - \BJ }
+ \lr{ c \rho_\txtm - \BM } I.
%\end{dmath}
}
\index{Maxwell's equation!Geometric Algebra}
%
\section{Covariant form.}
\index{covariant form}
\index{Maxwell's equation!covariant}
%
It was observed that these can be put into a tidy four vector form by premultiplying by \( \gamma_0 \), where
\index{four vector}
%
\begin{subequations}
\label{eqn:phasorMaxwellsWithElectricAndMagneticCharges:380}
\begin{equation}\label{eqn:phasorMaxwellsWithElectricAndMagneticCharges:400}
J = \gamma_\mu J^\mu = \lr{ c \rho, \BJ }
\end{equation}
\begin{equation}\label{eqn:phasorMaxwellsWithElectricAndMagneticCharges:420}
M = \gamma_\mu M^\mu = \lr{ c \rho_\txtm, \BM }
\end{equation}
\begin{equation}\label{eqn:phasorMaxwellsWithElectricAndMagneticCharges:440}
\grad = \gamma_0 \lr{ \spacegrad + j k } = \gamma^k \partial_k + j k \gamma_0,
\end{equation}
\end{subequations}
%
That gives

\boxedEquation{eqn:phasorMaxwellsWithElectricAndMagneticCharges:460}{
%\begin{equation}\label{eqn:phasorMaxwellsWithElectricAndMagneticCharges:460}
\grad \lr{ \BE + c \BB I } = \frac{J}{\epsilon_0 c} + M I.
%\end{equation}
}
%
\section{Trial potential solution.}
%
When there were only electric sources, it was observed that potential solutions were of the form \( \BE + c \BB I \propto \grad \wedge A \), whereas when there was only magnetic sources it was observed that potential solutions were of the form \( \BE + c \BB I \propto (\grad \wedge F) I \).  It seems reasonable to attempt a trial solution that contains both such contributions, say
\index{spacetime gradient}
%
\begin{equation}\label{eqn:phasorMaxwellsWithElectricAndMagneticCharges:480}
\BE + c \BB I = \grad \wedge A_\txte + \lr{\grad \wedge A_\txtm} I.
\end{equation}
%
Without any loss of generality Lorentz gauge conditions can be imposed on the four-vector fields \( A_\txte, A_\txtm \).  Those conditions are
\index{Lorentz gauge}
\index{four potential!electric}
\index{four potential!magnetic}
%
\begin{equation}\label{eqn:phasorMaxwellsWithElectricAndMagneticCharges:500}
\grad \cdot A_\txte = \grad \cdot A_\txtm = 0.
\end{equation}
%
Since \( \grad X = \grad \cdot X + \grad \wedge X \), for any four vector \( X \), the trial solution \cref{eqn:phasorMaxwellsWithElectricAndMagneticCharges:480} is reduced to
%
\begin{equation}\label{eqn:phasorMaxwellsWithElectricAndMagneticCharges:520}
\BE + c \BB I = \grad A_\txte + \grad A_\txtm I.
\end{equation}
%
Maxwell's equation is now
%
\begin{dmath}\label{eqn:phasorMaxwellsWithElectricAndMagneticCharges:540}
\frac{J}{\epsilon_0 c} + M I
=
\grad^2 \lr{ A_\txte + A_\txtm I }
=
\gamma_0 \lr{ \spacegrad + j k }
\gamma_0 \lr{ \spacegrad + j k }
\lr{ A_\txte + A_\txtm I }
=
\lr{ -\spacegrad + j k }
\lr{ \spacegrad + j k }
\lr{ A_\txte + A_\txtm I }
=
-\lr{ \spacegrad^2 + k^2 }
\lr{ A_\txte + A_\txtm I }.
\end{dmath}
\index{Maxwell's equation!magnetic and electric potential separation}
%
Notice how tidily this separates into vector and trivector components.  Those are
%
\begin{subequations}
\label{eqn:phasorMaxwellsWithElectricAndMagneticCharges:560}
\begin{dmath}\label{eqn:phasorMaxwellsWithElectricAndMagneticCharges:580}
-\lr{ \spacegrad^2 + k^2 } A_\txte = \frac{J}{\epsilon_0 c}
\end{dmath}
\begin{dmath}\label{eqn:phasorMaxwellsWithElectricAndMagneticCharges:600}
-\lr{ \spacegrad^2 + k^2 } A_\txtm = M.
\end{dmath}
\end{subequations}
%
The result is a single Helmholtz equation for each of the electric and magnetic four-potentials, and both can be solved completely independently.  This was claimed in class, but now the underlying reason is clear.
\index{Helmholtz equation}
\index{electric four potential}
\index{magnetic four potential}
%
\section{Lorentz gauge application to Helmholtz.}
\index{Lorentz gauge}
%
Because a single frequency phasor relationship was implied the scalar components of each of these four potentials is determined by the Lorentz gauge condition.  For example
%
\begin{dmath}\label{eqn:phasorMaxwellsWithElectricAndMagneticCharges:620}
0
=
\spacegrad \cdot \lr{ A_\txte e^{j k c t} }
=
\lr{ \gamma^0 \inv{c} \PD{t}{} + \gamma^k \PD{x^k}{} } \cdot
\lr{
\gamma_0 A_\txte^0 e^{j k c t}
+ \gamma_m A_\txte^m e^{j k c t}
}
=
\lr{ \gamma^0 j k + \gamma^r \PD{x^r}{} } \cdot
\lr{
\gamma_0 A_\txte^0
+ \gamma_s A_\txte^s
}
e^{j k c t}
=
\lr{
j k
A_\txte^0
+
\spacegrad \cdot
\BA_\txte
}
e^{j k c t},
\end{dmath}
%
so
%
\begin{dmath}\label{eqn:phasorMaxwellsWithElectricAndMagneticCharges:640}
A_\txte^0
=\frac{ j} { k }
\spacegrad \cdot
\BA_\txte.
\end{dmath}
%
The same sort of relationship will apply to the magnetic potential too.  This means that the Helmholtz equations can be solved in the three vector space as
%
\begin{subequations}
\label{eqn:phasorMaxwellsWithElectricAndMagneticCharges:660}
\begin{dmath}\label{eqn:phasorMaxwellsWithElectricAndMagneticCharges:680}
\lr{ \spacegrad^2 + k^2 } \BA_\txte = -\frac{\BJ}{\epsilon_0 c}
\end{dmath}
\begin{dmath}\label{eqn:phasorMaxwellsWithElectricAndMagneticCharges:700}
\lr{ \spacegrad^2 + k^2 } \BA_\txtm = -\BM.
\end{dmath}
\end{subequations}
\index{Helmholtz equation!electric current density}
\index{Helmholtz equation!magnetic current density}
%
\section{Recovering the fields.}
%
Relative to the observer frame implicitly specified by \( \gamma_0 \), here's an expansion of the curl of the electric four potential
%
\begin{dmath}\label{eqn:phasorMaxwellsWithElectricAndMagneticCharges:720}
\grad \wedge A_\txte
=
\inv{2}\lr{
\grad A_\txte
-
A_\txte \grad
}
=
\inv{2}\lr{
\gamma_0 \lr{ \spacegrad + j k } \gamma_0 \lr{  A_\txte^0 - \BA_\txte }
-
\gamma_0 \lr{  A_\txte^0 - \BA_\txte } \gamma_0 \lr{ \spacegrad + j k }
}
=
\inv{2}\lr{
\lr{ -\spacegrad + j k }  \lr{  A_\txte^0 - \BA_\txte }
-
 \lr{  A_\txte^0 + \BA_\txte }  \lr{ \spacegrad + j k }
}
=
\inv{2}\lr{
- 2 \spacegrad A_\txte^0 + \cancel{ j k A_\txte^0 } - \cancel{ j k A_\txte^0 }
+ \spacegrad \BA_\txte - \BA_\txte \spacegrad
- 2 j k \BA_\txte
}
=
- \lr{ \spacegrad A_\txte^0 + j k \BA_\txte }
+ \spacegrad \wedge \BA_\txte.
\end{dmath}
%
In the above expansion when the gradients appeared on the right of the field components, they are acting from the right (i.e. implicitly using the Hestenes dot convention.)

The electric and magnetic fields can be picked off directly from above, and in the units implied by this choice of four-potential are
%
\begin{subequations}
\label{eqn:phasorMaxwellsWithElectricAndMagneticCharges:740}
\begin{equation}\label{eqn:phasorMaxwellsWithElectricAndMagneticCharges:760}
\BE_\txte = - \lr{ \spacegrad A_\txte^0 + j k \BA_\txte } = -j \lr{ \inv{k}\spacegrad \spacegrad \cdot \BA_\txte + k \BA_\txte }
\end{equation}
\begin{equation}\label{eqn:phasorMaxwellsWithElectricAndMagneticCharges:780}
c \BB_\txte = \spacegrad \cross \BA_\txte.
\end{equation}
\end{subequations}
\index{electric field!in terms of potentials}
%
For the fields due to the magnetic potentials
%
\begin{dmath}\label{eqn:phasorMaxwellsWithElectricAndMagneticCharges:800}
\lr{ \grad \wedge A_\txte } I
=
- \lr{ \spacegrad A_\txte^0 + j k \BA_\txte } I
- \spacegrad \cross \BA_\txte,
\end{dmath}
\index{magnetic field!in terms of potentials}
%
so the fields are
%
\begin{subequations}
\label{eqn:phasorMaxwellsWithElectricAndMagneticCharges:820}
\begin{equation}\label{eqn:phasorMaxwellsWithElectricAndMagneticCharges:840}
c \BB_\txtm = - \lr{ \spacegrad A_\txtm^0 + j k \BA_\txtm } = -j \lr{ \inv{k}\spacegrad \spacegrad \cdot \BA_\txtm + k \BA_\txtm }
\end{equation}
\begin{equation}\label{eqn:phasorMaxwellsWithElectricAndMagneticCharges:860}
\BE_\txtm = -\spacegrad \cross \BA_\txtm.
\end{equation}
\end{subequations}
%
Including both electric and magnetic sources the fields are
%
\begin{subequations}
\label{eqn:phasorMaxwellsWithElectricAndMagneticCharges:880}
\begin{equation}\label{eqn:phasorMaxwellsWithElectricAndMagneticCharges:900}
\BE = -\spacegrad \cross \BA_\txtm -j \lr{ \inv{k}\spacegrad \spacegrad \cdot \BA_\txte + k \BA_\txte }
\end{equation}
\begin{equation}\label{eqn:phasorMaxwellsWithElectricAndMagneticCharges:920}
c \BB = \spacegrad \cross \BA_\txte -j \lr{ \inv{k}\spacegrad \spacegrad \cdot \BA_\txtm + k \BA_\txtm }.
\end{equation}
\end{subequations}
%
Observe that the alternation of signs is exactly that of a superposition of electric dipole and magnetic dipole fields.  This is consistent with the fact that the dual form of Maxwell's equations has been designed explicitly to model infinitesimal current loops as sources.
%
%\EndNoBibArticle
