%
% Copyright � 2015 Peeter Joot.  All Rights Reserved.
% Licenced as described in the file LICENSE under the root directory of this GIT repository.
%
\makeoproblem{Aperture antenna.}{advancedantenna:problemSet5:1}{2015 ps5, p1}{
\index{aperture antenna}
A \textAndIndex{rectangular aperture} lies along the x-y plane and has dimensions \( a \times b \). Let the electric
field aperture distribution be given by,
%
\begin{dmath}\label{eqn:advancedantennaProblemSet5Problem1:20}
\BE_{\textrm{ap}} = \ycap \cos\lr{ \frac{\pi}{a} x }.
\end{dmath}
%
The aperture is cut out of an infinite perfectly electric conductor.
The origin of the coordinate system is at the center of the aperture.

Using the theory of radiation from apertures based on the equivalence principle, calculate:

\begin{enumerate}
\item
An expression for \( E_\theta(\theta, \phi) \)
\item
An expression for \( E_\phi(\theta, \phi) \)
\item
Consider an aperture of dimensions \( a = b = 10 \si{cm} \) at \( f = 9.8 \si{GHz} \).
\end{enumerate}
%
\makesubproblem{}{advancedantenna:problemSet5:1a}
Plot the E-plane and H-plane patterns (power).
\makesubproblem{}{advancedantenna:problemSet5:1b}
Calculate the positions of the first nulls in the E and H planes.
\makesubproblem{}{advancedantenna:problemSet5:1c}
From the plot determine the levels of the first sidelobe in the E and H planes.
\makesubproblem{}{advancedantenna:problemSet5:1d}
From the plot determine the 3dB beamwidth of the main lobe in the E and H planes.
} % makeoproblem
%
\makeanswer{advancedantenna:problemSet5:1}{
%
Following the transformation procedure of \citep{balanis2005antenna} \textfigref{12.5}, the equivalent source for this electric field is a magnetic current
%
\begin{equation}\label{eqn:advancedantennaProblemSet5Problem1:40}
\BM_\txts
= - 2 \zcap \cross \ycap \cos\lr{ \frac{\pi}{a} x }
= 2 \xcap \cos\lr{ \frac{\pi}{a} x }.
\end{equation}
%
producing an electric vector potential that is approximately
%
\begin{dmath}\label{eqn:advancedantennaProblemSet5Problem1:60}
\BF
= \frac{\epsilon}{4 \pi r} \int_{-a/2}^{a/2} dx' \int_{-b/2}^{b/2} dy' \BM_\txts e^{-j k ( r - \rcap \cdot \Br' ) }
= \frac{\epsilon}{2 \pi r} e^{-j k r} \xcap \int_{-a/2}^{a/2} dx' \int_{-b/2}^{b/2} dy' \cos\lr{ \frac{\pi}{a} x' } e^{j k \rcap \cdot \Br' }
= \frac{\epsilon}{2 \pi r} e^{-j k r} \xcap \int_{-a/2}^{a/2} dx' \int_{-b/2}^{b/2} dy' \cos\lr{ \frac{\pi}{a} x' }
e^{j k \sin\theta \lr{\cos\phi x' + \sin\phi y'} }
= \frac{\epsilon}{4 \pi r} e^{-j k r} \xcap \int_{-a/2}^{a/2} dx'
\lr{
e^{j k \sin\theta \cos\phi x' + j \pi x'/a}
+
e^{j k \sin\theta \cos\phi x' - j \pi x'/a}
}
\int_{-b/2}^{b/2} dy'
e^{j k \sin\theta \sin\phi y' }.
\end{dmath}
%
A symmetric interval around the origin has been chosen to avoid the introduction of complex phases.

Each of these integrals is of the form
%
\begin{dmath}\label{eqn:advancedantennaProblemSet5Problem1:80}
\int_{-c/2}^{c/2} dz'
e^{j \alpha z' }
=
\evalrange{ \frac{e^{j \alpha z' }}{j \alpha} }{-c/2}{c/2}
=
\frac{ e^{j \alpha c/2} - e^{-j \alpha c/2} }{ j \alpha }
=
\frac{\sin\lr{\alpha c /2} }{\alpha/2}.
\end{dmath}
%
With
%
\begin{subequations}
\label{eqn:advancedantennaProblemSet5Problem1:360}
\begin{dmath}\label{eqn:advancedantennaProblemSet5Problem1:380}
X = k \sin\theta \cos\phi
\end{dmath}
\begin{dmath}\label{eqn:advancedantennaProblemSet5Problem1:400}
Y = k \sin\theta \sin\phi,
\end{dmath}
\end{subequations}
%
the electric vector potential is
%
\begin{dmath}\label{eqn:advancedantennaProblemSet5Problem1:100}
\BF
= \frac{\epsilon}{4 \pi r} e^{-j k r} \xcap
\lr{
\frac{ \sin\lr{ (X + \pi/a) a/2 } }{ (X + \pi/a)/2}
+
\frac{ \sin\lr{ (X - \pi/a) a/2 } }{ (X - \pi/a)/2}
}
\frac{ \sin\lr{ Y b/2 } }{ Y/2}
= \frac{\epsilon}{2 \pi r} e^{-j k r} \xcap
\frac{
(X - \pi/a) \sin\lr{ X a/2 + \pi/2 }
+
(X + \pi/a) \sin\lr{ X a/2 - \pi/2 }
}{ X^2 - (\pi/a)^2}
\frac{ \sin\lr{ Y b/2 } }{ Y/2}.
\end{dmath}
%
Since
\begin{subequations}
\label{eqn:advancedantennaProblemSet5Problem1:120}
\begin{dmath}\label{eqn:advancedantennaProblemSet5Problem1:140}
\sin( z + \pi/2 ) + \sin( z - \pi/2 ) = 0
\end{dmath}
\begin{dmath}\label{eqn:advancedantennaProblemSet5Problem1:160}
\sin( z + \pi/2 ) - \sin( z - \pi/2 ) = 2 \cos z,
\end{dmath}
\end{subequations}
%
this reduces to
%
\begin{dmath}\label{eqn:advancedantennaProblemSet5Problem1:180}
\BF
=
-\frac{\epsilon a b}{4 r} e^{-j k r} \xcap
\frac{
\cos\lr{ X a/2 }
}{ (X a/2)^2 - (\pi/2)^2}
\frac{ \sin\lr{ Y b/2 } }{ Y b/2}.
\end{dmath}
%
The far field magnetic field is
%
\begin{dmath}\label{eqn:advancedantennaProblemSet5Problem1:200}
\BH
= -j \omega \BF_T
=
j k c \frac{\epsilon a b}{4 r } e^{-j k r} \lr{ \xcap - \lr{\xcap \cdot \rcap} \rcap }
\frac{
\cos\lr{ X a/2 }
}{ a^2 X^2/4 - (\pi/2)^2}
\frac{ \sin\lr{ Y b/2 } }{ Y b/2}
=
\frac{j k a b}{4 r \eta} e^{-j k r} \lr{ \xcap - \lr{ \xcap \cdot \rcap} \rcap }
\frac{
\cos\lr{ X a/2 }
}{ (X a/2)^2 - (\pi/2)^2}
\frac{ \sin\lr{ Y b/2 } }{ Y b/2}.
\end{dmath}
%
Since
%
\begin{dmath}\label{eqn:advancedantennaProblemSet5Problem1:280}
\xcap = \sin\theta \cos\phi \rcap + \cos\theta \cos\phi \thetacap - \sin\phi \phicap,
\end{dmath}
%
the far field magnetic field is

\boxedEquation{eqn:advancedantennaProblemSet5Problem1:420}{
\BH
=
\frac{j k a b}{4 r \eta} e^{-j k r} \lr{ \cos\theta \cos\phi \thetacap - \sin\phi \phicap}
\frac{
\cos\lr{ X a/2 }
}{ (X a/2)^2 - (\pi/2)^2}
\frac{ \sin\lr{ Y b/2 } }{ Y b/2}.
}

This can be related to the electric field noting that that the dual of the far field relationship
%
\begin{dmath}\label{eqn:advancedantennaProblemSet5Problem1:220}
\BH_A = \inv{\eta} \rcap \cross \BE_A,
\end{dmath}
%
is
%
\begin{dmath}\label{eqn:advancedantennaProblemSet5Problem1:240}
-\BE_F = \eta \rcap \cross \BH_F,
\end{dmath}
%
so the far field electric field is
%
\begin{dmath}\label{eqn:advancedantennaProblemSet5Problem1:260}
\BE
= -\eta \rcap \cross \BH
=
-\frac{j k a b}{4 r } e^{-j k r} \rcap \cross \xcap
\frac{
\cos\lr{ X a/2 }
}{ (X a/2)^2 - (\pi/2)^2}
\frac{ \sin\lr{ Y b/2 } }{ Y b/2}.
\end{dmath}
%
That electric field direction is
%
\begin{dmath}\label{eqn:advancedantennaProblemSet5Problem1:300}
\rcap \cross \xcap
=
\cos\theta \cos\phi \rcap \cross \thetacap - \sin\phi \rcap \cross \phicap
=
\cos\theta \cos\phi \phicap + \sin\phi \thetacap,
\end{dmath}
%
so the electric field is

\boxedEquation{eqn:advancedantennaProblemSet5Problem1:440}{
\BE
=
-\frac{j k a b}{4 r } e^{-j k r} \lr{ \cos\theta \cos\phi \phicap + \sin\phi \thetacap }
\frac{
\cos\lr{ X a/2 }
}{ (X a/2)^2 - (\pi/2)^2}
\frac{ \sin\lr{ Y b/2 } }{ Y b/2}.
}

Note that the electric and magnetic fields are perpendicular, as expected.

\begin{enumerate}
\item
The polar coordinate of the electric field is
%
\begin{dmath}\label{eqn:advancedantennaProblemSet5Problem1:320}
E_\theta =
-\frac{j k a b}{4 r } e^{-j k r} \sin\phi
\frac{
\cos\lr{ X a/2 }
}{ (X a/2)^2 - (\pi/2)^2}
\frac{ \sin\lr{ Y b/2 } }{ Y b/2}.
\end{dmath}
%
\item
The azimuthal coordinate of the electric field is
%
\begin{dmath}\label{eqn:advancedantennaProblemSet5Problem1:340}
E_\phi
=
-\frac{j k a b}{4 r } e^{-j k r} \cos\theta \cos\phi
\frac{
\cos\lr{ X a/2 }
}{ (X a/2)^2 - (\pi/2)^2}
\frac{ \sin\lr{ Y b/2 } }{ Yb/2}.
\end{dmath}
%
\item
Now for the plots and numeric values requested for the given aperture size and source frequency.

The electric field power pattern for
an aperture of dimensions \( a = b = 10 \si{cm} \) at \( f = 9.8 \si{GHz} \) is plotted in \si{dB} scale from 0 \si{dB} down to 40 \si{dB} in \cref{fig:sphericalPolarPlot:sphericalPolarPlotFig6}.
\mathImageFigure{../figures/ece1229-antenna/sphericalPolarPlotFig6}{Electric field power pattern, 0 \si{dB} to -40 \si{dB}.}{fig:sphericalPolarPlot:sphericalPolarPlotFig6}{0.4}{ps5:problem1plot.nb}
\end{enumerate}
\makeSubAnswer{}{advancedantenna:problemSet5:1a}
%
The maximum field is found at \( \theta = 0 \).
The value of \( \phi \) is inconsequential, so we have an infinite number of E-plane surfaces, and can pick the \( \theta = \phi = 0 \) wave vector direction for simplicity.  For such a wave vector direction \( \Ecap = \ycap, \Hcap = \xcap \), and the corresponding E-plane and H-plane power fields are plotted in 
\cref{fig:hplane:hplaneFig2}.
%\cref{fig:eplane:eplaneFig2}.
%, and \cref{fig:hplane:hplaneFig3}.

\mathImageTwoFigures{../figures/ece1229-antenna/eplaneFig2}{../figures/ece1229-antenna/hplaneFig3}{E-H-plane (power) for \( \phi =0 \).}{fig:hplane:hplaneFig2}{scale=0.4}{ps5:vecE.m}
%\mathImageFigure{../figures/ece1229-antenna/eplaneFig2}{E-plane (power) for \( \phi =0 \).}{fig:eplane:eplaneFig2}{0.3}{ps5:vecE.m}
%\mathImageFigure{../figures/ece1229-antenna/hplaneFig3}{H-plane (power) for \( \phi =0 \).}{fig:hplane:hplaneFig3}{0.3}{ps5:vecE.m}

These fields are also plotted on a log scale in \cref{fig:eplanePolar:eplanePolarFig4}
%, and \cref{fig:hplanePolar:hplanePolarFig5}
, from 0 \si{dB} down to -50 \si{dB}.

\mathImageTwoFigures{../figures/ece1229-antenna/eplanePolarFig4}{../figures/ece1229-antenna/hplanePolarFig5}{E,H-plane (power) for \( \phi =0 \), \si{dB} scale.}{fig:eplanePolar:eplanePolarFig4}{scale=0.5}{ps5:vecE.m}
%\mathImageFigure{../figures/ece1229-antenna/eplanePolarFig4}{E-plane (power) for \( \phi =0 \), \si{dB} scale.}{fig:eplanePolar:eplanePolarFig4}{0.3}{ps5:vecE.m}
%\mathImageFigure{../figures/ece1229-antenna/hplanePolarFig5}{H-plane (power) for \( \phi =0 \), \si{dB} scale.}{fig:hplanePolar:hplanePolarFig5}{0.3}{ps5:vecE.m}
%
\makeSubAnswer{}{advancedantenna:problemSet5:1b}
%
For the E-plane the zeros are found at
\ang{27}, \ang{50}, \ang{90}, \ang{130}, and \ang{153}.

For the H-plane the zeros are found at
\ang{18}, \ang{38}, \ang{67}, \ang{113}, \ang{142}, \ang{162}.

These were determined numerically, but these can also be visually verified against the \si{dB} power plots above, which are marked in degrees.
%
\makeSubAnswer{}{advancedantenna:problemSet5:1c}
%
For the E-plane the sidelobe peaks are found at
\ang{35}, \ang{60}, \ang{120}, \ang{145}, with respective levels (\si{dB}) of
 -25,
 -37,
 -37,
 -25.

For the H-plane the sidelobe peaks are found at
\ang{26}, \ang{49}, \ang{90}, \ang{131}, \ang{154}, with respective levels (\si{dB}) of
 -13.2666,
 -17.8436,
 -22.8361,
 -17.8436,
 -13.2666.

These were also calculated numerically, but can also be visually verified against the \si{dB} power plots above.
%
\makeSubAnswer{}{advancedantenna:problemSet5:1d}
%
The -3 \si{dB} point of the main lobe is found where \( \Abs{\BE} = 10^{-3/20} \).  For the E-plane this is at \ang{10}, and for the H-plane this is found at \ang{8}.
}
