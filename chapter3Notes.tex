%
% Copyright © 2015 Peeter Joot.  All Rights Reserved.
% Licenced as described in the file LICENSE under the root directory of this GIT repository.
%
%\input{../blogpost.tex}
%\renewcommand{\basename}{chapter3Notes}
%\renewcommand{\dirname}{notes/ece1229/}
%%\newcommand{\dateintitle}{}
%%\newcommand{\keywords}{}
%
%\input{../peeter_prologue_print2.tex}
%
%\usepackage{ece1229}
%
%\beginArtNoToc
%
%\generatetitle{Maxwell's equations}
%%\chapter{Maxwell's equations}
%%\label{chap:chapter3Notes}
%
%These are notes for the UofT course ECE1229, Advanced Antenna Theory, taught by Prof. Eleftheriades, covering \chaptext 3 \citep{balanis2005antenna} content.
%
%Unlike most of the other classes I have taken, I am not attempting to take comprehensive notes for this class.  The class is taught on slides that match the textbook so closely, there is little value to me taking notes that just replicate the text.  Instead, I am annotating my copy of textbook with little details instead.  My usual notes collection for the class will contain musings of details that were unclear, or in some cases, details that were provided in class, but are not in the text (and too long to pencil into my book.)

%\section{Maxwell's equation review}
\index{Maxwell's equations}
\section{Review}

For reasons that are yet to be seen (and justified), we work with a generalization of Maxwell's equations to include
electric AND magnetic charge densities.

%
% Copyright © 2015 Peeter Joot.  All Rights Reserved.
% Licenced as described in the file LICENSE under the root directory of this GIT repository.
%
\index{magnetic charge density}
\index{electric charge density}
\index{magnetic current density}
\index{electric current density}
%
\begin{subequations}
\label{eqn:chapter3Notes:19}
\begin{dmath}\label{eqn:chapter3Notes:20}
\spacegrad \cross \bcE = - \bcM - \PD{t}{\bcB}
\end{dmath}
\begin{dmath}\label{eqn:chapter3Notes:40}
\spacegrad \cross \bcH = \bcJ + \PD{t}{\bcD}
\end{dmath}
\begin{dmath}\label{eqn:chapter3Notes:60}
\spacegrad \cdot \bcD = q_\txte
\end{dmath}
\begin{dmath}\label{eqn:chapter3Notes:80}
\spacegrad \cdot \bcB = q_\txtm.
\end{dmath}
\end{subequations}
%


Assuming a phasor relationships of the form \( \bcE = \Real \lr{ \BE(\Br) e^{j \omega t}} \) for the fields and the currents, these reduce to

%
% Copyright © 2015 Peeter Joot.  All Rights Reserved.
% Licenced as described in the file LICENSE under the root directory of this GIT repository.
%
\begin{subequations}
\label{eqn:chapter3Notes:99}
\begin{dmath}\label{eqn:chapter3Notes:100}
\spacegrad \cross \BE = - \BM - j \omega \BB
\end{dmath}
\begin{dmath}\label{eqn:chapter3Notes:120}
\spacegrad \cross \BH = \BJ + j \omega \BD
\end{dmath}
\begin{dmath}\label{eqn:chapter3Notes:140}
\spacegrad \cdot \BD = \rho
\end{dmath}
\begin{dmath}\label{eqn:chapter3Notes:160}
\spacegrad \cdot \BB = \rho_\txtm.
\end{dmath}
\end{subequations}
%

%
% Copyright © 2015 Peeter Joot.  All Rights Reserved.
% Licenced as described in the file LICENSE under the root directory of this GIT repository.
%
In engineering the fields
\begin{itemize}
	\item \( \bcE(\Bx, t) \) : Electric field intensity [\si{V/m}] (Volts/meter)
	\item \( \bcH(\Bx, t) \) : Magnetic field intensity [\si{A/m}] (Amperes/meter)
\end{itemize}
are designated the primary fields, whereas
\begin{itemize}
	\item \( \bcD(\Bx, t) \) : Electric flux density (or displacement vector) [\si{C/m}] (Coulombs/meter)
	\item \( \bcB(\Bx, t) \) : Magnetic flux density [\si{W/m^2}] (Webers/square meter)
\end{itemize}
are designated the induced fields.  The currents and charges are
\begin{itemize}
	\item \( \bcJ(\Bx, t) \) : Electric current density [\si{A/m^2}] (Amperes/square meter)
	\item \( \bcM(\Bx, t) \) : Magnetic current density [\si{V/m^2}] (Volts/square meter)
	\item \( q_\txte(\Bx, t) \) : Electric charge density [\si{C/m^3}] (Coulombs/cubic meter)
	\item \( q_\txtm(\Bx, t) \) : Magnetic charge density [\si{W/m^3}] (Webers/cubic meter)
\end{itemize}


Because \( \spacegrad \cdot \lr{ \spacegrad \cross \Bf } = 0 \) for any (sufficiently continuous) vector \( \Bf \), divergence relations between the currents and the charges follow from \cref{eqn:chapter3Notes:99}
%
\begin{dmath}\label{eqn:chapter3Notes:180}
0
= -\spacegrad \cdot \BM - j \omega \spacegrad \cdot \BB
= -\spacegrad \cdot \BM - j \omega \rho_\txtm,
\end{dmath}
%
and
%
\begin{dmath}\label{eqn:chapter3Notes:200}
0
= \spacegrad \cdot \BJ + j \omega \spacegrad \cdot \BD
= \spacegrad \cdot \BJ + j \omega \rho,
\end{dmath}
%
These are the phasor forms of the continuity equations
\index{continuity equation}

\begin{subequations}
\begin{dmath}\label{eqn:chapter3Notes:220}
\spacegrad \cdot \BM = - j \omega \rho_\txtm
\end{dmath}
\begin{dmath}\label{eqn:chapter3Notes:240}
\spacegrad \cdot \BJ = -j \omega \rho.
\end{dmath}
\end{subequations}

\paragraph{Integral forms}
\index{Maxwell's equation!integral forms}

The integral forms of Maxwell's equations follow from Stokes' theorem and the divergence theorems.  Stokes' theorem is a relation between the integral of the curl and the outwards normal differential area element of a surface, to the boundary of that surface, and applies to any surface with that boundary
\index{Stokes' theorem}
\index{divergence theorem}
%
\begin{dmath}\label{eqn:chapter3Notes:260}
\iint
d\BA \cdot \lr{\spacegrad \cross \Bf}
= \ointctrclockwise \Bf \cdot d\Bl.
\end{dmath}
%
The divergence theorem, a special case of the general Stokes' theorem is
%
\begin{dmath}\label{eqn:chapter3Notes:280}
\iiint_{V} \spacegrad \cdot \Bf \, dV
= \iint_{\partial V} \Bf \cdot d\BA,
\end{dmath}
%
where the integral is over the surface of the volume, and the area element of the bounding integral has an outwards normal orientation.

See \citep{gabookI:stokesTheoremGeometricAlgebra} for a derivation of this and various generalizations.

Applying these to \cref{eqn:chapter3Notes:99} gives

\begin{subequations}
\begin{dmath}\label{eqn:chapter3Notes:320}
\ointctrclockwise d\Bl \cdot \BE = -
\iint d\BA \cdot \lr{
\BM + j \omega \BB
}
\end{dmath}
\begin{dmath}\label{eqn:chapter3Notes:340}
\ointctrclockwise d\Bl \cdot \BH =
\iint d\BA \cdot \lr{
\BJ + j \omega \BD
}
\end{dmath}
\begin{dmath}\label{eqn:chapter3Notes:360}
\iint_{\partial V} d\BA \cdot \BD = \iiint \rho \,dV
\end{dmath}
\begin{dmath}\label{eqn:chapter3Notes:380}
\iint_{\partial V} d\BA \cdot \BB = \iiint \rho_\txtm \,dV
\end{dmath}
\end{subequations}

\section{Constitutive relations}
\index{constitutive relations}

For linear isotropic homogeneous materials, the following constitutive relations apply
\index{linear media}
\index{homogeneous media}

\begin{itemize}
\item \( \BD = \epsilon \BE \)
\item \( \BB = \mu \BH \)
\item \( \BJ = \sigma \BE \), Ohm's law.
\end{itemize}

where

\begin{itemize}
\item \( \epsilon = \epsilon_r \epsilon_0\), is the permittivity (\si{F/m}, \si{Farads/meter} ).
\index{permittivity}
\item \( \mu = \mu_r \mu_0 \), is the permeability (\si{H/m}, \si{Henries/meter}), \( \mu_0 = 4 \pi \times 10^{-7} \).
\item \( \sigma \), is the conductivity (\( \inv{\Omega \si{m}}\), where \( 1/\Omega \) is a Siemens.)
\end{itemize}

In AM radio, will see ferrite cores with the inductors, which introduces non-unit \( \mu_r \).  This is to increase the radiation resistance.

\section{Boundary conditions}
\index{boundary conditions}

For good electric conductor \( \BE = 0 \).
For good magnetic conductor \( \BB = 0 \).

(more on class slides)

\section{Linear time invariant}
\index{linear time invariant}

Linear time invariant meant that the impulse response \( h(t,t') \) was a function of just the difference in times \( h(t,t') = h(t-t') \).
\index{linear time invariant}
\index{impulse response}

\section{Green's functions}
\index{Green's function}

For electromagnetic problems the impulse function sources \( \delta(\Br - \Br') \) also has a direction, and can yield any of \( E_x, E_y, E_z \).  A tensor impulse response is required.

Some overview of an approach that uses such tensor Green's functions is outlined on the slides.  It gets really messy since we require four tensor Green's functions to handle electric and magnetic current and charges.  Because of this complexity, we don't go down this path, and use potentials instead.

In \S 3.5 \citep{balanis2005antenna} and the class notes, a verification of the spherical wave form for the Helmholtz Green's function was developed.  This was much simpler than the same verification I did in \citep{phy456:helmoltzGreens}.  Part of the reason for that was that I worked in Cartesian coordinates, which made things much messier.  The other part of the reason, for treating a neighbourhood of \( \Abs{\Br - \Br'} \sim 0 \), I verified the convolution, whereas Prof. Eleftheriades argues that a verification that \( \int \lr{\spacegrad^2 + k^2} G(\Br, \Br') dV' = 1\) is sufficient.  Balanis, on the other hand, argues that knowing the solution for \( k \ne 0 \) must just be the solution for \( k = 0 \) (i.e. the Poisson solution) provided it is multiplied by the \( e^{-j k r} \) factor.
\index{Helmholtz equation!Green's function}

Note that back when I did that derivation, I used a different sign convention for the Green's function, and in QM we used a positive sign instead of the negative in \( e^{-j k r } \).

%\section{Mathematica notebooks}
%
%\input{../ece1229/mathematica.tex}

%\EndArticle
