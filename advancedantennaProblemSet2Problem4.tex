%
% Copyright � 2015 Peeter Joot.  All Rights Reserved.
% Licenced as described in the file LICENSE under the root directory of this GIT repository.
%
\makeoproblem{Infinitesimal electric dipole.}{advancedantenna:problemSet2:4}{2015 ps2, p4}{
\index{infinitesimal electric dipole}
%
Show that in the near field \( k r \rightarrow 0 \), the electric field of an infinitesimal electrical dipole of length \( l \) and current \( \BI = I \zcap \) can be derived from the field of an electric dipole moment \( \Bp = q l \zcap \). The electrostatic field of a dipole moment is given by,
%
\begin{dmath}\label{eqn:advancedantennaProblemSet2Problem4:20}
\BE = \frac{ 3 \lr{ \Bp \cdot \rcap } \rcap - \Bp}{4 \pi \epsilon_0 r^3}.
\end{dmath}
%
\index{infinitesimal electric dipole}
\index{dipole moment}
%
} % makeoproblem
%
\makeanswer{advancedantenna:problemSet2:4}{
%
To write the electrostatic field in spherical coordinates, first note that
%
\begin{dmath}\label{eqn:advancedantennaProblemSet2Problem4:40}
\Bp \cdot \rcap
= q l \zcap \cdot \lr{ \cos\phi \sin\theta, \sin\phi \sin\theta, \cos\theta }
= q l \cos\theta,
\end{dmath}
%
so the electrostatic dipole field is
%
\begin{dmath}\label{eqn:advancedantennaProblemSet2Problem4:60}
\BE
= q l \frac{3 \cos\theta \rcap - \zcap}{4 \pi \epsilon_0 r^3}.
\end{dmath}
%
To calculate the radial component of this field, note that
%
\begin{dmath}\label{eqn:advancedantennaProblemSet2Problem4:80}
\lr{3 \cos\theta \rcap - \zcap} \cdot \rcap
=
3 \cos\theta - \zcap \cdot \rcap
=
2 \cos\theta,
\end{dmath}
%
so
%
\begin{dmath}\label{eqn:advancedantennaProblemSet2Problem4:100}
E_r = q l \frac{\cos\theta}{2 \pi \epsilon_0 r^3}.
\end{dmath}
%
For the \( \theta \) component, noting that \( \thetacap = \lr{\cos\theta \cos\phi, \cos\theta \sin\phi, -\sin\theta} \),
%
\begin{dmath}\label{eqn:advancedantennaProblemSet2Problem4:120}
\lr{ 3 \cos\theta \rcap - \zcap } \cdot \thetacap
=
- \zcap \cdot \thetacap
=
\sin\theta,
\end{dmath}
%
so
%
\begin{dmath}\label{eqn:advancedantennaProblemSet2Problem4:140}
E_\theta = q l \frac{\sin\theta}{4 \pi \epsilon_0 r^3}.
\end{dmath}
%
Finally the \( E_\phi \) component of this electrostatic field is zero since
%
\begin{dmath}\label{eqn:advancedantennaProblemSet2Problem4:160}
\lr{ 3 \cos \theta \rcap - \zcap } \cdot \phicap = 0,
\end{dmath}
%
because \( \rcap \) and \( \phicap \) are orthonormal, and \( \phicap \) lies in the x-y plane, always perpendicular to \( \zcap \).
%
\paragraph{Current for the dipole configuration.}
\index{phasor!dipole current}
%
A set of equal magnitude oscillating charges \( \pm q(t) \) separated by distance \( l \), have the phasor representation
%
\begin{dmath}\label{eqn:advancedantennaProblemSet2Problem4:280}
q(t) = q e^{j \omega t}.
\end{dmath}
%
The dipole moment associated with such a charge distribution is
%
\begin{dmath}\label{eqn:advancedantennaProblemSet2Problem4:300}
\Bp(t) = \lr{ +q \frac{l}{2} + (-q) \lr{\frac{-l}{2}} } e^{j \omega t} \zcap
= q l e^{j \omega t} \zcap.
\end{dmath}
%
This has the desired dipole moment magnitude \( \Bp = q l \zcap \).  The current for this dipole configuration is
%
\begin{dmath}\label{eqn:advancedantennaProblemSet2Problem4:320}
I(t)
= \frac{dq(t)}{dt}
= j \omega q e^{j \omega t},
\end{dmath}
%
allowing a phasor identification for the current magnitude
%
\begin{dmath}\label{eqn:advancedantennaProblemSet2Problem4:340}
I_0 = j \omega q.
\end{dmath}
%
\paragraph{Near field equivalence.}
\index{dipole!near field}
%
The near field electric field equations derived from the magnetic vector potential are expressed in terms of \( I_0 \), not \( q \).  Since the ratio of charge to permittivity is
\index{permittivity}
%
\begin{dmath}\label{eqn:advancedantennaProblemSet2Problem4:180}
\frac{q}{\epsilon_0}
=
\frac{I_0}{j \omega \epsilon_0}
=
-j I_0 \frac{ 1}{ k c \epsilon_0 }
=
-j I_0 \frac{ \sqrt{\mu_0 \epsilon_0} }{ k \epsilon_0 }
=
-j I_0 \frac{ \eta }{k}
,
\end{dmath}
%
the electric field components
\cref{eqn:advancedantennaProblemSet2Problem4:100} and \cref{eqn:advancedantennaProblemSet2Problem4:140} calculated from the dipole moment take the form
%
\begin{subequations}
\label{eqn:advancedantennaProblemSet2Problem4:200}
\begin{dmath}\label{eqn:advancedantennaProblemSet2Problem4:220}
E_r = -j \eta I_0 l \frac{\cos\theta}{2 \pi k r^3}
\end{dmath}
\begin{dmath}\label{eqn:advancedantennaProblemSet2Problem4:240}
E_\theta = -j \eta I_0 l \frac{\sin\theta}{4 \pi k r^3}
\end{dmath}
\begin{dmath}\label{eqn:advancedantennaProblemSet2Problem4:260}
E_\phi = 0.
\end{dmath}
\end{subequations}
%
This reproduces the near field electric field equations for a vertical infinitesimal dipole (4.20a,b) from the text \citep{balanis2005antenna} in the limit \( k r \rightarrow 0 \).
}
