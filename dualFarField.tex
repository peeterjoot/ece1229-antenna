%
% Copyright � 2015 Peeter Joot.  All Rights Reserved.
% Licenced as described in the file LICENSE under the root directory of this GIT repository.
%
%\input{../blogpost.tex}
%\renewcommand{\basename}{dualFarField}
%\renewcommand{\dirname}{notes/ece1229/}
%%\newcommand{\dateintitle}{}
%%\newcommand{\keywords}{}
%
%\input{../peeter_prologue_print2.tex}
%
%\beginArtNoToc
%
%\generatetitle{Duality transformation of the far field fields.}
\section{Duality transformation of the far field fields}
\index{duality transformation}
\index{far field}
%\label{chap:dualFarField}
We've seen that the far field electric and magnetic fields associated with a magnetic vector potential were
%
\begin{subequations}
\label{eqn:dualFarField:20}
\begin{dmath}\label{eqn:dualFarField:40}
\BE = -j \omega \Proj_\T \BA,
\end{dmath}
\begin{dmath}\label{eqn:dualFarField:60}
\BH = \inv{\eta} \kcap \cross \BE.
\end{dmath}
\end{subequations}
%
What does \( \BH \) look like in terms of \( \BA \)?  Expanding the rejection of the radial component answers that
%
\begin{dmath}\label{eqn:dualFarField:140}
\BH
= -\frac{j \omega}{\eta} \kcap \cross \lr{ \BA - \lr{\BA \cdot \kcap} \kcap }.
\end{dmath}
%
The \( \kcap \) crossed terms are killed, leaving
%
\begin{dmath}\label{eqn:dualFarField:160}
\BH
= -\frac{j \omega}{\eta} \kcap \cross \BA.
\end{dmath}
%
It's worth a quick note that the duality transformation for this, referring to \citep{balanis2005antenna} \texttabref{3.2}, is
%
\begin{subequations}
\label{eqn:dualFarField:80}
\begin{dmath}\label{eqn:dualFarField:100}
\BH = -j \omega \Proj_\T \BF
\end{dmath}
\begin{equation}\label{eqn:dualFarField:120}
\BE = \eta \kcap \cross \BH = j \omega \eta \kcap \cross \BF.
\end{equation}
\end{subequations}
%
These show explicitly that neither the electric or magnetic far field have any radial component, matching with intuition for transverse propagation of the fields.
%
%\EndArticle
