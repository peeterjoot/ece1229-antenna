%
% Copyright � 2015 Peeter Joot.  All Rights Reserved.
% Licenced as described in the file LICENSE under the root directory of this GIT repository.
%
%
% This follows example 2.6:
%
\makeoproblem{Max power density and directivity.}{advancedantenna:problemSet1:1}{2015 ps1, p1}{
\index{directivity}
\index{max power density}
The power radiated by a lossless antenna is 10 \si{W}. The corresponding radiation intensity is
given by,
\index{power density}
\index{directivity}
%
\begin{equation}\label{eqn:advancedantennaProblemSet1Problem1:20}
U = B_0 \cos^3 \theta, \qquad 0 \le \theta < \pi/2, 0 \le \phi < 2 \pi.
\end{equation}
%
Calculate
\makesubproblem{}{advancedantenna:problemSet1:1a}
the maximum power density at a distance of 1 \si{km}.
\makesubproblem{}{advancedantenna:problemSet1:1b}
the directivity of the antenna (dimensionless and \si{dB}).
%
} % makeoproblem
%
\makeanswer{advancedantenna:problemSet1:1}{
\makeSubAnswer{}{advancedantenna:problemSet1:1a}
%
The radiated power density is
\index{radiated power density}
%
\begin{equation}\label{eqn:advancedantennaProblemSet1Problem1:40}
W_\txtr(r, \theta) = \frac{U}{r^2} = \frac{B_0 \cos^3 \theta}{r^2}, \qquad \si{W/m^2},
\end{equation}
%
with the maximum at \( \theta = 0 \) of
%
\begin{dmath}\label{eqn:advancedantennaProblemSet1Problem1:60}
\evalbar{W_\txtr(r)}{\tmax} = \frac{B_0}{r^2}, \qquad \si{W/m^2}.
\end{dmath}
%
Since the average power density is
\index{average power density}
%
\begin{dmath}\label{eqn:advancedantennaProblemSet1Problem1:80}
P_\tav
= \oiint U d\Omega
= 2\pi B_0 \int_0^{\pi/2} \cos^3 \theta \sin\theta d\theta
= -2\pi B_0 \int_0^{\pi/2} \cos^3 \theta d\cos\theta
= 2\pi B_0 \evalrange{\frac{\cos^4\theta}{4}}{\pi/2}{0}
= \frac{\pi B_0}{2}
= 10 \qquad (\si{W}),
\end{dmath}
%
the constant \( B_0 = 20/\pi \approx 6.37 \si{W} \), so the maximum power density at 1 \si{km} is
%
\begin{equation}\label{eqn:advancedantennaProblemSet1Problem1:100}
\evalbar{W_\txtr(1 \,\si{km})}{\tmax} = \frac{20}{\pi} \times 10^{-6} \approx 6.37 \times 10^{-6} \qquad (\si{W/m^2}).
\end{equation}
%
\makeSubAnswer{}{advancedantenna:problemSet1:1b}
%
The maximum directivity of the antenna is
\index{maximum directivity}
%
\begin{dmath}\label{eqn:advancedantennaProblemSet1Problem1:120}
D_0
=
4 \pi \frac{U_\tmax}{P_\trad}
=
4 \pi \frac{U_\tmax}{P_\tav}
=
\frac{4 \cancel{\pi B_0}}{\cancel{\pi B_0}/2}
= 8,
\end{dmath}
%
so the directivity is
%
\begin{equation}\label{eqn:advancedantennaProblemSet1Problem1:160}
D = 8 \cos^3 \theta.
\end{equation}
%
In \si{dB} the maximum directivity is
%
\begin{equation}\label{eqn:advancedantennaProblemSet1Problem1:140}
D_0 = 10 \log_{10} 8 = 9 \, \si{dB}.
\end{equation}
%
}
