%
% Copyright � 2015 Peeter Joot.  All Rights Reserved.
% Licenced as described in the file LICENSE under the root directory of this GIT repository.
%
%\input{../blogpost.tex}
%\renewcommand{\basename}{chapter2Notes}
%\renewcommand{\dirname}{notes/ece1229/}
%\newcommand{\dateintitle}{}
%\newcommand{\keywords}{}
%
%\input{../peeter_prologue_print2.tex}
%
%\usepackage{ece1229}
%
%\beginArtNoToc
%
%\generatetitle{Fundamental parameters of antennas}
%\label{chap:chapter2Notes}
%
%This is my first set of notes for the UofT course ECE1229, Advanced Antenna Theory, taught by Prof. Eleftheriades, covering \chaptext 2 \citep{balanis2005antenna} content.
%
%Unlike most of the other classes I have taken, I am not attempting to take comprehensive notes for this class.  The class is taught on slides that match the textbook so closely, there is little value to me taking notes that just replicate the text.  Instead, I am annotating my copy of textbook with little details instead.  My usual notes collection for the class will contain musings of details that were unclear, or in some cases, details that were provided in class, but are not in the text (and too long to pencil into my book.)
%
\section{Poynting vector.}
\index{Poynting vector}
%
The Poynting vector was written in an unfamiliar form
%
\begin{dmath}\label{eqn:chapter2Notes:560}
\bcW = \bcE \cross \bcH.
\end{dmath}
%
I can roll with the use of a different symbol (i.e. not \(\BS\)) for the Poynting vector, but I'm used to seeing a \( \ifrac{c}{4\pi} \) factor (\citep{landau1980classical} and \citep{jackson1975cew}).  I remembered something like that in SI units too, so was slightly confused not to see it here.

Per \citep{griffiths1999introduction} that something is a \( \mu_0 \), as in
%
\begin{dmath}\label{eqn:chapter2Notes:580}
\bcW = \inv{\mu_0} \bcE \cross \bcB.
\end{dmath}
%
Note that the use of \( \bcH \) instead of \( \bcB \) is what wipes out the requirement for the \( \ifrac{1}{\mu_0} \) term since \( \bcH = \bcB/\mu_0 \), assuming linear media, and no magnetization.
\index{linear media}
\index{magnetization}
%
\section{Typical far-field radiation intensity.}
\index{radiation intensity}
\index{far field}
%
It was mentioned that
%
\begin{dmath}\label{eqn:advancedantennaL1:20}
U(\theta, \phi)
=
\frac{r^2}{2 \eta_0} \Abs{ \BE( r, \theta, \phi) }^2
=
\frac{1}{2 \eta_0} \lr{ \Abs{ E_\theta(\theta, \phi) }^2 + \Abs{ E_\phi(\theta, \phi) }^2},
\end{dmath}
%
where the intrinsic impedance of free space is
\index{intrinsic impedance}
%
\begin{dmath}\label{eqn:advancedantennaL1:480}
\eta_0 = \sqrt{\frac{\mu_0}{\epsilon_0}} = 377 \Omega.
\end{dmath}
%
(this is also \texteqnref{2-19} in the text.)
%
To get an understanding where this comes from, consider the far field radial solutions to the electric and magnetic dipole problems, which have the respective forms (from \citep{griffiths1999introduction}) of
%
\begin{subequations}
\begin{dmath}\label{eqn:chapter2Notes:740}
\begin{aligned}
\bcE &= -\frac{\mu_0 p_0 \omega^2 }{4 \pi } \frac{\sin\theta}{r} \cos\lr{w t - k r} \thetacap \\
\bcB &= -\frac{\mu_0 p_0 \omega^2 }{4 \pi c} \frac{\sin\theta}{r} \cos\lr{w t - k r} \phicap \\
\end{aligned}
\end{dmath}
\begin{dmath}\label{eqn:chapter2Notes:760}
\begin{aligned}
\bcE &= \frac{\mu_0 m_0 \omega^2 }{4 \pi c} \frac{\sin\theta}{r} \cos\lr{w t - k r} \phicap \\
\bcB &= -\frac{\mu_0 m_0 \omega^2 }{4 \pi c^2} \frac{\sin\theta}{r} \cos\lr{w t - k r} \thetacap.
\end{aligned}
\end{dmath}
\end{subequations}
%
In neither case is there a component in the direction of propagation, and in both cases (using \( \mu_0 \epsilon_0 = 1/c^2\))
%
\begin{dmath}\label{eqn:chapter2Notes:780}
\Abs{\bcH}
= \frac{\Abs{\bcE}}{\mu_0 c}
= \Abs{\bcE} \sqrt{\frac{\epsilon_0}{\mu_0}}
= \inv{\eta_0}\Abs{\bcE} .
\end{dmath}
%
Note that the signs of \(\bcE\) vs. \(\bcB\) in \cref{eqn:chapter2Notes:740} and \cref{eqn:chapter2Notes:760} and are determined by the far field relation \( \BE = c \BB \cross \rcap \) (see: \texteqnref{9.19,9.39} \citep{jackson1975cew}).  The effect of dependency is that the Poynting vector will be radial, which will be seen below.
%
A superposition of the phasors for such dipole fields, in the far field, will have the form
\index{superposition}
%
\begin{dmath}\label{eqn:chapter2Notes:800}
\begin{aligned}
\BE &= \inv{r} \lr{ E_\theta(\theta, \phi) \thetacap + E_\phi(\theta, \phi) \phicap } \\
\BB &= \inv{r c} \lr{ E_\theta(\theta, \phi) \thetacap - E_\phi(\theta, \phi) \phicap },
\end{aligned}
\end{dmath}
%
with a corresponding time averaged Poynting vector
\index{Poynting vector!time average}
%
\begin{dmath}\label{eqn:chapter2Notes:820}
\BW_\tav
= \inv{2 \mu_0} \BE \cross \BB^\conj
=
\inv{2 \mu_0 c r^2}
\lr{ E_\theta \thetacap + E_\phi \phicap } \cross
\lr{ E_\theta^\conj \thetacap - E_\phi^\conj \phicap }
=
\frac{\thetacap \cross \phicap}{2 \mu_0 c r^2}
\lr{ \Abs{E_\theta}^2 + \Abs{E_\phi}^2 }
=
\frac{\rcap}{2 \eta_0 r^2}
\lr{ \Abs{E_\theta}^2 + \Abs{E_\phi}^2 },
\end{dmath}
%
verifying \cref{eqn:advancedantennaL1:20} for a superposition of electric and magnetic dipole fields.  This can likely be shown for more general fields too.
\index{superposition}
%
\section{Field plots.}
%
We can plot the fields, or intensity (or log plots in \si{dB} of these).
\index{dB}
It is pointed out in
\citep{griffiths1999introduction}
that when there is \( r \) dependence these plots are done by considering the values of at fixed \( r \).
%
The field plots are conceptually the simplest, since that vector parameterizes a surface.  Any such radial field with magnitude \( f(r, \theta, \phi) \) can be plotted in Mathematica in the \( \phi = 0 \) plane at \( r = r_0 \), or in 3D (respectively, but also at \( r = r_0\)) with code like \cref{eqn:chapter2Notes:600}
\index{Mathematica}
%
%\mathematicaListing{
% can't use the newcommand with pure-function parameters (#1 is also a latex beastie).
\begin{figure}
\caption{Plot methods for fields and intensities.}\label{eqn:chapter2Notes:600}
\begin{shaded}
\begin{mat}
rcap = {Cos[#], Sin[#]} & ;
scap = {Sin[#1] Cos[#2], Sin[#1] Sin[#2], Cos[#1]} & ;
ParametricPlot[ f[$r_0$, $\theta$, 0] rcap, {$\theta$, 0, Pi}]
ParametricPlot3D[ f[$r_0$, $\theta$, $\phi$] scap, {$\theta$, 0, Pi}, {$\phi$, 0, 2 Pi}]
\end{mat}
\end{shaded}
\end{figure}
\index{ParametricPlot}
\index{ParametricPlot3D}
%
Intensity plots can use the same code, with the only difference being the interpretation.  The surface doesn't represent the value of a vector valued radial function, but is the magnitude of a scalar valued function evaluated at \( f( r_0, \theta, \phi) \).
%
%For example, to plot an average power density with the form \( \BP_\trad = \rcap \sin^2\theta /r^2 \), at the point \( (r,\theta, \phi) \),
%as sketched in the \( \phi = 0 \) plane in \cref{fig:antennaIntensityPlot:antennaIntensityPlotFig1}, and in spherical coordinates in \cref{fig:antennaIntensityPlot:antennaIntensityPlotFig2}.
%
%\imageFigure{../figures/ece1229-antenna/antennaIntensityPlotFig1}{power density components in a plane}{fig:antennaIntensityPlot:antennaIntensityPlotFig1}{0.3}
%\imageFigure{../figures/ece1229-antenna/antennaIntensityPlotFig2}{power density components in spherical coordinates}{fig:antennaIntensityPlot:antennaIntensityPlotFig2}{0.3}
%
%Noting that in the plane
%
%\begin{dmath}\label{eqn:chapter2Notes:520}
%\rcap = \lr{\cos\theta, \sin\theta},
%\end{dmath}
%
%and in 3D
%
%\begin{dmath}\label{eqn:chapter2Notes:540}
%\rcap = \lr{\sin\theta \cos\phi, \sin\theta\sin\phi, \cos\theta}.
%\end{dmath}
%
%A plot of nt to
%For the choice of \( \BP_\trad \) above \( U \) happens to not depend on \( r \), so the surface parameterized by \( \rcap U = r^2 \BP_\trad \) visualizes the intensity at all points on the surface.
%For more general \( r \) dependence, it is pointed out in
%\citep{griffiths1999introduction}
%that such antenna radiation intensity plots
%can be made by considering the values of \( U \) at fixed \( r \).

The surfaces for
\( U = \cos\theta, \cos^2\theta \)
and for
\( U = \sin\theta, \sin^2\theta \)
in the plane are parametrically plotted in
\cref{fig:SineAndSinSq:SineAndSinSqFig3}, and for cosines in \cref{fig:CoSineAndCoSineSq:CoSineAndCoSineSqFig1} to compare with textbook figures.
%
\mathImageTwoFigures{../figures/ece1229-antenna/CoSineAndCoSineSqFig1}{../figures/ece1229-antenna/SineAndSinSqFig3}{Cosinusoidal and sinusoidal radiation intensities.}{fig:SineAndSinSq:SineAndSinSqFig3}{scale=0.3}{sphericalPlot3d.nb}
%
%\mathImageFigure{../figures/ece1229-antenna/CoSineAndCoSineSqFig1}{Cosinusoidal radiation intensities}{fig:CoSineAndCoSineSq:CoSineAndCoSineSqFig1}{0.3}{sphericalPlot3d.nb}
%\mathImageFigure{../figures/ece1229-antenna/SineAndSinSqFig3}{Sinusoidal radiation intensities}{fig:SineAndSinSq:SineAndSinSqFig3}{0.3}{sphericalPlot3d.nb}
%
Three dimensional visualizations of \( U = \sin^2 \theta\) and
\( U = \cos^2 \theta\) can be found in \cref{fig:SineSq3D:SineSq3DFig4}
% and
%\cref{fig:CoSineSq3D:CoSineSq3DFig2} respectively.
Even for such simple functions these look pretty cool.
%
\mathImageTwoFigures{../figures/ece1229-antenna/SineSq3DFig4}{../figures/ece1229-antenna/CoSineSq3DFig2}{Square sinusoidal and cosinusoidal radiation intensity.}{fig:SineSq3D:SineSq3DFig4}{scale=0.3}{sphericalPlot3d.nb}
%\mathImageFigure{../figures/ece1229-antenna/SineSq3DFig4}{Square sinusoidal radiation intensity}{fig:SineSq3D:SineSq3DFig4}{0.3}{sphericalPlot3d.nb}
%\mathImageFigure{../figures/ece1229-antenna/CoSineSq3DFig2}{Square cosinusoidal radiation intensity}{fig:CoSineSq3D:CoSineSq3DFig2}{0.3}{sphericalPlot3d.nb}
\index{radiation intensity}
%
%\section{\texorpdfstring{\si{dB}}{dB} vs \texorpdfstring{\si{dBi}}{dBi}}
%\section{\si{dB} vs \si{dBi}}
\section{dB vs dBi.}
\index{dBi}
%
Note that \si{dBi} is used to indicate that the gain is with respect to an ``isotropic'' radiator.
\index{isotropic radiator}
This is detailed more in \citep{digidbivsdbd}.
%
\section{Trig integrals.}
%
%Tables \cref{tab:chapter2Notes:1} and \cref{tab:chapter2Notes:2} produced with \nbref{tableOfTrigIntegrals.nb}
%have some of the sine and cosine integrals that are pervasive in this chapter.
%
%\captionedTable{\(\int_0^{\pi/2} \sin^r\theta \cos^s\theta d\theta\)}{tab:chapter2Notes:1}{
%\begin{tabular}{|l|l|l|l|l|l|l|}
%\hline
%   & \multicolumn{6}{|l|}{r} \\ \hline
% s & 0 & 1 & 2 & 3 & 4 & 5 \\ \hline \hline
% 0 & \(\ifrac{\pi}{2}\) & \(1\) & \(\ifrac{\pi}{4}\) & \(\ifrac{2}{3}\) & \(\ifrac{3 \pi}{16}\) & \(\ifrac{8}{15} \) \\ \hline
% 1 & \(1\) & \(\ifrac{1}{2}\) & \(\ifrac{1}{3}\) & \(\ifrac{1}{4}\) & \(\ifrac{1}{5}\) & \(\ifrac{1}{6} \) \\ \hline
% 2 & \(\ifrac{\pi}{4}\) & \(\ifrac{1}{3}\) & \(\ifrac{\pi}{16}\) & \(\ifrac{2}{15}\) & \(\ifrac{\pi}{32}\) & \(\ifrac{8}{105} \) \\ \hline
% 3 & \(\ifrac{2}{3}\) & \(\ifrac{1}{4}\) & \(\ifrac{2}{15}\) & \(\ifrac{1}{12}\) & \(\ifrac{2}{35}\) & \(\ifrac{1}{24} \) \\ \hline
% 4 & \(\ifrac{3 \pi}{16}\) & \(\ifrac{1}{5}\) & \(\ifrac{\pi}{32}\) & \(\ifrac{2}{35}\) & \(\ifrac{3 \pi}{256}\) & \(\ifrac{8}{315} \) \\ \hline
% 5 & \(\ifrac{8}{15}\) & \(\ifrac{1}{6}\) & \(\ifrac{8}{105}\) & \(\ifrac{1}{24}\) & \(\ifrac{8}{315}\) & \(\ifrac{1}{60} \) \\ \hline
%\end{tabular}
%}

%FIXME: these table headers are confusing.  \( s = 0 \) is the first column.  i.e. \( \int_0^\pi \sin^3\theta d\theta  = 4/3 \).
%
%\captionedTable{\(\int_0^{\pi} \sin^r\theta \cos^s\theta d\theta\)}{tab:chapter2Notes:2}{
%\begin{tabular}{|l|l|l|l|l|l|l|}
%\hline
%   & \multicolumn{6}{|l|}{r} \\ \hline
% s & 0 & 1 & 2 & 3 & 4 & 5 \\ \hline \hline
% 0 & \(\pi \) & \(0\) & \(\ifrac{\pi }{2}\) & \(0\) & \(\ifrac{3 \pi }{8}\) & \(0 \) \\ \hline
% 1 & \(2\) & \(0\) & \(\ifrac{2}{3}\) & \(0\) & \(\ifrac{2}{5}\) & \(0 \) \\ \hline
% 2 & \(\ifrac{\pi }{2}\) & \(0\) & \(\ifrac{\pi }{8}\) & \(0\) & \(\ifrac{\pi }{16}\) & \(0 \) \\ \hline
% 3 & \(\ifrac{4}{3}\) & \(0\) & \(\ifrac{4}{15}\) & \(0\) & \(\ifrac{4}{35}\) & \(0 \) \\ \hline
% 4 & \(\ifrac{3 \pi }{8}\) & \(0\) & \(\ifrac{\pi }{16}\) & \(0\) & \(\ifrac{3 \pi }{128}\) & \(0 \) \\ \hline
% 5 & \(\ifrac{16}{15}\) & \(0\) & \(\ifrac{16}{105}\) & \(0\) & \(\ifrac{16}{315}\) & \(0 \) \\ \hline
%\end{tabular}
%}
%
\captionedTable{\(\int_0^{\pi/2.} \sin^n\theta d\theta = \int_0^{\pi/2} \cos^n\theta d\theta\)}{tab:chapter2Notes:4}{
\begin{tabular}{|l|l|l|l|l|l|l|l|}
\hline
n & 1 & 2 & 3 & 4 & 5 & 6 & 7 \\
\hline
& \(1\)&\(\ifrac{\pi}{4}\)&\(\ifrac{2}{3}\)&\(\ifrac{3 \pi }{16}\)&\(\ifrac{8}{15}\)&\(\ifrac{5 \pi }{32}\)&\(\ifrac{16}{35}\)
\\ \hline
\end{tabular}
}
%
\captionedTable{\(\int_0^{\pi.} \sin^n\theta d\theta \)}{tab:chapter2Notes:5}{
\begin{tabular}{|l|l|l|l|l|l|l|l|}
\hline
n & 1 & 2 & 3 & 4 & 5 & 6 & 7 \\
\hline
& \(2\)&\(\ifrac{\pi }{2}\)&\(\ifrac{4}{3}\)&\(\ifrac{3 \pi }{8}\)&\(\ifrac{16}{15}\)&\(\ifrac{5 \pi }{16}\)&\(\ifrac{32}{35}\)
\\ \hline
\end{tabular}
}
%
\captionedTable{\(\int_0^{\pi.} \cos^n\theta d\theta \)}{tab:chapter2Notes:6}{
\begin{tabular}{|l|l|l|l|l|l|l|l|}
\hline
n & 1 & 2 & 3 & 4 & 5 & 6 & 7 \\
\hline
& \(0\)&\(\ifrac{\pi }{2}\)&\(0\)&\(\ifrac{3 \pi }{8}\)&\(0\)&\(\ifrac{5 \pi }{16}\)&\(0\)
\\ \hline
\end{tabular}
}
%
\section{Polarization vectors.}
\index{polarization vector}
%
The text introduces polarization vectors \( \rhocap \) , but doesn't spell out their form.  Consider a plane wave field of the form
%
\begin{dmath}\label{eqn:chapter2Notes:840}
\BE
=
E_x e^{j \phi_x} e^{j \lr{ \omega t - k z }} \xcap
+
E_y e^{j \phi_y} e^{j \lr{ \omega t - k z }} \ycap.
\end{dmath}
%
The \( x, y \) plane directionality of this phasor can be written
%
\begin{dmath}\label{eqn:chapter2Notes:860}
\Brho =
E_x e^{j \phi_x} \xcap
+
E_y e^{j \phi_y} \ycap,
\end{dmath}
%
so that
%
\begin{dmath}\label{eqn:chapter2Notes:880}
\BE = \Brho e^{j \lr{ \omega t - k z }}.
\end{dmath}
%
Separating this direction and magnitude into factors
%
\begin{dmath}\label{eqn:chapter2Notes:900}
\Brho = \Abs{\BE} \rhocap,
\end{dmath}
%
allows the phasor to be expressed as
\index{phasor}
%
\begin{dmath}\label{eqn:chapter2Notes:920}
\BE = \rhocap \Abs{\BE} e^{j \lr{ \omega t - k z }}.
\end{dmath}
%
As an example, suppose that \( E_x = E_y \), and set \( \phi_x = 0 \).  Then
%
\begin{dmath}\label{eqn:chapter2Notes:940}
\rhocap = \xcap + \ycap e^{j \phi_y}.
\end{dmath}
%
%
% Copyright � 2015 Peeter Joot.  All Rights Reserved.
% Licenced as described in the file LICENSE under the root directory of this GIT repository.
%
%\input{../blogpost.tex}
%\renewcommand{\basename}{polarizationReview}
%\renewcommand{\dirname}{notes/ece1229/}
%%\newcommand{\dateintitle}{}
%%\newcommand{\keywords}{}
%
%\input{../peeter_prologue_print2.tex}
%
%\usepackage{macros_bm}
%
%\beginArtNoToc
%
%\generatetitle{Polarization review}
%\chapter{Polarization review}
\index{polarization}
%\label{chap:polarizationReview}
%\section{Motivation}
%\section{Guts}
%
\paragraph{Demonstrating the geometry.}
It seems worthwhile to review how a generally polarized field phasor leads to linear, circular, and elliptic geometries.

The most general field polarized in the \( x, y \) plane has the form
%
\begin{dmath}\label{eqn:polarizationReview:20}
\BE
= \lr{ \xcap a e^{j \alpha} + \ycap b e^{j \beta} } e^{j \lr{ \omega t -k z }}
= \lr{ \xcap a e^{j \lr{\alpha - \beta}/2} + \ycap b e^{j \lr{ \beta - \alpha}/2} } e^{j \lr{ \omega t -k z + \lr{\alpha + \beta}/2 }}.
\end{dmath}
%
Knowing to factor out the average phase angle above is only because I tried initially without that and things got ugly and messy.  I guessed this would help (it does).

Let \( \bcE = \Real \BE = \xcap x + \ycap y \), \( \theta = \omega t + (\alpha + \beta)/2 \), and \( \phi = (\alpha - \beta)/2 \), so that
%
\begin{dmath}\label{eqn:polarizationReview:40}
\BE
= \lr{ \xcap a e^{j \phi} + \ycap b e^{-j \phi} } e^{j \theta }.
\end{dmath}
%
The coordinates can now be read off
%
\begin{subequations}
\label{eqn:polarizationReview:50}
\begin{dmath}\label{eqn:polarizationReview:60}
\frac{x}{a} = \cos\phi \cos\theta - \sin\phi \sin\theta
\end{dmath}
\begin{dmath}\label{eqn:polarizationReview:80}
\frac{y}{b} = \cos\phi \cos\theta + \sin\phi \sin\theta,
\end{dmath}
\end{subequations}
%
or in matrix form
%
\begin{dmath}\label{eqn:polarizationReview:100}
\begin{bmatrix}
x/a \\
y/b \\
\end{bmatrix}
=
\begin{bmatrix}
\cos\phi & - \sin\phi \\
\cos\phi & \sin\phi
\end{bmatrix}
\begin{bmatrix}
\cos\theta \\
\sin\theta
\end{bmatrix}.
\end{dmath}
%
The goal is to eliminate all the \( \theta \) (i.e. time dependence), converting the parametric relationship into a conic form.
Assuming that neither \( \cos\theta \), nor \( \sin\theta \) are zero for now (those are special cases and lead to linear polarization), inverting the matrix will allow the \( \theta \) dependence to be eliminated
%
\begin{dmath}\label{eqn:polarizationReview:120}
\inv{\sin\lr{ 2\phi }}
\begin{bmatrix}
\sin\phi & \sin\phi \\
- \cos\phi & \cos\phi
\end{bmatrix}
\begin{bmatrix}
x/a \\
y/b \\
\end{bmatrix}
=
\begin{bmatrix}
\cos\theta \\
\sin\theta
\end{bmatrix}.
\end{dmath}
%
Squaring and summing both rows of these equation gives
\begin{dmath}\label{eqn:polarizationReview:140}
1
=
\inv{\sin^2 \lr{ 2\phi}}
\lr{
\sin^2\phi
\lr{
    \frac{x}{a}
   +\frac{y}{b}
}^2
+
\cos^2\phi
\lr{
   -\frac{x}{a}
   +\frac{y}{b}
}^2
}
=
\inv{\sin^2 \lr{ 2\phi}}
\lr{
    \frac{x^2}{a^2}
    +\frac{y^2}{b^2}
+2 \frac{x y}{a b} \lr{ \sin^2\phi - \cos^2\phi }
}
=
\inv{\sin^2 \lr{ 2\phi}}
\lr{
    \frac{x^2}{a^2}
    +\frac{y^2}{b^2}
-2 \frac{x y}{a b} \cos \lr{2\phi}
}
\end{dmath}
%
Time to summarize and handle the special cases.
%
\begin{enumerate}
\item To have \( \cos\phi = 0 \), the phase angles must satisfy \( \alpha - \beta = \lr{ 1 + 2 k } \pi, \, k \in \mathbb{Z} \).

For this case \cref{eqn:polarizationReview:50} reduces to
%
\begin{dmath}\label{eqn:polarizationReview:160}
-\frac{x}{a} = \frac{y}{b},
\end{dmath}
%
which is just a line.
%
\makeexample{Linear polarization.}{example:polarizationReview:1}{
%
Let \( \alpha = 0, \beta = -\pi \), so that the phasor has the value
%
\begin{dmath}\label{eqn:polarizationReview:260}
\BE = \lr{ \xcap a - \ycap b } e^{j \omega t}
\end{dmath}
%
\item For have \( \sin\phi = 0 \), the phase angles must satisfy \( \alpha - \beta = 2 \pi k, \, k \in \mathbb{Z} \).

For this case \cref{eqn:polarizationReview:50} reduces to
%
\begin{dmath}\label{eqn:polarizationReview:180}
\frac{x}{a} = \frac{y}{b},
\end{dmath}
%
also just a line.
} % example
%
\makeexample{Elliptical polarization.}{example:polarizationReview:2}{
%
Let \( \alpha = \beta = 0 \), so that the phasor has the value
%
\begin{dmath}\label{eqn:polarizationReview:280}
\BE = \lr{ \xcap a + \ycap b } e^{j \omega t}
\end{dmath}
%
\item Last is the circular and elliptically polarized case.  The system is clearly elliptically polarized if \( \cos(2 \phi) = 0\), or \( \alpha - \beta = (\pi/2)( 1 + 2 k ), k \in \mathbb{Z}\).  When that is the case and \( a = b \) also holds, the ellipse is a circle.

When the \( \cos( 2 \phi) = 0 \) condition does not hold, a rotation of coordinates
%
\begin{dmath}\label{eqn:polarizationReview:200}
\begin{bmatrix}
x \\
y
\end{bmatrix}
=
\begin{bmatrix}
\cos\mu & \sin\mu \\
-\sin\mu & \cos\mu
\end{bmatrix}
\begin{bmatrix}
u \\
v
\end{bmatrix}
\end{dmath}
%
where
%where \( \mu = \pi( 1 + 2 k )/2 \) if \( a = b \), or for \( a \ne b \)
%
\begin{dmath}\label{eqn:polarizationReview:220}
\mu = \inv{2} \tan^{-1} \lr{ \frac{ 2 \cos (\alpha - \beta)}{b - a}}
\end{dmath}
%
puts the trajectory into a standard (but messy) conic form
%
\begin{dmath}\label{eqn:polarizationReview:240}
1 = \frac{u^2}{ab} \lr{
\frac{b}{a} \cos^2 \mu
+ \frac{a}{b} \sin^2 \mu
+ \inv{2} \sin\lr{2 \mu + \alpha - \beta}
}
+
\frac{v^2}{ab} \lr{
\frac{b}{a} \sin^2 \mu
+ \frac{a}{b} \cos^2 \mu
- \inv{2} \sin\lr{2 \mu + \alpha - \beta}
}
\end{dmath}
%
It isn't obvious to me that the factors of the \( u^2, v^2 \) terms are necessarily positive, which is required for the conic to be an ellipse and not a hyperbola.
} % example
%
\makeexample{Circular polarization.}{example:polarizationReview:n}{
%
With \( a = b = E_0 \), \( \alpha = 0 \), \( \beta = \pm \pi/2 \), all the circular polarization conditions are met, leaving the phasor with values
%
\begin{dmath}\label{eqn:polarizationReview:300}
\BE = E_0 \lr{ \xcap \pm j \ycap } e^{j \omega t}
\end{dmath}
} % example
%
\end{enumerate}
%
%\EndNoBibArticle

%
\section{Phasor power.}
\index{power!phasor}
%
In \S 2.13 the phasor power is written as
%
\begin{dmath}\label{eqn:chapter2Notes:620}
I^2 R/2,
\end{dmath}
%
where \( I, R \) are the magnitudes of phasors in the circuit.

I vaguely recall this relation, but had to refer back to
\citep{irwin2007bec}
for the details.
This relation expresses average power over a period associated with the frequency of the phasor
\index{power!average}
%
\begin{dmath}\label{eqn:chapter2Notes:640}
P
= \inv{T} \int_{t_0}^{t_0 + T} p(t) dt
= \inv{T} \int_{t_0}^{t_0 + T} \Abs{\BV} \cos\lr{ \omega t + \phi_V } \Abs{\BI} \cos\lr{ \omega t + \phi_I} dt
= \inv{T} \int_{t_0}^{t_0 + T} \Abs{\BV} \Abs{\BI}
\lr{
   \cos\lr{ \phi_V - \phi_I } + \cos\lr{ 2 \omega t + \phi_V + \phi_I}
}
dt
= \inv{2} \Abs{\BV} \Abs{\BI} \cos\lr{ \phi_V - \phi_I }.
\end{dmath}
%
Introducing the impedance for this circuit element
\index{impedance}
%
\begin{dmath}\label{eqn:chapter2Notes:660}
\BZ = \frac{ \Abs{\BV} e^{j\phi_V} }{ \Abs{\BI} e^{j\phi_I} } = \frac{\Abs{\BV}}{\Abs{\BI}} e^{j\lr{\phi_V - \phi_I}},
\end{dmath}
%
this average power can be written in phasor form
%
\begin{dmath}\label{eqn:chapter2Notes:680}
\BP = \inv{2} \Abs{\BI}^2 \BZ,
\end{dmath}
%
with
\begin{dmath}\label{eqn:chapter2Notes:700}
P = \Real \BP.
\end{dmath}
%
Observe that we have to be careful to use the absolute value of the current phasor \( \BI \), since \( \BI^2 \) differs in phase from \( \Abs{\BI}^2 \).  This explains the conjugation in the \citep{irwin2007bec} definition of complex power, which had the form
\index{power!complex}
%
\begin{dmath}\label{eqn:chapter2Notes:720}
\BS = \BV_\trms \BI^\conj_\trms.
\end{dmath}
%
\section{Radar cross section examples.}
\index{radar cross section}
%
\paragraph{Flat plate.}
\index{RCS!flat plate}
%
\begin{dmath}\label{eqn:chapter2Notes:960}
\sigma_\tmax = \frac{4 \pi \lr{L W}^2}{\lambda^2}
\end{dmath}
%
%\cref{fig:RCSsquareGeometry:RCSsquareGeometryFig1}.
\imageFigure{../figures/ece1229-antenna/RCSsquareGeometryFig1}{Square geometry for RCS example.}{fig:RCSsquareGeometry:RCSsquareGeometryFig1}{0.1}
%
\paragraph{Sphere.}
\index{RCS!sphere}
%
In the optical limit the radar cross section for a sphere
\index{optical limit}
\index{radar cross section}
%
%\cref{fig:RCSsphereGeometry:RCSsphereGeometryFig3}.
\imageFigure{../figures/ece1229-antenna/RCSsphereGeometryFig3}{Sphere geometry for RCS example.}{fig:RCSsphereGeometry:RCSsphereGeometryFig3}{0.1}
%
\begin{dmath}\label{eqn:chapter2Notes:980}
\sigma_\tmax = \pi r^2
\end{dmath}
%
Note that this is smaller than the physical area \( 4 \pi r^2 \).
%
\paragraph{Cylinder.}
\index{RCS!cylinder}
%
%\cref{fig:RCScylinderGeometry:RCScylinderGeometryFig1}.
\imageFigure{../figures/ece1229-antenna/RCScylinderGeometryFig1}{Cylinder geometry for RCS example.}{fig:RCScylinderGeometry:RCScylinderGeometryFig1}{0.1}
%
\begin{dmath}\label{eqn:chapter2Notes:1000}
\sigma_\tmax = \frac{ 2 \pi r h^2}{\lambda}
\end{dmath}
%
\paragraph{Tridedral corner reflector.}
\index{RCS!corner reflector}
%
%\cref{fig:trihedralCornerReflector:trihedralCornerReflectorFig6}.
\imageFigure{../figures/ece1229-antenna/trihedralCornerReflectorFig6}{Trihedral corner reflector geometry for RCS example.}{fig:trihedralCornerReflector:trihedralCornerReflectorFig6}{0.1}
%
\begin{dmath}\label{eqn:chapter2Notes:1020}
\sigma_\tmax = \frac{ 4 \pi L^4}{3 \lambda^2}
\end{dmath}
%
\section{Scattering from a sphere vs frequency.}
\index{scattering}
%
Frequency dependence of spherical scattering is sketched in \cref{fig:sphericalScattering:sphericalScatteringFig5}.
\index{spherical scattering}
%
\begin{itemize}
\item Low frequency (or small particles): Rayleigh
%
\begin{equation}\label{eqn:chapter2Notes:1040}
\sigma = \lr{\pi r^2} 7.11 \lr{\kappa r}^4, \qquad \kappa = 2 \pi/\lambda.
\end{equation}
%
\item Mie scattering (resonance),
\index{Mie scattering}
%
\begin{equation}\label{eqn:chapter2Notes:1060}
\sigma_\tmax(A) = 4 \pi r^2
\end{equation}
\begin{equation}\label{eqn:chapter2Notes:1080}
\sigma_\tmax(B) = 0.26 \pi r^2.
\end{equation}
%
\item optical limit ( \(r \gg \lambda\) )
%
\begin{equation}\label{eqn:chapter2Notes:1100}
\sigma = \pi r^2.
\end{equation}
\end{itemize}
%
\imageFigure{../figures/ece1229-antenna/sphericalScatteringFig5}{Scattering from a sphere vs frequency (from Prof. Eleftheriades' class notes).}{fig:sphericalScattering:sphericalScatteringFig5}{0.2}
%
FIXME: Do I have a derivation of this in my optics notes?
%
\section{EIRP.}
\index{EIRP}
\index{Effective Isotropic Receiving Power}
%
Prof. Eleftheriades introduces the term \textAndIndex{EIRP}, the Effective Isotropic Receiving Power, the product of power and gain \( P_t G_t \), measured in \si{W}.
%
\section{Free space impedance.}
\index{free space impedance}
%
In class we've seen
%
\begin{dmath}\label{eqn:chapter2Notes:111}
\eta = \sqrt{\frac{\mu_0}{\epsilon_0}}.
\end{dmath}
%
expressed as \( 120 \pi \approx 377 \).  It seemed curious to me that this was an exact value.  With
%
\begin{itemize}
\item \( \epsilon_0 = 8.85 \times 10^{-12} \) \si{C^2/N m^2} (number from \citep{griffiths1999introduction})
\item \( \mu_0 = 4 \pi \times 10^{-7} \si{N/A^2} \) (exact),
\end{itemize}
%
the numeric value of \( \eta/\pi \) is \( 119.945 \) (\nbref{eta.jl}), which is close to \( 120 \).  It's pointed out in \citep{wiki:freespaceImpedance} that this is just the consequence of using \( c = 3 \times 10^8 \si{m/s} \).

This can be seen by writing \( \eta \) in an alternate form
%
\begin{dmath}\label{eqn:chapter2Notes:112}
\eta
= \inv{c \epsilon_0} = \mu_0 c
= ( 4 \pi \times 10^{-7} \si{N/A^2} ) ( 3 \times 10^8 \si{m/s} )
= 120 \pi \si{N m/A^2 s}
= 120 \pi \Omega.
\end{dmath}
%
\section{Notation.}
%
\begin{itemize}
\item Time average.
Both Prof. Eleftheriades
and the text \citep{balanis2005antenna} use square brackets \( \timeaverage{\cdots} \) for time averages, not \( \bracketaverage{\cdots}\).  Was that an engineering convention?
\index{notation!time average}
\item Bold vectors are usually phasors, with (bold) calligraphic script used for the time domain fields.  Example: \( \BE(x,y,z,t) = \ecap E(x,y) e^{j \lr{\omega t - k z}}, \bcE(x, y, z, t) = \Real \BE \).
\end{itemize}
\index{notation!bold vectors}
\index{notation!caligraphic vectors}
%
%\section{Mathematica notebooks}
%
%\input{../ece1229/mathematica.tex}
%
%\EndArticle
