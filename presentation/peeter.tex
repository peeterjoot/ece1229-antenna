
\section{Discussion}
%(1 page)
\begin{frame}
\frametitle{Advantages}

\begin{itemize}
\item Flat.
\item Low losses.
\item No need to feed the patch elements directly.
\item Phase shifting electronics for each patch element is not required, lowering cost.
\item Reconfiguration and scanning is possible if phase shifting electronics are included.
\item Large aperture reflectarrays can be folded more easily than parabolic equivalents (for space applications).
\item Like parabolic reflectors, multiple beam capability is possible with additional feed sources.
\item Variable sized patch elements can lead to low cross-polarization.
\end{itemize}
\end{frame}

\begin{frame}
\frametitle{Disadvantages}

\begin{itemize}
\item Narrow bandwidth, due to narrow bandwidth of patch elements (3-5\%)
\item Differential spatial phase delays limit bandwidth, due to sensitivity of phase to the resonance frequencies.
\citep{pozar1997design}
%FIXME: Huang discusses this latter point as a major source of limited bandwidth,
%but doesn't make it clear why there is a connection between the phase differences at points in the array
%and how that relates to the overall bandwidth limitation.
\item Bandwidth limited by physical extent of the array, limiting space applications.
\citep{encinar2001design}
\end{itemize}
\end{frame}

%\begin{frame}[allowframebreaks]
\begin{frame}
\frametitle{Bandwidth Improvement}
\begin{itemize}
\item Stacked layers can reduce phase range limitations (to several times \ang{360}).
\citep{pozar1997design}
\item Delay lines aperture-coupled to printed patches.  Used to achieve 20 \% bandwidth in 80 \si{cm} reflectarray.
\citep{encinar2010recent}
\item Multifacet (\( 2 \si{m} \times 0.5 \si{m} \)) designs can reduce spatial phase delays.
%Both very large fixed piecewise flat configurations, as well as some very clever unfolding to parabolic designs .
%\makefigure{0.15}{huangFig58multifacet.png}{Multifacet reflectarray}
%huang7_58BpiecewiseFlatParabolic.png
%huang7_58ApiecewiseFlatParabolic.png
\makefigure{0.15}{huang7_58BpiecewiseFlatParabolic.png}{Multifacet \( 2 \si{m} \times 0.5 \si{m} \) reflectarray}
\end{itemize}
\end{frame}


\begin{frame}
\frametitle{Applications: Solar cell hybrid reflectarray.}

\begin{itemize}
\item Can be combined with solar cells to provide both power and antenna function. %\cref{fig:huang:huangFig2_10_solarCellWithRA}.
Small crossed dipole patches do not significantly inhibit solar cell performance. \citep{huang2008reflectarray}
\makefigure{0.15}{huangFig2_10_solarCellWithRA.png}{Hybrid solar cell reflectarray \citep{huang2008reflectarray}}
\end{itemize}
\end{frame}

\begin{frame}
\frametitle{Applications: Inflatable reflectarray.}

\begin{itemize}
\item
Example of 1 \si{m} 8.3 \si{GHz} (X-band), and 3 \si{m} 32 \si{GHz} (Ka-band) reflectarrays.  The 3\si{m} array has 200000 elements!
\end{itemize}

\twofigures{0.2}{huang7_1_inflatable1meter.png}{0.27}{huang7_3_inflatable3m.png}{1 \si{m} X-band and 3 \si{m} Ka-band inflatable reflectarrays (JPL and ILC Dover).}
\end{frame}

\begin{frame}
\frametitle{Applications: Contoured beam applications.}

\begin{itemize}
\item It can be desirable to have a beam that targets multiple geographies.
\item Don't want to waste power on unpopulated areas.
\item Example: 12 \si{GHz}, 1 \si{m} three-layer reflectarray for geostationary targeting of three regions. From: \citep{encinar2004three}
\end{itemize}

\makefigure{0.3}{threeLayerReflectarrayContouredBeamEncinarPaperFig3.png}{Geostationary targeting of three regions. From: \citep{encinar2004three}}
\end{frame}

%TODO: 4-5 page.  2 left.

\begin{frame}
\frametitle{Transmitarray (array lenses)}
\begin{itemize}
\item Feed from behind the array instead of centered or offset reflected feed.  From: \citep{lau2012reconfigurable}
\makefigure{0.15}{lau_jonathan_figure1_2_transmitarray.png}{transmitarray feed configuration.  From: \citep{lau2012reconfigurable}}
\item Can avoid blocking the beam with the feed.
\item Design care is required to avoid undesirable reflection off the back surface.
\item With less reliance on ground plane reflection and resonance than reflectarrays, higher bandwidth designs may be possible.
\end{itemize}
\end{frame}

\begin{frame}[allowframebreaks]
\frametitle{Future directions}
\begin{itemize}
\item Gathered elements with common phase control has been used/proposed to reduce the number of active control devices for scannable and reconfigurable devices.  \citep{carrasco2012recent}
\item Multi-beam configurations with a single feed. \citep{carrasco2012recent}
\item Sub-reflectarrays in parabolic configuration for beam scanning.  \citep{arrebola2010phase} describes \( \pm \ang{12} \) scanning for 12 \si{GHz} 1 meter parabolic dish with 22 \si{cm} reflectarrays.
\item LCD substrates.  Bias voltages can alter the dielectric constant of the substrate.  Sub-millimeter (\si{Thz}) scanning reflectarrays have been demonstrated using this method.
\item ``Folded'' reflectarray.  The feed is put inline or behind the patch elements, with a polarizing grid above the array elements.
\twofigures{0.2}{huang_foldedSchematicFig7_45a.png}{0.27}{huang_foldedRenderingFig7_45b.png}{Folded reflectarray.}
\end{itemize}
\end{frame}

%TODO: 2-1 page

\begin{frame}
\frametitle{Summary}

\begin{itemize}
\item Desirable physical characteristics (flat, foldable, ...)
\item Can be manufactured with printed circuit techniques (relatively cheap, once designed).
\item Large variety of applications (space, ground, ...)
\item Can be combined with other techniques (arrays of reflectarrays, electronic phase control, ...)
\item Satisfactory solutions to original bandwidth limitations have been devised.
\item Active research area.
\end{itemize}

\end{frame}
