%
% Copyright � 2015 Peeter Joot.  All Rights Reserved.
% Licenced as described in the file LICENSE under the root directory of this GIT repository.
%
%\input{../blogpost.tex}
%\renewcommand{\basename}{energyMomentumWithMagneticSources}
%\renewcommand{\dirname}{notes/ece1229/}
%%\newcommand{\dateintitle}{}
%%\newcommand{\keywords}{}
%
%\input{../peeter_prologue_print2.tex}
%
%\usepackage{macros_bm}
%
%\beginArtNoToc
%
%\generatetitle{energy momentum conservation with magnetic sources}
\section{Energy momentum conservation}
\index{energy momentum conservation}
%%\label{chap:energyMomentumWithMagneticSources}
%%\section{Motivation}
%\section{Guts}

\paragraph{Maxwell's equations with magnetic sources}
\index{magnetic source}
%
In this section, the form of Maxwell's equations to be used are expressed in terms of \( \bcE \) and \( \bcH \), assume \textAndIndex{linear media}, and do not assume a \textAndIndex{phasor} representation

\begin{subequations}
\label{eqn:energyMomentumWithMagneticSources:100}
\begin{dmath}\label{eqn:energyMomentumWithMagneticSources:120}
\spacegrad \cross \bcE = - \bcM - \mu_0 \PD{t}{\bcH}
\end{dmath}
\begin{dmath}\label{eqn:energyMomentumWithMagneticSources:140}
\spacegrad \cross \bcH = \bcJ + \epsilon_0 \PD{t}{\bcE}
\end{dmath}
\begin{dmath}\label{eqn:energyMomentumWithMagneticSources:160}
\spacegrad \cdot \bcE = \rho/\epsilon_0
\end{dmath}
\begin{dmath}\label{eqn:energyMomentumWithMagneticSources:180}
\spacegrad \cdot \bcH = \rho_\txtm/\mu_0.
\end{dmath}
\end{subequations}

\paragraph{Energy momentum conservation}
\index{energy momentum conservation}
%
With magnetic sources the Poynting and \textAndIndex{energy conservation} relationship has to be adjusted slightly.  Let's derive that result, starting with the \textAndIndex{divergence} of the \textAndIndex{Poynting vector}
%
\begin{dmath}\label{eqn:energyMomentumWithMagneticSources:20}
\spacegrad \cdot \lr{ \bcE \cross \bcH }
=
\bcH \cdot \lr{ \spacegrad \cross \bcE }
-\bcE \cdot \lr{ \spacegrad \cross \bcH }
=
-\bcH \cdot \lr{ \mu_0 \partial_t \bcH + \bcM }
-\bcE \cdot \lr{ \bcJ + \epsilon_0 \partial_t \bcE }
=
- \mu_0 \bcH \cdot \partial_t \bcH - \bcH \cdot \bcM
- \epsilon_0 \bcE \cdot \partial_t \bcE - \bcE \cdot \bcJ,
\end{dmath}
%
or

\boxedEquation{eqn:energyMomentumWithMagneticSources:40}{
%\begin{dmath}\label{eqn:energyMomentumWithMagneticSources:40}
%\boxed{
\inv{2} \PD{t}{} \lr{ \epsilon_0 \bcE^2 + \mu_0 \bcH^2 }
+
\spacegrad \cdot \lr{ \bcE \cross \bcH }
=
- \bcH \cdot \bcM
- \bcE \cdot \bcJ.
%\end{dmath}
}

\paragraph{Momentum conservation}
\index{momentum conservation}
%
The usual relationship is only modified by one additional term.  Recall from electrodynamics \citep{phy450:relativisticElectrodynamicsL23} that \cref{eqn:energyMomentumWithMagneticSources:40} (when the magnetic current density \( \bcM \) is omitted) is just one of four components of the energy momentum conservation equation
%
\begin{equation}\label{eqn:energyMomentumWithMagneticSources:80}
\partial_\mu T^{\mu \nu} = - \inv{c} F^{\nu \lambda} j_\lambda.
\end{equation}
%
Note that \cref{eqn:energyMomentumWithMagneticSources:80} was likely not in SI units.  The next task is to generalize this classical relationship to incorporate the magnetic sources used in antenna theory.  With an eye towards the relativistic nature of the \textAndIndex{energy momentum tensor}, it is natural to assume that the remainder of the energy momentum tensor conservation relation can be found by taking the time derivatives of the Poynting vector.
%
\begin{dmath}\label{eqn:energyMomentumWithMagneticSources:200}
\PD{t}{} \lr{ \bcE \cross \bcH }
=
\PD{t}{\bcE} \cross \bcH
+ \bcE \cross \PD{t}{\bcH }
=
\inv{\epsilon_0}
\lr{ \spacegrad \cross \bcH - \bcJ } \cross \bcH
+
\inv{\mu_0}
\bcE \cross
\lr{
-
\spacegrad \cross \bcE - \bcM },
\end{dmath}
%
or
%
\begin{dmath}\label{eqn:energyMomentumWithMagneticSources:220}
\begin{aligned}
\inv{c^2} \PD{t}{} \lr{ \bcE \cross \bcH }
&
+ \mu_0 \bcJ \cross \bcH
+ \epsilon_0 \bcE \cross \bcM \\
&=
-\mu_0 \bcH \cross \lr{ \spacegrad \cross \bcH }  \\
&\quad - \epsilon_0 \bcE \cross \lr{ \spacegrad \cross \bcE }.
\end{aligned}
\end{dmath}
%
The \( \mu_0 \bcJ \cross \bcH = \bcJ \cross \BB \) is a portion of the Lorentz force equation \index{Lorentz force equation} in its density form.  To put
\cref{eqn:energyMomentumWithMagneticSources:220} into the desired form, the remainder of the
Lorentz force force equation
\( \rho \bcE = \epsilon_0 \bcE \spacegrad \cdot \bcE \) must be added to both sides.  To extend the magnetic current term to its full dual (magnetic) Lorentz force \index{Lorentz force!dual} structure, the quantity to add to both sides is \( \rho_\txtm \bcH = \mu_0 \bcH \spacegrad \cdot \bcH \).  Performing these manipulations gives
%
\begin{dmath}\label{eqn:energyMomentumWithMagneticSources:240}
\inv{c^2} \PD{t}{} \lr{ \bcE \cross \bcH }
+
\rho \BE + \mu_0 \bcJ \cross \bcH
+ \rho_\txtm \bcH
+ \epsilon_0 \bcE \cross \bcM
=
\mu_0
\lr{
\bcH \spacegrad \cdot \bcH
-\bcH \cross \lr{ \spacegrad \cross \bcH }
}
+ \epsilon_0
\lr{
 \bcE \spacegrad \cdot \bcE
-
\bcE \cross \lr{ \spacegrad \cross \bcE }
}.
\end{dmath}
%
It seems slightly surprising the sign of the magnetic equivalent of the Lorentz force terms have an alternation of sign.  This is, however, consistent with the \textAndIndex{duality} transformations outlined in (\citep{balanis2005antenna} table 3.2)

\begin{subequations}
\label{eqn:energyMomentumWithMagneticSources:260}
\begin{dmath}\label{eqn:energyMomentumWithMagneticSources:280}
\rho \rightarrow \rho_\txtm
\end{dmath}
\begin{dmath}\label{eqn:energyMomentumWithMagneticSources:300}
\bcJ \rightarrow \bcM
\end{dmath}
\begin{dmath}\label{eqn:energyMomentumWithMagneticSources:320}
\mu_0 \rightarrow \epsilon_0
\end{dmath}
\begin{dmath}\label{eqn:energyMomentumWithMagneticSources:340}
\bcE \rightarrow \bcH
\end{dmath}
\begin{dmath}\label{eqn:energyMomentumWithMagneticSources:360}
\bcH \rightarrow -\bcE,
\end{dmath}
\end{subequations}

for
%
\begin{equation}\label{eqn:energyMomentumWithMagneticSources:380}
\rho \BE + \mu_0 \bcJ \cross \bcH
\rightarrow
\rho_\txtm \BH + \epsilon_0 \bcM \cross \lr{ -\bcE}
=
\rho_\txtm \BH + \epsilon_0 \bcE \cross \bcM.
\end{equation}
%
Comfortable that the LHS has the desired structure, the RHS can expressed as a divergence.  Just expanding one of the differences of vector products on the RHS does not obviously show that this is possible, for example
%
\begin{dmath}\label{eqn:energyMomentumWithMagneticSources:400}
\Be_a \cdot
\lr{
\bcE \spacegrad \cdot \bcE
-
\bcE \cross \lr{ \spacegrad \cross \bcE }
}
=
E_a \partial_b E_b
-
\epsilon_{a b c} E_b \epsilon_{c r s} \partial_r E_s
=
E_a \partial_b E_b
-
\delta_{a b}^{[r s]} E_b \partial_r E_s
=
E_a \partial_b E_b
-
E_b \lr{
\partial_a E_b
-\partial_b E_a
}
=
  E_a \partial_b E_b
- E_b \partial_a E_b
+ E_b \partial_b E_a.
\end{dmath}
%
This happens to equal
%
\begin{dmath}\label{eqn:energyMomentumWithMagneticSources:420}
\spacegrad \cdot \lr{ \lr{E_a E_b - \inv{2} \delta_{a b} \bcE^2 } \Be_b }
=
\partial_b
\lr{E_a E_b - \inv{2} \delta_{a b} \bcE^2 }
=
E_b \partial_b E_a
+ E_a \partial_b E_b
-
\inv{2} \delta_{a b} 2 E_c \partial_b E_c
=
  E_b \partial_b E_a
+ E_a \partial_b E_b
- E_b \partial_a E_b.
\end{dmath}
%
This allows a final formulation of the remaining energy momentum conservation equation in its divergence form.  Let
%
\begin{dmath}\label{eqn:energyMomentumWithMagneticSources:440}
T^{a b} =
\epsilon_0 \lr{ E_a E_b - \inv{2} \delta_{a b} \bcE^2 }
+ \mu_0 \lr{ H_a H_b - \inv{2} \delta_{a b} \bcH^2 },
\end{dmath}
%
so that the remaining energy momentum conservation equation, extended to both electric \index{electric source} and magnetic sources \index{magnetic source}, is
%
\boxedEquation{eqn:energyMomentumWithMagneticSources:460}{
%\begin{dmath}\label{eqn:energyMomentumWithMagneticSources:460}
%\boxed{
\inv{c^2} \PD{t}{} \lr{ \bcE \cross \bcH }
+ \lr{ \rho \BE + \mu_0 \bcJ \cross \bcH }
+ \lr{ \rho_\txtm \bcH + \epsilon_0 \bcE \cross \bcM }
=
\Be_a \spacegrad \cdot \lr{ T^{a b} \Be_b }.
}
%\end{dmath}
%
On the LHS we have the rate of change of momentum density, the electric Lorentz force density terms, the dual (magnetic) Lorentz force density terms, and on the RHS the the \textAndIndex{momentum flux} terms.

\paragraph{In the frequency domain}
\index{frequency domain}
%
In the frequency domain with \( \bcE = \Real \BE e^{j \omega t}, \bcH = \Real \BH e^{j \omega t} \).  Using the electric field dot product as an example, note that we can write
%
\begin{dmath}\label{eqn:energyMomentumWithMagneticSources:480}
\bcE = \inv{2} \lr{ \BE e^{j \omega t} + \BE^\conj e^{-j \omega t} },
\end{dmath}
%
so
%
\begin{dmath}\label{eqn:energyMomentumWithMagneticSources:500}
\bcE^2
=
\inv{2} \lr{ \BE e^{j \omega t} + \BE^\conj e^{-j \omega t} }
\cdot
\inv{2} \lr{ \BE e^{j \omega t} + \BE^\conj e^{-j \omega t} }
=
\inv{4} \lr{
\BE^2 e^{2 j \omega t}
+ \BE \cdot \BE^\conj + \BE^\conj \cdot \BE
+\lr{\BE^\conj}^2 e^{-2 j \omega t}
}
=
\inv{2} \Real
\lr{
\BE \cdot \BE^\conj
+
\BE^2 e^{2 j \omega t}
}.
\end{dmath}
%
Similarly, for the cross product
%
\begin{dmath}\label{eqn:energyMomentumWithMagneticSources:540}
\bcE \cross \bcH
=
\inv{4}
\lr{
\BE \cross \BH e^{2 j \omega t}
+ \BE \cross \BH^\conj + \BE^\conj \cross \BH
+ \lr{ \BE^\conj \cross \BH^\conj } e^{-2 j \omega t}
}
=
\inv{2}
\Real
\lr{
\BE \cross \BH^\conj
+
\BE \cross \BH e^{2 j \omega t}
}.
\end{dmath}
%
Given phasor representations of the sources \( \bcM = \BM e^{j \omega t}, \bcJ = \BJ e^{j \omega t} \), \cref{eqn:energyMomentumWithMagneticSources:40} can be recast into (a messy) phasor form
%
\begin{dmath}\label{eqn:energyMomentumWithMagneticSources:560}
\begin{aligned}
\inv{2} &\Real \inv{2} \PD{t}{} \lr{
\epsilon_0 \BE \cdot \BE^\conj
+ \mu_0 \BH \cdot \BH^\conj
+ \epsilon_0 \BE^2 e^{ 2 j \omega t}
+ \mu_0 \BH^2 e^{ 2 j \omega t}
} \\
&+
\inv{2} \Real \spacegrad \cdot \lr{
\BE \cross \BH^\conj
+\BE \cross \BH e^{ 2 j \omega t}
} \\
&=
\inv{2} \Real
\lr{
- \BH \cdot \BM^\conj
- \BE \cdot \BJ^\conj
- \BH \cdot \BM e^{2 j \omega t}
- \BE \cdot \BJ e^{2 j \omega t}
}.
\end{aligned}
\end{dmath}
%
All the time dependence has been moved into the exponential factors, so the \( \epsilon_0 \BE \cdot \BE^\conj + \mu_0 \BH \cdot \BH^\conj  \) terms are killed by the time derivative operator.  Averaging over one period kills the rest of the oscillatory terms, leaving just

\boxedEquation{eqn:energyMomentumWithMagneticSources:580}{
%\begin{dmath}\label{eqn:energyMomentumWithMagneticSources:580}
0 =
\spacegrad \cdot \lr{
\BE \cross \BH^\conj
}
+
\BH \cdot \BM^\conj
+
\BE \cdot \BJ^\conj.
%\end{dmath}
}

\paragraph{Comparison to the reciprocity theorem result}
\index{reciprocity theorem}
%
The reciprocity theorem had a striking similarity to the \textAndIndex{Poynting theorem} above, which isn't suprising since both were derived by calculating the divergence of a Poynting like quantity.
%Repeating the divergence calculation of the reciprocity derivation in the time domain shows that the \textAndIndex{reciprocity theorem} as stated also has a dependency on single frequency sources and fields
Here's a repetition of the reciprocity divergence calculation without the single frequency (phasor) assumption
%
\begin{dmath}\label{eqn:energyMomentumWithMagneticSources:600}
\begin{aligned}
\spacegrad \cdot &\lr{
\bcE^{(a)} \cross \bcH^{(b)}
-\bcE^{(b)} \cross \bcH^{(a)}
} \\
&=
\bcH^{(b)} \cdot \lr{ \spacegrad \cross \bcE^{(a)} } -\bcE^{(a)} \cdot \lr{ \spacegrad \cross \bcH^{(b)} } \\
&\quad
-\bcH^{(a)} \cdot \lr{ \spacegrad \cross \bcE^{(b)} } +\bcE^{(b)} \cdot \lr{ \spacegrad \cross \bcH^{(a)} } \\
&=
-\bcH^{(b)} \cdot \lr{ \mu_0 \partial_t \bcH^{(a)} + \bcM^{(a)} }
-\bcE^{(a)} \cdot \lr{ \bcJ^{(b)} + \epsilon_0 \partial_t \bcE^{(b)} } \\
&\quad
+\bcH^{(a)} \cdot \lr{ \mu_0 \partial_t \bcH^{(b)} + \bcM^{(b)} }
+\bcE^{(b)} \cdot \lr{ \bcJ^{(a)} + \epsilon_0 \partial_t \bcE^{(a)} } \\
&=
\epsilon_0
\lr{
 \bcE^{(b)} \cdot \partial_t \bcE^{(a)}
-\bcE^{(a)} \cdot \partial_t \bcE^{(b)}
}
+
\mu_0
\lr{
 \bcH^{(a)} \cdot \partial_t \bcH^{(b)}
-\bcH^{(b)} \cdot \partial_t \bcH^{(a)}
} \\
&+\bcH^{(a)} \cdot \bcM^{(b)}
-\bcH^{(b)} \cdot \bcM^{(a)}
+\bcE^{(b)} \cdot \bcJ^{(a)}
-\bcE^{(a)} \cdot \bcJ^{(b)}
\end{aligned}
\end{dmath}
%
What do these time derivative terms look like in the \textAndIndex{frequency domain}?
%
\begin{dmath}\label{eqn:energyMomentumWithMagneticSources:620}
\bcE^{(b)} \cdot \partial_t \bcE^{(a)}
=
\inv{4}
\lr{
\BE^{(b)} e^{j \omega t}
+
{\BE^{(b)}}^\conj e^{-j \omega t}
}
\cdot
\partial_t
\lr{
\BE^{(a)} e^{j \omega t}
+
{\BE^{(a)}}^\conj e^{-j \omega t}
}
=
\frac{j \omega}{4}
\lr{
\BE^{(b)} e^{j \omega t}
+
{\BE^{(b)}}^\conj e^{-j \omega t}
}
\cdot
\lr{
\BE^{(a)} e^{j \omega t}
-
{\BE^{(a)}}^\conj e^{-j \omega t}
}
=
\frac{\omega}{4}
\lr{
j \BE^{(a)} \cdot { \BE^{(b)} }^\conj
-j \BE^{(b)} \cdot { \BE^{(a)} }^\conj
+j \BE^{(a)} \cdot \BE^{(b)} e^{ 2 j \omega t }
-j { \BE^{(a)}}^\conj \cdot { \BE^{(b)} }^\conj e^{ -2 j \omega t }
}
=
\inv{2} \Real
\lr{
  j \omega \BE^{(a)} \cdot { \BE^{(b)} }^\conj
+ j \omega \BE^{(a)} \cdot \BE^{(b)} e^{ 2 j \omega t }
}
\end{dmath}
%
Taking the difference,
%
\begin{equation}\label{eqn:energyMomentumWithMagneticSources:640}
\begin{aligned}
&\bcE^{(b)} \cdot \partial_t \bcE^{(a)}
-\bcE^{(a)} \cdot \partial_t \bcE^{(b)} \\
&=
\inv{2} \Real
\lr{
  j \omega \BE^{(a)} \cdot { \BE^{(b)} }^\conj
- j \omega \BE^{(b)} \cdot { \BE^{(a)} }^\conj
+ j \omega \BE^{(a)} \cdot \BE^{(b)} e^{ 2 j \omega t }
- j \omega \BE^{(b)} \cdot \BE^{(a)} e^{ 2 j \omega t }
} \\
&=
- \omega \Imag
\lr{
  \BE^{(a)} \cdot { \BE^{(b)} }^\conj
+ \BE^{(a)} \cdot \BE^{(b)} e^{ 2 j \omega t }
},
\end{aligned}
\end{equation}
%
so we have
%
\begin{dmath}\label{eqn:energyMomentumWithMagneticSources:660}
0
=
\timeaverage{
\spacegrad \cdot \Real \lr{
\BE^{(a)} \cross {\BH^{(b)}}^\conj
-\BE^{(b)} \cross {\BH^{(a)}}^\conj
}
+
\omega \Imag
\lr{
\epsilon_0
\BE^{(a)} \cdot { \BE^{(b)} }^\conj
+
\mu_0
\BH^{(a)} \cdot { \BH^{(b)} }^\conj
}
+ \Real
\lr{
-\BH^{(a)} \cdot { \BM^{(b)} }^\conj
+\BH^{(b)} \cdot { \BM^{(a)} }^\conj
-\BE^{(b)} \cdot { \BJ^{(a)} }^\conj
+\BE^{(a)} \cdot { \BJ^{(b)} }^\conj
}
}_{\textrm{av}}.
\end{dmath}
%
Observe that the perfect cancellation of the time derivative terms only occurs when the cross product differences were those of the phasors.  When those cross differences are those of the actual fields, like those in the Poynting theorem, there is a frequency dependent term is that expansion.

\paragraph{Followup Questions}

FIXME: TODO.

\begin{enumerate}
\item
What do the energy momentum conservation equations look like in geometric algebra form with magnetic sources?
\item
What do the energy momentum conservation equations look like in tensor form with magnetic sources?
\end{enumerate}

%\EndArticle
