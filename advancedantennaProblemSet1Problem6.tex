%
% Copyright � 2015 Peeter Joot.  All Rights Reserved.
% Licenced as described in the file LICENSE under the root directory of this GIT repository.
%
\makeoproblem{Radar cross section.}{advancedantenna:problemSet1:6}{2015 problem set 1, p6}{
\index{radar cross section}
A rectangular X-band horn, with aperture dimensions \( 5.5 \si{cm} \times 7.4 \si{cm} \) and a gain of 16.3 \si{dB}
at 10 \si{GHz}, transmits and receives power scattered by the objects specified below.
\index{X-band}
\index{scattered power}
\index{radar cross section}
\index{RCS}

In each case, determine the maximum scattered power delivered to the load when the
distance between the horn and scattering object is \( n \lambda \), where \( n \) is

\begin{enumerate}
\item
200
\item
500.
\end{enumerate}

The scattering objects to consider are a perfectly conducting
%
\makesubproblem{}{advancedantenna:problemSet1:6a}
%
sphere of radius \( a = 5 \lambda \),
%
\makesubproblem{}{advancedantenna:problemSet1:6b}
%
plate of dimensions \( 10 \lambda \times 10 \lambda \).

} % makeoproblem
%
\makeanswer{advancedantenna:problemSet1:6}{
%
This is an application of the Radar Cross Section equation
%, and it doesn't appear that the horn aperture dimensions are relevant
%
\begin{dmath}\label{eqn:advancedantennaProblemSet1Problem6:120}
\frac{P_\txtr}{P_\txtt}
= \sigma \frac{G^2}{4 \pi} \lr{\frac{\lambda}{4 \pi n^2 \lambda^2}}^2
= \sigma \frac{G^2}{4 \pi} \lr{\frac{1}{4 \pi n^2 \lambda}}^2.
\end{dmath}
%
The same gain is used for transmission and reception, since both are for the same horn.  That gain (not in \si{dB}) is
%
\begin{dmath}\label{eqn:advancedantennaProblemSet1Problem6:40}
G = 10^{16.3/10} = 43.
\end{dmath}
%
The wavelength is
%
\begin{dmath}\label{eqn:advancedantennaProblemSet1Problem6:60}
\lambda = \frac{3 \times 10^8 \,\si{m/s}}{10^{10} \,\si{m}} = 0.03 \,\si{m}.
\end{dmath}
%
\makeSubAnswer{}{advancedantenna:problemSet1:6a}
%
For the sphere the scattering area is
%
\begin{dmath}\label{eqn:advancedantennaProblemSet1Problem6:80}
\sigma = \pi r^2 = \pi ( 5 \lambda )^2 = 25 \pi \lambda^2
\end{dmath}
%
so the ratio of delivered power to the transmitted power is
%
\begin{dmath}\label{eqn:advancedantennaProblemSet1Problem6:100}
\frac{P_\txtr}{P_\txtt}
= 25 \pi \lambda^2 \frac{G^2}{4 \pi} \lr{\frac{1}{4 \pi n^2 \lambda}}^2
= \frac{25 (43)^2}{64 \pi^2 n^4}
= \frac{73}{n^4}.
\end{dmath}
%
For the \( n = 200, 500 \) cases respectively, the delivered power ratio is

\begin{enumerate}
\item \( n = 200 \)
%
\begin{dmath}\label{eqn:advancedantennaProblemSet1Problem6:140}
\frac{P_\txtr}{P_\txtt}
= \frac{73}{200^4}
= 4.6 \times 10^{-8}
\end{dmath}
%
\item \( R = 500 \lambda \)
%
\begin{dmath}\label{eqn:advancedantennaProblemSet1Problem6:160}
\frac{P_\txtr}{P_\txtt}
= \frac{73}{500^4}
= 1.2 \times 10^{-9}
\end{dmath}
%
\end{enumerate}
%
\makeSubAnswer{}{advancedantenna:problemSet1:6b}
%
For the plate the scattering area is
%
\begin{dmath}\label{eqn:advancedantennaProblemSet1Problem6:180}
\sigma
= 4 \pi \frac{\lr{ L W }^2}{\lambda^2 }
= 4 \pi \frac{\lr{ 100 \lambda^2 }^2}{\lambda^2 }
= 4 \pi \lambda^2 \times 10^4.
\end{dmath}
%
so the ratio of delivered power to the transmitted power is
%
\begin{dmath}\label{eqn:advancedantennaProblemSet1Problem6:200}
\frac{P_\txtr}{P_\txtt}
= 4 \pi \lambda ^2 \times 10^4 \frac{G^2}{4 \pi} \lr{\frac{1}{4 \pi n^2 \lambda}}^2
= \frac{(43)^2 \times 10^4}{16 \pi^2 n^4}
= \frac{1.2 \times 10^5}{n^4}.
\end{dmath}
%
For the \( n = 200, 500 \) cases respectively, the delivered power ratio is

\begin{enumerate}
\item \( n = 200 \)
%
\begin{dmath}\label{eqn:advancedantennaProblemSet1Problem6:220}
\frac{P_\txtr}{P_\txtt}
= \frac{1.2 \times 10^5}{200^4}
= 7.3 \times 10^{-5}.
\end{dmath}
%
\item \( R = 500 \lambda \)
%
\begin{dmath}\label{eqn:advancedantennaProblemSet1Problem6:240}
\frac{P_\txtr}{P_\txtt}
= \frac{1.2 \times 10^5}{500^4}
= 1.9 \times 10^{-6}.
\end{dmath}
%
\end{enumerate}
}
