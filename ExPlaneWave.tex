%
% Copyright � 2015 Peeter Joot.  All Rights Reserved.
% Licenced as described in the file LICENSE under the root directory of this GIT repository.
%
%\input{../blogpost.tex}
%\renewcommand{\basename}{ExPlaneWave}
%\renewcommand{\dirname}{notes/ece1229/}
%%\newcommand{\dateintitle}{}
%%\newcommand{\keywords}{}
%
%\input{../peeter_prologue_print2.tex}
%\usepackage{peeters_layout_exercise}
%
%\beginArtNoToc
%
%\generatetitle{Plane wave solution directly from Maxwell's equations}
%\section{Plane wave solution directly from Maxwell's equations}
%\label{chap:ExPlaneWave}
%
%Here's a problem that I thought was fun, an exercise for the reader to show that the plane wave solution to Maxwell's equations can be found with ease directly from Maxwell's equations.  This is in contrast to the what seems like the usual method of first showing that Maxwell's equations imply wave equations for the fields, and then solving those wave equations.
%
\makeoproblem{\( \xcap \) oriented plane wave electric field.}{problem:ExPlaneWave:1}{\citep{balanis1989advanced} ex. 4.1}{
A uniform plane wave having only an \( x \) component of the electric field is traveling in the \( + z \) direction in an unbounded lossless, source-0free region.  Using Maxwell's equations write expressions for the electric and corresponding magnetic field intensities.
} % problem
%
\makeanswer{problem:ExPlaneWave:1}{
The phasor form of Maxwell's equations for a source free region are
%
\begin{subequations}
\label{eqn:ExPlaneWave:20}
\begin{equation}\label{eqn:ExPlaneWave:40}
\spacegrad \cross \BE = -j \omega \BB
\end{equation}
\begin{equation}\label{eqn:ExPlaneWave:60}
\spacegrad \cross \BH = j \omega \BD
\end{equation}
\begin{equation}\label{eqn:ExPlaneWave:80}
\spacegrad \cdot \BD = 0
\end{equation}
\begin{equation}\label{eqn:ExPlaneWave:100}
\spacegrad \cdot \BB = 0.
\end{equation}
\end{subequations}
%
Since \( \BE = \xcap E(z) \), the magnetic field follows from \cref{eqn:ExPlaneWave:40}
%
\begin{dmath}\label{eqn:ExPlaneWave:120}
-j \omega \BB
= \spacegrad \cross \BE
=
\begin{vmatrix}
\xcap & \ycap & \zcap \\
\partial_x & \partial_y & \partial_z \\
E & 0 & 0
\end{vmatrix}
=
\ycap \partial_z E(z)
- \zcap \cancel{\partial_y E(z)},
\end{dmath}
%
or
%
\begin{dmath}\label{eqn:ExPlaneWave:140}
\BB =
-\inv{j \omega} \partial_z E.
\end{dmath}
%
This is constrained by \cref{eqn:ExPlaneWave:60}
%
\begin{dmath}\label{eqn:ExPlaneWave:160}
j \omega \epsilon \xcap E
=
\inv{\mu} \spacegrad \cross \BB
=
-\inv{\mu j \omega}
\begin{vmatrix}
\xcap & \ycap & \zcap \\
\partial_x & \partial_y & \partial_z \\
0 & \partial_z E & 0
\end{vmatrix}
=
-\inv{\mu j \omega}
\lr{
-\xcap \partial_{z z} E
+ \zcap \partial_x \partial_z E
}.
\end{dmath}
%
Since \( \partial_x \partial_z E = \partial_z \lr{ \partial_x E } = \partial_z \inv{\epsilon} \spacegrad \cdot \BD = \partial_z 0 \), this means
%
\begin{equation}\label{eqn:ExPlaneWave:180}
\partial_{zz} E = -\omega^2 \epsilon\mu E = -k^2 E.
\end{equation}
%
This is the usual starting place that we use to show that the plane wave has an exponential form
%
\begin{equation}\label{eqn:ExPlaneWave:200}
\BE(z) =
\xcap
\lr{
E_{+} e^{-j k z}
+
E_{-} e^{j k z}
}.
\end{equation}
%
The magnetic field from \cref{eqn:ExPlaneWave:140} is
%
\begin{dmath}\label{eqn:ExPlaneWave:220}
\BB
= \frac{j}{\omega} \lr{ -j k E_{+} e^{-j k z} + j k E_{-} e^{j k z} }
= \inv{c} \lr{ E_{+} e^{-j k z} - E_{-} e^{j k z} },
\end{dmath}
%
or
%
\begin{dmath}\label{eqn:ExPlaneWave:240}
\BH
= \inv{\mu c} \lr{ E_{+} e^{-j k z} - E_{-} e^{j k z} }
= \inv{\eta} \lr{ E_{+} e^{-j k z} - E_{-} e^{j k z} }.
\end{dmath}
%
A solution requires zero divergence for the magnetic field, but that can be seen to be the case by inspection.
} % answer
%
%\EndArticle
%\EndNoBibArticle
