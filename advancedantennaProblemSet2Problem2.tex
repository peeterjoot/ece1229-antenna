%
% Copyright � 2015 Peeter Joot.  All Rights Reserved.
% Licenced as described in the file LICENSE under the root directory of this GIT repository.
%
\makeoproblem{Testing antenna gain.}{advancedantenna:problemSet2:2}{2015 ps2, p2}{
\index{antenna gain}
%
One way to measure the absolute gain of an antenna under test (AUT) is to use a ``standard-gain'' antenna (usually a horn) which has a known gain \( G_{\textrm{sg}}\).
Consider a two antenna setup, where port \#1 is connected to a transmitting antenna \( G_\txtx \).
First, the second antenna connected to port \#2 is the standard-gain one. Then at port \#2 we connect the unknown antenna under test \( G_{\textrm{AUT}} \).
%
\index{antenna under test}
\index{AUT}
\index{standard gain}
Show that,
%
\begin{dmath}\label{eqn:advancedantennaProblemSet2Problem2:20}
\frac{P_2^{\textrm{AUT}}}{P_2^{\textrm{sg}}}
=
\frac
{G_{\textrm{AUT}}}
{G_{\textrm{sg}}}
,
\end{dmath}
%
where the left-hand side of the above equation represents the ratio of the powers received by the antenna under test and the standard-gain antenna.

} % makeoproblem
%
\makeanswer{advancedantenna:problemSet2:2}{
%
The Friis equation can be used for this measurement task.
\index{Friis equation}
For the respective set of antenna configurations, and for fixed transmission power, there are two such equations
%
\begin{subequations}
\label{eqn:advancedantennaProblemSet2Problem2:40}
\begin{dmath}\label{eqn:advancedantennaProblemSet2Problem2:60}
\frac{P_2^{\textrm{AUT}}}{P_t} = \lr{\frac{\lambda}{4 \pi R}}^2 G_{\textrm{AUT}} G_\txtt
\end{dmath}
\begin{dmath}\label{eqn:advancedantennaProblemSet2Problem2:80}
\frac{P_2^{\textrm{sg}}}{P_t} = \lr{\frac{\lambda}{4 \pi R}}^2 G_{\textrm{sg}} G_\txtt,
\end{dmath}
\end{subequations}
%
where the transmission power is \( P_t \) and transmission antenna gain is \( G_\txtt \).  That transmit antenna power and gain need not be known, since dividing these equations cancels the common factors, including those, leaving
%
\begin{dmath}\label{eqn:advancedantennaProblemSet2Problem2:100}
\frac{P_2^{\textrm{AUT}}}{P_2^{\textrm{sg}}} = \frac{G_{\textrm{AUT}}}{G_{\textrm{sg}}},
\end{dmath}
%
as desired.

This procedure assumes that the standard gain antenna and the antenna under test have identical polarization, and that neither is orthogonally polarized with respect to the antenna at port \#1.
\index{polarization}
}
