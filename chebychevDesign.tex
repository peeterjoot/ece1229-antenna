%
% Copyright � 2015 Peeter Joot.  All Rights Reserved.
% Licenced as described in the file LICENSE under the root directory of this GIT repository.
%
%\input{../blogpost.tex}
%\renewcommand{\basename}{chebychevDesign}
%\renewcommand{\dirname}{notes/ece1229/}
%
%\input{../peeter_prologue_print2.tex}
%
%\usepackage{peeters_layout_exercise}
%\usepackage{ece1229}
%\usepackage{siunitx}
%\usepackage{esint} % \oiint
%
%\beginArtNoToc
%
%\generatetitle{Cheybshev antenna design}
%\chapter{Cheybshev antenna design}
\index{Cheybshev array}
%\label{chap:chebychevDesign}
%
\makeexample{Cheybshev antenna design.}{example:chebychevDesign:10}{
In our text \citep{balanis2005antenna} is a design procedure that applies Cheybshev polynomials to the selection of current magnitudes for an evenly spaced array of identical antennas placed along the z-axis.

For an even number \( 2 M \) of identical antennas placed at positions \( \Br_m = (d/2) \lr{2 m -1} \Be_3 \), the array factor is
%
\begin{equation}\label{eqn:chebychevDesign:20}
\begin{aligned}
\textrm{AF}
&=
\sum_{m =-N}^N I_m e^{j k \rcap \cdot \Br_m }.
\end{aligned}
\end{equation}
%
Assuming the currents are symmetric \( I_{-m} = I_m \), with \( \rcap = (\sin\theta \cos\phi, \sin\theta \sin\phi, \cos\theta ) \), and \( u = \frac{\pi d}{\lambda} \cos\theta \), this is
%
\begin{equation}\label{eqn:chebychevDesign:40}
\begin{aligned}
\textrm{AF}
&=
\sum_{m =-N}^N I_m e^{j k (d/2) (  2 m -1 )\cos\theta }
\\ &=
2 \sum_{m=1}^N I_m \cos\lr{ k (d/2) ( 2 m -1)\cos\theta } %
\\ &=
2 \sum_{m=1}^N I_m \cos\lr{ (2 m -1) u }.
\end{aligned}
\end{equation}
%
This is a sum of only odd cosines, and can be expanded as a sum that includes all the odd powers of \( \cos u \).  Suppose for example that this is a four element array with \( N = 2 \).  In this case the array factor has the form
%
\begin{equation}\label{eqn:chebychevDesign:60}
\begin{aligned}
\textrm{AF}
&=
2 \lr{ I_1 \cos u + I_2 \lr{ 4 \cos^3 u - 3 \cos u } }
\\ &=
2 \lr{ \lr{ I_1 - 3 I_2 } \cos u + 4 I_2 \cos^3 u }.
\end{aligned}
\end{equation}
%
The design procedure in the text sets \( \cos u = z/z_0 \), and then equates this to \( T_3(z) = 4 z^3 - 3 z \) to determine the current amplitudes \( I_m \).  That is
%
\begin{equation}\label{eqn:chebychevDesign:80}
\frac{ 2 I_1 - 6 I_2 }{z_0} z  + \frac{8 I_2}{z_0^3} z^3 = -3 z + 4 z^3,
\end{equation}
%\frac{ 2 I_1 - 3 z_0^3 }{z_0} z  + \frac{8 I_2}{z_0^3} z^3 = -3 z + 4 z^3,
%2 I_1 = 3 (z_0^2 -1) z_0

or
%
\begin{equation}\label{eqn:chebychevDesign:100}
\begin{aligned}
\begin{bmatrix}
I_1 \\
I_2
\end{bmatrix}
&=
{\begin{bmatrix}
2/z_0 & -6/z_0 \\
0 & 8/z_0^3
\end{bmatrix}}^{-1}
\begin{bmatrix}
-3 \\
4
\end{bmatrix}
\\ &=
\frac{z_0}{2}
\begin{bmatrix}
3 (z_0^2 -1) \\
z_0^2
\end{bmatrix}.
\end{aligned}
\end{equation}
%
The currents in the array factor are fully determined up to a scale factor, reducing the array factor to
%
\begin{equation}\label{eqn:chebychevDesign:140}
\textrm{AF} = 4 z_0^3 \cos^3 u - 3 z_0 \cos u.
\end{equation}
%
The zeros of this array factor are located at the zeros of
%
\begin{equation}\label{eqn:chebychevDesign:120}
T_3( z_0 \cos u ) = \cos( 3 \cos^{-1} \lr{ z_0 \cos u } ),
\end{equation}
%
which are at \( 3 \cos^{-1} \lr{ z_0 \cos u } = \pi/2 + m \pi = \pi \lr{ m + \inv{2} } \)
%
\begin{equation}\label{eqn:chebychevDesign:160}
\begin{aligned}
\cos u &= \inv{z_0} \cos\lr{ \frac{\pi}{3} \lr{ m + \inv{2} } } \\ &= \setlr{ 0, \pm \frac{\sqrt{3}}{2 z_0} }.
\end{aligned}
\end{equation}
%
showing that the scaling factor \( z_0 \) effects the locations of the zeros.  It also allows the values at the extremes \( \cos u = \pm 1 \), to increase past the \( \pm 1 \) non-scaled limit values.  
% chebychevPlotsManipulate.cdf
These effects can be explored in
\nbref{chebychevPlotsManipulate.cdf}%
%\url{https://goo.gl/KPqcjX}
, but can also be seen in \cref{fig:ChebyChevThreeScaled:ChebyChevThreeScaledFig2}.
%
\mathImageFigure{../figures/ece1229-antenna/ChebyChevThreeScaledFig2}{\( T_3( z_0 x) \) for a few different scale factors \( z_0 \).}{fig:ChebyChevThreeScaled:ChebyChevThreeScaledFig2}{0.2}{chebychevPlotsII.nb}
%
The scale factor can be fixed for a desired maximum power gain.  For \( R \si{dB} \), that will be when
%
\begin{equation}\label{eqn:chebychevDesign:180}
20 \log_{10} \cosh( 3 \cosh^{-1} z_0 ) = R \si{dB},
\end{equation}
%
or
%
\begin{equation}\label{eqn:chebychevDesign:200}
z_0 = \cosh \lr{ \inv{3} \cosh^{-1} \lr{ 10^{\frac{R}{20}} } }.
\end{equation}
%
For \( R = 30 \) dB (say), we have \( z_0 = 2.1 \), and
%
\begin{equation}\label{eqn:chebychevDesign:220}
\textrm{AF}
= 40 \cos^3 \lr{ \frac{\pi d}{\lambda} \cos\theta }  - 6.4 \cos \lr{ \frac{\pi d}{\lambda} \cos\theta }.
\end{equation}
%
These are plotted in \cref{fig:ChebychevT3Fitting:ChebychevT3FittingFig3}
%(linear scale), and
%\cref{fig:ChebychevT3FittingDb:ChebychevT3FittingDbFig4} (\si{dB} scale)
for a couple values of \( d/\lambda \).

%
\mathImageTwoFigures{../figures/ece1229-antenna/chebLinearFig1}{../figures/ece1229-antenna/chebLogScale40Fig1}{\( T_3 \) fitting of \( N = 4 \) array in linear and \si{dB} scales.}{fig:ChebychevT3Fitting:ChebychevT3FittingFig3}{scale=0.2}{cheb:c.jl}
%\mathImageFigure{../figures/ece1229-antenna/ChebychevT3FittingFig3}{\( T_3 \) fitting of \( N = 4 \) array.}{fig:ChebychevT3Fitting:ChebychevT3FittingFig3}{0.2}{polarPlot.nb}
%\mathImageFigure{../figures/ece1229-antenna/ChebychevT3FittingDbFig4}{\( T_3 \) fitting of \( N = 4 \) array (dB scale).}{fig:ChebychevT3FittingDb:ChebychevT3FittingDbFig4}{0.2}{polarPlot.nb}
%
A Manipulate for exploring the \( d/\lambda \) dependence is available in 
\nbref{chebychevN4ArrayFitManipulate.cdf}.
%\url{http://goo.gl/8FhUwC}.
} % example
%
%\EndArticle
