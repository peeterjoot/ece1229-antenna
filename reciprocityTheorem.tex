%
% Copyright � 2015 Peeter Joot.  All Rights Reserved.
% Licenced as described in the file LICENSE under the root directory of this GIT repository.
%
%\input{../blogpost.tex}
%\renewcommand{\basename}{reciprocityTheorem}
%\renewcommand{\dirname}{notes/ece1229/}
%%\newcommand{\dateintitle}{}
%%\newcommand{\keywords}{}
%
%\input{../peeter_prologue_print2.tex}
%\usepackage{peeters_layout_exercise}
%\usepackage{macros_bm}
%
%\beginArtNoToc
%
%\generatetitle{Reciprocity theorem}
%\chapter{Reciprocity theorem}
\section{Reciprocity theorem.}
\index{reciprocity theorem}
%\label{chap:reciprocityTheorem}
%
The class slides presented a derivation of the \textAndIndex{reciprocity theorem}, a theorem that contained the integral \index{curl} of
%
\begin{equation}\label{eqn:reciprocityTheorem:360}
\int \lr{ \BE^{(a)} \cross \BH^{(b)} - \BE^{(b)} \cross \BH^{(a)} } \cdot d\BS = \cdots
\end{equation}
%
over a surface, where the RHS was a volume integral involving the fields and (electric and magnetic) current sources.
The idea was to consider two different source loading configurations of the same system, and to show that the fields and sources in the two configurations can be related.
%
To derive the result in question, a simple way to start is to look at the divergence of the difference of cross products above.  This will require the \textAndIndex{phasor} form of the two cross product Maxwell's equations
%
\begin{subequations}
\label{eqn:reciprocityTheorem:99}
\begin{dmath}\label{eqn:reciprocityTheorem:100}
\spacegrad \cross \BE = - (\BM + j \omega \mu_0 \BH) % \BM^{(a)} + j \omega \mu_0 \BH^{(a)}
\end{dmath}
\begin{dmath}\label{eqn:reciprocityTheorem:120}
\spacegrad \cross \BH = \BJ + j \omega \epsilon_0 \BE, % \BJ^{(a)} + j \omega \epsilon_0 \BE^{(a)}
\end{dmath}
\end{subequations}
%
so the \textAndIndex{divergence} is
%
\begin{dmath}\label{eqn:reciprocityTheorem:380}
\begin{aligned}
\spacegrad &\cdot
\lr{ \BE^{(a)} \cross \BH^{(b)} - \BE^{(b)} \cross \BH^{(a)} } \\
&=
\BH^{(b)} \cdot \lr{ \spacegrad \cross \BE^{(a)} } -\BE^{(a)} \cdot \lr{ \spacegrad \cross \BH^{(b)} } \\
&-\BH^{(a)} \cdot \lr{ \spacegrad \cross \BE^{(b)} } +\BE^{(b)} \cdot \lr{ \spacegrad \cross \BH^{(a)} } \\
&=
-\BH^{(b)} \cdot \lr{ \BM^{(a)} + j \omega \mu_0 \BH^{(a)} } -\BE^{(a)} \cdot \lr{ \BJ^{(b)} + j \omega \epsilon_0 \BE^{(b)} } \\
&+\BH^{(a)} \cdot \lr{ \BM^{(b)} + j \omega \mu_0 \BH^{(b)} } +\BE^{(b)} \cdot \lr{ \BJ^{(a)} + j \omega \epsilon_0 \BE^{(a)} }.
\end{aligned}
\end{dmath}
%
The non-source terms cancel, leaving
%
\begin{equation}\label{eqn:reciprocityTheorem:440}
%\boxedEquation{eqn:reciprocityTheorem:440}{
\begin{aligned}
\spacegrad &\cdot
\lr{ \BE^{(a)} \cross \BH^{(b)} - \BE^{(b)} \cross \BH^{(a)} } \\
&=
-\BH^{(b)} \cdot \BM^{(a)} -\BE^{(a)} \cdot \BJ^{(b)}
+\BH^{(a)} \cdot \BM^{(b)} +\BE^{(b)} \cdot \BJ^{(a)}
.
%}
\end{aligned}
\end{equation}
%
Should we be surprised to have a relation of this form?  Probably not, given that the energy momentum relationship between the fields and currents of a single source has the form
%
\begin{equation}\label{eqn:reciprocityTheorem:600}
\PD{t}{}\frac{\epsilon_0}{2} \left(\bcE^2 + c^2 \bcB^2\right) + \spacegrad \cdot \inv{\mu_0}(\bcE \cross \bcB) = -\bcE \cdot \bcJ.
\end{equation}
%
(this is without magnetic sources \index{magnetic source}).
%
This initially suggests that the reciprocity theorem can be expressed more generally in terms of the energy-momentum tensor.
However, there are some subtle differences since the time domain products lead to averages in terms of the real parts of conjugate pairs such as \( \bcE \cross \bcB \rightarrow \BE \cross \BB^\conj \), and \( \bcE \cdot \bcJ \rightarrow \BE \cdot \BJ^\conj \).
%
\paragraph{Far field integral form.}
\index{far field}
Employing the \textAndIndex{divergence theorem} over a sphere the identity above takes the form
%
\begin{dmath}\label{eqn:reciprocityTheorem:480}
\int_S
\lr{ \BE^{(a)} \cross \BH^{(b)} - \BE^{(b)} \cross \BH^{(a)} } \cdot \rcap dS
=
\int_V \lr{
-\BH^{(b)} \cdot \BM^{(a)} -\BE^{(a)} \cdot \BJ^{(b)}
+\BH^{(a)} \cdot \BM^{(b)} +\BE^{(b)} \cdot \BJ^{(a)}
}
dV
\end{dmath}
%
In the far field, the cross products are strictly radial.  That surface integral can be written as
%
\begin{dmath}\label{eqn:reciprocityTheorem:500}
\int_S
\lr{ \BE^{(a)} \cross \BH^{(b)} - \BE^{(b)} \cross \BH^{(a)} } \cdot \rcap dS
=
\inv{\mu_0}
\int_S
\lr{ \BE^{(a)} \cross \lr{ \rcap \cross \BE^{(b)}} - \BE^{(b)} \cross \lr{ \rcap \cross \BE^{(a)}} } \cdot \rcap dS
=
\inv{\mu_0}
\int_S
\lr{ \BE^{(a)} \cdot \BE^{(b)} - \BE^{(b)} \cdot \BE^{(a)}
}
dS
= 0.
\end{dmath}
%
The above expansions used \cref{eqn:reciprocityTheorem:540} to expand the terms of the form
%
\begin{dmath}\label{eqn:reciprocityTheorem:560}
\lr{ \BA \cross \lr{ \rcap \cross \BC } } \cdot \rcap
= \BA \cdot \BC -\lr{ \BA \cdot \rcap } \lr{ \BC \cdot \rcap },
\end{dmath}
%
in which only the first dot product survives due to the \index{transverse field} transverse nature of the fields.
%
So in the far field we have a direct relation between the fields and sources of two source configurations of the same system of the form

\boxedEquation{eqn:reciprocityTheorem:580}{
%\begin{dmath}\label{eqn:reciprocityTheorem:580}
%\boxed{
\int_V \lr{
\BH^{(a)} \cdot \BM^{(b)} +\BE^{(b)} \cdot \BJ^{(a)}
}
dV
=
\int_V \lr{
\BH^{(b)} \cdot \BM^{(a)} +\BE^{(a)} \cdot \BJ^{(b)}
}
dV.
%}
%\end{dmath}
}
%
\paragraph{Application to antenna theory.}
%
This is the underlying reason that we are able to pose the problem of what an antenna can receive, in terms of what the antenna can transmit.
%
Prof. Eleftheriades explained the the send-receive equivalence using the concepts of a two-port network (\citep{irwin2007bec}, \citep{sedra1982microelectronic}).
%
An alternate, and very intuitive, explanation can be found in appendix A.1 \citep{chen2005reciprocity}, that directly related the current density \index{current density} sources and scalar current to the voltages in those regions using an integral representation of the reciprocity theorem.
%
\paragraph{Identities.}
%
\index{divergence!cross product}
\makelemma{Divergence of a cross product}{thm:polarizationReview:400}{
\begin{equation*}
\spacegrad \cdot \lr{ \BA \cross \BB } =
\BB \lr{\spacegrad \cross \BA}
-\BA \lr{\spacegrad \cross \BB}.
\end{equation*}
}

Proof.
%
\begin{dmath}\label{eqn:reciprocityTheorem:420}
\spacegrad \cdot \lr{ \BA \cross \BB }
=
\partial_a \epsilon_{a b c} A_b B_c
=
\epsilon_{a b c} (\partial_a A_b )B_c
-
\epsilon_{b a c} A_b (\partial_a B_c)
=
\BB \cdot (\spacegrad \cross \BA)
-\BB \cdot (\spacegrad \cross \BA).
\end{dmath}
%
\makelemma{Triple cross product dotted}{thm:polarizationReview:520}{
\begin{equation*}
\lr{ \BA \cross \lr{ \BB \cross \BC } } \cdot \BD
=
\lr{ \BA \cdot \BC } \lr{ \BB \cdot \BD }
-\lr{ \BA \cdot \BB } \lr{ \BC \cdot \BD }.
\end{equation*}
}
Proof.
%
\begin{dmath}\label{eqn:reciprocityTheorem:540}
\lr{ \BA \cross \lr{ \BB \cross \BC } } \cdot \BD
=
\epsilon_{a b c} A_b \epsilon_{r s c } B_r C_s D_a
=
\delta_{[a b]}^{r s}
A_b B_r C_s D_a
=
A_s B_r C_s D_r
-A_r B_r C_s D_s
=
\lr{ \BA \cdot \BC } \lr{ \BB \cdot \BD }
-\lr{ \BA \cdot \BB } \lr{ \BC \cdot \BD }.
\end{dmath}
%
%\EndArticle
