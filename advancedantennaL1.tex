%
% Copyright � 2015 Peeter Joot.  All Rights Reserved.
% Licenced as described in the file LICENSE under the root directory of this GIT repository.
%
\input{../latex/blogpost.tex}
\renewcommand{\basename}{advancedantennaL1}
\renewcommand{\dirname}{notes/ece1229/}
\newcommand{\keywords}{ECE1229H}
\input{../latex/peeter_prologue_print2.tex}

%\usepackage{kbordermatrix}
\usepackage{esint} % oiint
\usepackage{peeters_layout_exercise}
\usepackage{siunitx}

\newcommand{\tiso}[0]{{\textrm{iso}}}
\newcommand{\tin}[0]{{\textrm{in}}}
\newcommand{\trad}[0]{{\textrm{rad}}}
\newcommand{\tmax}[0]{{\textrm{max}}}
\newcommand{\tA}[0]{{\textrm{A}}}
\newcommand{\tav}[0]{{\textrm{av}}}
\newcommand{\tdB}[0]{{\textrm{dB}}}
\newcommand{\tdBi}[0]{{\textrm{dBi}}}

\newcommand{\timeaverage}[1]{\left[#1\right]}

\beginArtNoToc

\generatetitle{ECE1229H Advanced Antenna Theory.  Lecture 1: Introduction.  Taught by Prof.\ G.V. Eleftheriades}
%\chapter{Introduction}
\label{chap:advancedantennaL1}
%
%\section{Disclaimer}
%
%Peeter's lecture notes from class.  These may be incoherent and rough.
%
\section{What is an antenna?}
%
An antenna is a linear passive device that converts a bounded EM wave to an unbounded one.

%FIXME: F1
We can think of antennas as sensors of electromagnetic radiation.

Issues of interest:
\begin{itemize}
\item don't loose power
\item directional reception when desired
\end{itemize}

If we have a source we can do the opposite, using the antenna to broadcast instead of receive.  This is, in fact, how all the antenna discussion to follow proceeds, where we talk of the power radiated by the antenna.  Perhaps this is because it is easier to develop the mathematics for the field produced by a moving charge distribution, than the mathematics to determine the currents produced by a field acting on a free charge distribution?
%
\section{Antenna Pattern}
%
\begin{itemize}
\item A mathematical function (graphical representation) describing the radiation properties of the antenna as a function of space coordinates
\item Usually this is done in the far field for the power flux density, radiation indesity, or field strength as a function of directional coordinates \(\lr{ \theta, \phi }\).
\end{itemize}
%
%\section{Description of various antennas from google image search}
%
%F2
%
%Monopole:
%
%(picture on )
%
%Can make the antenna length only \( \lambda/4 \) if a portion reflects off the ground.
%
\section{Principle Planes}
\index{principle plane}
%
For linearly polarized antennas we define the \( \BE \) and \( \BH \) principle planes

\begin{itemize}
\item \underline{\(\BE\)-plane}: contains the electric field vector and the maximum direction of radiation.
\item \underline{\(\BH\)-plane}: contains the magnetic field vector and the maximum direction of radiation.
\end{itemize}

See \textfigref{2.2, 2.3} in \citep{balanis2005antenna} for a couple examples of the plane patterns.
%
\section{Radiation Power Density}
\index{radiation power density}
%
Time average Poynting vector (average power density, \( W/m^2 \))
%
\begin{dmath}\label{eqn:advancedantennaL1:380}
\BW_\tav(x,y,z) = \inv{2} \Real \timeaverage{ \BE \cross \BH^\conj}.
\end{dmath}
%
Here \( \BE, \BH \) are phasors (peak values).

In the far-field \( \BW_\tav \) is real and is referred to as the \underlineAndIndex{radiation density}

The average power (W) radiated by an antenna is
%
\begin{dmath}\label{eqn:advancedantennaL1:400}
P_\trad = P_\tav = \oiint_S \BW_\tav \cdot \ncap ds = \inv{2} \oiint \Real \timeaverage{ \BE \cross \BH^\conj } \cdot d \Bs
\end{dmath}
%
The integration element is illustrated in \cref{fig:farfieldsurfaceElement:farfieldsurfaceElementFig1}.

%\imageFigure{../figures/ece1229-antenna/farfieldsurfaceElementFig1}{Far field surface}{fig:farfieldsurfaceElement:farfieldsurfaceElementFig1}{0.2}
%
\section{Isotropic radiator (antenna)}
\index{isotropic radiator}
%
It radiates equally in all directions.  Such an antenna does not really exist, but is a convient fiction with an average power density
%
\begin{dmath}\label{eqn:advancedantennaL1:420}
\BW_\tiso = \rcap \frac{P_0}{4 \pi r^2 } \qquad (W/m^2)
\end{dmath}
%
Note that this has no angular dependence.

The total radiated power, integrating over a far field spherical surface is
%
\begin{dmath}\label{eqn:advancedantennaL1:440}
\oiint_S \BW_\tiso \cdot \rcap d\Bs
= \frac{P_0}{4 \pi} \oiint \inv{r^2} ds
= \frac{P_0}{4 \pi} \int_0^{2 \pi} \int_0^\pi r^2 \inv{r^2} \sin\theta d\theta d\phi
= \frac{P_0}{4 \pi} \iint d\Omega
= \frac{P_0}{4 \pi} 4 \pi
= P_0.
\end{dmath}
%
Here \( d\Omega = \sin\theta d\theta d\phi \) is an element of solid angle.
%
\section{Radiation intensity}
\index{radiation intensity}
%
Power radiated per unit solid angle
%
\begin{dmath}\label{eqn:advancedantennaL1:460}
U = r^2 \BW_\tav \qquad (W/\text{solid-angle})
\end{dmath}
%
\section{Typical far-field radiation intensity}
\index{radiation intensity!far field}
%
\begin{dmath}\label{eqn:advancedantennaL1:20}
U(\theta, \phi)
=
\frac{r^2}{2 \eta_0} \Abs{ \BE(r, \theta, \phi) }^2
\approx
\frac{r^2}{2 \eta_0}
\lr{
\Abs{ E_\theta(r, \theta, \phi) }^2
+
\Abs{ E_\phi(r, \theta, \phi) }^2
},
\end{dmath}
%
where the intrinsic impedance of free space is
%
\begin{dmath}\label{eqn:advancedantennaL1:480}
\eta_0 = \sqrt{\frac{\mu_0}{\epsilon_0}} = 377 \Omega
\end{dmath}
%
For a spherical wave write
%
\begin{dmath}\label{eqn:advancedantennaL1:500}
\BE(r, \theta, \phi) = \BE^0\lr{ \theta, \phi } \frac{e^{-j k r}}{r}
\end{dmath}
%
so that
%
\begin{dmath}\label{eqn:advancedantennaL1:40}
P_\trad
= \oiint_\Omega U\lr{ \theta, \phi } d\Omega
= \int_0^{2 \pi} \int_0^\pi U \sin\theta d\theta d\phi.
\end{dmath}
%
For an isotropic antenna where \( P_0 = P_\trad \),
%
\begin{dmath}\label{eqn:advancedantennaL1:520}
U_\tiso = r^2 \BW_\tiso = \frac{P_0}{2 \pi}
\end{dmath}
%
FIXME: \( \BW_\tiso \) a vector, \(U_\tiso\) not?
%
\section{Directivity}
\index{directivity}
%
The ratio of the radiation intensity in a given direction from an antenna to the radition intensity of an isotropic source (radiating the same power \(P_\trad\)) is
%
\begin{dmath}\label{eqn:advancedantennaL1:60}
D(\theta, \phi)
= \frac{U(\theta, \phi)}{U_\tiso}
= \frac{U(\theta, \phi)}{
\mathLabelBox
{
P_\trad/{4 \pi}
}
{average radiation intensity}
}
= 4 \pi \frac{U(\theta, \phi)}{P_\trad} \qquad (\text{dimensionless})
\end{dmath}
%
where \( P_\trad \) is defined by \cref{eqn:advancedantennaL1:160}.
%
\begin{equation}\label{eqn:advancedantennaL1:80}
D_{max} = 4 \pi \frac{U_{max}}{P_\trad}
\end{equation}
%
Many times \( D \) is calculated at the direction of maximum intensity
%
\begin{dmath}\label{eqn:advancedantennaL1:540}
D_\tmax = 4 \pi \frac{U_\tmax}{P_\trad} \qquad (\text{dimensionless})
\end{dmath}
%
What is the power density (\(W/m^2\)) radiated at point \( (r, \theta) \)?

F1
\begin{equation}\label{eqn:advancedantennaL1:100}
W(\theta) =
\mathLabelBox
[
   labelstyle={below of=m\themathLableNode, below of=m\themathLableNode}
]
{
\frac{P_\trad}{4 \pi r^2}
}
{isotropic radiator}
\mathLabelBox
[
   labelstyle={xshift=2cm},
   linestyle={out=270,in=90, latex-}
]
{
D(\theta).
}
{Directive gain}
\end{equation}
%
Calling this gain is somewhat misrepresentitive, since the antenna does not amplify at all.  However, since it is able to focus the power (not radiated isotropically), there will typically be larger than the isotropic intensity at the points where the field is found.
%
\makeexample{Radiated power ensity of an infitesimal dipole.}{example:advancedantennaL1:1}{
%
\begin{dmath}\label{eqn:advancedantennaL1:120}
\BW_\tav = \rcap \frac{A_0 \sin^2 \theta}{r^2}
\end{dmath}
%
F5
%
\begin{dmath}\label{eqn:advancedantennaL1:140}
U(\theta, \phi) = r^2 W_{av} = A_0 \sin^2 \theta
\end{dmath}
%
(radiation intensity pattern)
%
\begin{dmath}\label{eqn:advancedantennaL1:560}
D(\theta, \phi) = \frac{4 \pi U}{P_\trad},
\end{dmath}
%
\begin{dmath}\label{eqn:advancedantennaL1:160}
P_\trad
= \oiint_\Omega U\lr{ \theta, \phi } d\Omega
= \int_0^{2 \pi} \int_0 A_0 \sin^2 \theta \sin \theta d\theta d\phi
= A_0 \frac{8 \pi}{3}
\end{dmath}
%
Hence
\begin{dmath}\label{eqn:advancedantennaL1:180}
D\lr{ \theta, \phi }
= \frac{ 4 \cancel{\pi} \cancel{A_0} \sin^2 \theta }{ \cancel{A_0} \frac{8 \cancel{\pi}}{3}}
= \frac{3}{2} \sin^2 \theta
\end{dmath}
%
Maximum directivity at \( \theta = 0 \), is \( D = 3/2 = 1.5 \).
%
\begin{dmath}\label{eqn:advancedantennaL1:200}
10 \log_{10} D = 1.76 \, \tdBi
\end{dmath}
%
\paragraph{Important note}
%
Integrating over a far field sphere
%
\begin{dmath}\label{eqn:advancedantennaL1:220}
\iint U d\Omega = P_\trad \implies
\end{dmath}
%
\begin{equation}\label{eqn:advancedantennaL1:240}
\iint D d\Omega = \frac{4 \pi}{P_\trad} \iint U d\Omega =
\mytikzmark{left}{1}
4 \pi
\mytikzmark{right}{1}
\end{equation}
%
\DrawMyBox[thick, Maroon, rounded corners]{1}
This can be used to check that the directivity is correct.
}
%
\section{Maximum directivity}
\index{directivity!maximum}
%
The maximum directivity is given by
%
\begin{dmath}\label{eqn:advancedantennaL1:260}
D_0 = 4 \pi \frac{U_\tmax}{P_\trad} = 4 \pi \frac{U\lr{ \theta, \phi }_\tmax }{
\oiint_\Omega U\lr{ \theta, \phi } d\Omega
}
\end{dmath}
%
Here \( \Omega \) is the \( 4 \pi \) solid angle
%
\begin{dmath}\label{eqn:advancedantennaL1:280}
4 \pi = \int_0^{2 \pi} \int_0^\pi \sin\theta d\theta d\phi = \iint d\Omega.
\end{dmath}
%
Let us define \( \Omega_\tA \), a beam solid angle for the antenna
%
\begin{dmath}\label{eqn:advancedantennaL1:300}
\Omega_\tA = \inv{U\lr{ \theta, \phi }_\tmax} \int_0^{2\pi} \int_0^\pi U\lr{ \theta, \phi } \sin\theta d\theta d\phi
\end{dmath}
%
Fx.

Then
\begin{dmath}\label{eqn:advancedantennaL1:320}
D_0 = \frac{4 \pi}{\Omega_\tA}
\end{dmath}
%
i.e the ration of the entire solid angle (\(4 \pi\)) to that of the main beam of the antenna.
%
\begin{dmath}\label{eqn:advancedantennaL1:580}
D_0 = \frac{4 \pi}{\Omega_\tA}
\approx \frac{4 \pi}{\theta_{1r} \theta_{2r}}
\end{dmath}
%
where the beam solid angle is approximated (good for narrow beams with directive antennas) by
%
\begin{dmath}\label{eqn:advancedantennaL1:600}
\Omega_\tA \approx \theta_{1r} \theta_{2r}
\end{dmath}
%
The values \( \theta_{1r}, \theta_{2r} \) are the half power beamwidth angles in two orthoganal planes (typically the \(\BE\) and \(\BH\) planes respectively.)
%
\makeexample{Upper half plane radiator.}{example:advancedantennaL1:620}{
%
Find the directivity, given an upper half plane \( 0 \le \theta \le \pi/2 \), \( 0 \le \phi \le 2 \pi \) radiator of the form
%
\begin{dmath}\label{eqn:advancedantennaL1:340}
U = A_0 \cos\theta,
\end{dmath}
%
At \( \theta = \pi/3 \) (\ang{60}), \( U = A_0/2 \) which is the half-max power point.  Thus the full beamwidth at half-maximum is
%
\begin{dmath}\label{eqn:advancedantennaL1:640}
\theta_{1r} = \theta_{2r} = \frac{2 \pi}{3}.
\end{dmath}
%
(These are the \underline{full} half-max angles)

If we use the approximate formula
%
\begin{dmath}\label{eqn:advancedantennaL1:660}
D_0 = \frac{4 \pi}{
\theta_{1r} \theta_{2r} }
=
\frac{4\pi}{\lr{2\pi/3}^2} = 2.86.
\end{dmath}
%
Exact solution is

%U = A_0 \cos\theta
\begin{dmath}\label{eqn:advancedantennaL1:680}
U_\tmax = A_0 \qquad (\theta = 0)
\end{dmath}
\begin{dmath}\label{eqn:advancedantennaL1:700}
P_\trad
= \int_0^{2 \pi} \int_0^\pi A_0 \cos\theta \sin\theta d\theta d\phi
= 2 \pi A_0 \int_0^{\pi/2} \sin\theta \cos\theta d\theta
= \pi A_0 \int_0^{\pi/2} \sin 2\theta d\theta
= \pi A_0.
\end{dmath}
%
so,
\begin{dmath}\label{eqn:advancedantennaL1:720}
D_0 = \frac{ 4\pi A_0 }{\pi A_0} = 4.
\end{dmath}
%
This is bigger than the \( 2.86 \) approximation, since this is a wide beam.  That approximation works better for a more directed radiator.
}
%
\section{Directivity measurement}
\index{directivity!dB}
%
Directivity is measured in dB
%
\begin{dmath}\label{eqn:optimizationToyProblem:n}
D_\tdB = 10 \log_{10}(
\mathLabelBox
[
   labelstyle={xshift=2cm},
   linestyle={out=270,in=90, latex-}
]
{
D
}
{dimensionless}
).
\end{dmath}
%
If we consider more directive patterns of the form,
%
\begin{dmath}\label{eqn:optimizationToyProblem:n}
U = \cos^n \theta,
\end{dmath}
%
it can be found that the Directivity approximation above generally gets better as \( n \) increases.  An alternate approximation (the Tai and Pereira equation), given by
%
\begin{dmath}\label{eqn:optimizationToyProblem:n}
D_0 \approx \frac{32 \ln 2}{\theta_{1r}^2 + \theta_{2r}^2},
\end{dmath}
%
also improves with \( n \).  Values for both are tabulated in \
\citep{balanis2005antenna}
tab. 2.1.
%
%%%\section{Directivity vs. Gain}
%%%
%%%Gain = 4 \pi \frac{Uthetaphi}{P_\tin}
%%%
%%%F6
%%%
%%%Due to material losses in the antenna
%%%
%%%P_\trad < P_\tin and so
%%%
%%%D = ...
%%%
%%%To include mismatch losses we can use
%%%
%%%\begin{dmath}\label{eqn:advancedantennaL1:360}
%%%\eta_\rad \lr{ 1 - \Abs{\Gamma}^2 }
%%%\end{dmath}
%%%
%%%instead where \( \Gamma \) is the
%%%
%%%\section{Polarization loss factor (PLF)}
%%%
%%%Assume that the incient electric field is of the form
%%%
%%%\BE_i = \rhocap_i E_i
%%%
%%%Assume that our recieviing antenna when it transmits has a field of the form
%%%
%%%\BE_a = \rhocap_a E_a
%%%
%%%\example Linearly polarized antennas
%%%
%%%Let the incident field be
%%%
%%%\BE_i = \xcap E_0(x, y) e^{ j k z }
%%%
%%%F7
%%%
%%%and the antenna field is given by
%%%
%%%\BE_a = \lr{ \xcap + \ycap } E \lr{ r, \theta, \phi }
%%%
%%%Then \( PLF = \Abs{ \rhocap_i \cdot \rhocap_a }^2  \) where
%%%
%%%\rhocap_i = \xcap
%%%
%%%...
%%%
%%%Then PLF = \Abs{ \xcap \cdot \lr{ \xcap + \ycap } }
%%%
%%%\section{RH/LH Polarized Antennas}
%%%
%%%Can build a circular polarized source with
%%%
%%%F9
%%%
%%%Stand on an antenna s as transmitter.  For such an antenna radiating along the \( + \zcap \) axis
%%%
%%%F8
%%%
%%%Let \( \rho_a = \lr{ \xcap + j \ycap }/ \sqrt{2}
%%%
%%%then in the time domain \( e^{j \omega t} \) :
%%%
%%%\Real{ [\BE_a e^{j \omega t} ] } \implies
%%%
%%%[ \xcap \cos(\omega t) + \ycap \sin(\omega t) ] \inv{\sqrt{2}}
%%%
%%%At \( \omega t = 0 \), \( \BE_a \propto \xcap \)
%%%At \( \omega t = \pi/4 \), \( \BE_a \propto \xcap + \ycap \)
%%%At \( \omega t = \pi/2 \), \( \BE_a \propto \ycap \)
%%%
%%%This is circular polarization, but is it left or right handed?
%%%
%%%As time increases
%%%
%%%\example ( CP antennas) :
%%%
%%%F10
%%%
%%%%problem set 1 due in two weeks.
%%%XX

\EndArticle
