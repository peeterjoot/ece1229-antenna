%
% Copyright � 2015 Peeter Joot.  All Rights Reserved.
% Licenced as described in the file LICENSE under the root directory of this GIT repository.
%
\makeoproblem{Mobile power reception.}{advancedantenna:problemSet2:5}{2015 ps2, p5}{
\index{mobile reception}
%
A mobile is located 5 \si{km} from a base station and uses a vertical wire antenna of gain 2.55 \si{dB} to receive cellular radio signals.
The carrier frequency used is 900 \si{MHz} and the EIRP of the base station is 30 \si{mW}.
If the base station and mobile are located 50 \si{m} and 1.5 \si{m} above ground respectively, calculate the power level received at the mobile.

} % makeoproblem
%
\makeanswer{advancedantenna:problemSet2:5}{
%
The mobile in a transmitting geometry is sketched in \cref{fig:verticalDipoleReflection:verticalDipoleReflectionFig1}.

\imageFigure{../figures/ece1229-antenna//verticalDipoleReflectionFig1}{Vertical dipole reflection geometry.}{fig:verticalDipoleReflection:verticalDipoleReflectionFig1}{0.2}

For a vertical dipole the magnetic vector potential has the form
%
\begin{dmath}\label{eqn:advancedantennaProblemSet2Problem5:20}
\BA = \zcap \frac{A_0}{r} e^{-j k r},
\end{dmath}
%
where \( A_0 = \mu_0 I_0 l /4 \pi \).  The far field electric field is
%
\begin{dmath}\label{eqn:advancedantennaProblemSet2Problem5:40}
\BE = - j \omega \BA_\T
= -j \omega
\frac{A_0}{r} e^{-j k r}
\lr{ \zcap - \lr{\zcap \cdot \kcap } \kcap },
\end{dmath}
%
where all the radial (non-transverse) components of the magnetic vector potential have been subtracted out.
%
\paragraph{Reflection coefficient}
\index{reflection coefficient}
%
To determine the sign of the reflection coefficient for a vertical dipole configuration, consider a wave vector directed in the z-y plane at an angle \( \theta \) from the pole
%
\begin{equation}\label{eqn:advancedantennaProblemSet2Problem5:60}
\kcap = \zcap e^{\zcap \ycap \theta} = \zcap \cos\theta + \ycap \sin\theta.
\end{equation}
%
The far field electric field that propagates along this direction, has direction
%
\begin{dmath}\label{eqn:advancedantennaProblemSet2Problem5:80}
\zcap - \lr{\zcap \cdot \kcap } \kcap
=
\zcap - \lr{\zcap \cdot \lr{
\zcap \cos\theta + \ycap \sin\theta
}} \kcap
=
\zcap -
\cos\theta \kcap
=
\zcap -
\cos\theta
\lr{
\zcap \cos\theta + \ycap \sin\theta
}
=
\zcap \sin^2 \theta - \sin\theta \cos\theta \ycap
=
-\sin\theta \lr{ \cos\theta \ycap - \sin\theta \zcap }.
\end{dmath}
%
When there is reflection, the electric (far) field is directed entirely in the plane of incidence (with the magnetic field entirely parallel to the reflecting interface).  The Fresnel reflection coefficient (\citep{hecht1998hecht} \texteqnref{4.40}) for such a polarization is
%
\begin{dmath}\label{eqn:advancedantennaProblemSet2Problem5:100}
R =
\frac{
n_\txtt \cos\theta_\txti - n_\txti \cos\theta_\txtt
}
{
n_\txti \cos\theta_\txti + n_\txtt \cos\theta_\txtt
}.
\end{dmath}
%
For no transmission the transmitted speed of the radiation \( v_\txtt \rightarrow 0 \), and the the index of refraction of the ground approaches \( n_\txtt = c/v_\txtt \rightarrow \infty \).  This shows that the reflection coefficient for the vertical dipole configuration is \( + 1 \).  Because of the symmetry of this dipole's orientation the sign of the reflection coefficient has no azimuthal dependency. % (as it appears to for the horizontal dipole, despite the use of \( -1 \) for all azimuthal orientations in \citep{} )
%
\paragraph{Effects of ground reflection}
\index{ground reflection}
%
%An exact vector sum of the reflected and line of sight field components is possible.
Let
%
\begin{subequations}
\label{eqn:advancedantennaProblemSet2Problem5:120}
\begin{dmath}\label{eqn:advancedantennaProblemSet2Problem5:140}
\alpha_{\textrm{ref}} = \arctan 55/5000
\end{dmath}
\begin{dmath}\label{eqn:advancedantennaProblemSet2Problem5:160}
\alpha_{\textrm{los}} = \arctan 45/5000
\end{dmath}
\begin{dmath}\label{eqn:advancedantennaProblemSet2Problem5:180}
r_{\textrm{ref}} = \sqrt{55^2 + 5000^2}
\end{dmath}
\begin{dmath}\label{eqn:advancedantennaProblemSet2Problem5:200}
r_{\textrm{los}} = \sqrt{45^2 + 5000^2}
\end{dmath}
\begin{dmath}\label{eqn:advancedantennaProblemSet2Problem5:220}
\kcap_{\textrm{ref}} = \ycap \cos \alpha_{\textrm{ref}} + \zcap \sin\alpha_{\textrm{ref}}
\end{dmath}
\begin{dmath}\label{eqn:advancedantennaProblemSet2Problem5:240}
\kcap_{\textrm{los}} = \ycap \cos \alpha_{\textrm{los}} + \zcap \sin\alpha_{\textrm{los}},
\end{dmath}
\label{eqn:advancedantennaProblemSet2Problem5:260}
\end{subequations}
%
Summing the line of sight and reflected (image source contribution) gives
\index{line of sight}
%
\begin{dmath}\label{eqn:advancedantennaProblemSet2Problem5:280}
\BE =
j \omega A_0
\sum_{i \in \setlr{\textrm{los}, \textrm{ref}}}
\inv{r_i} e^{-j k r_i}
\cos\alpha_i \lr{ \sin\alpha_i \ycap - \cos\alpha_i \zcap }
\approx
j \omega A_0
\cos\theta_\txtA \lr{ \sin\theta_\txtA \ycap - \cos\theta_\txtA \zcap }
\inv{r}
\sum_{i \in \setlr{\textrm{los}, \textrm{ref}}}
e^{-j k r_i},
\end{dmath}
%
where an average distance \( r = \sqrt{ h_\txtt^2 + d^2 } = \sqrt{ 50^2 + 5000^2} \), the distance from the origin to the base station has been factored out in the denominator.

The sum of the phase terms is
%
\begin{dmath}\label{eqn:advancedantennaProblemSet2Problem5:300}
e^{-j k \sqrt{ r^2 + 2 h_\txtt h_\txtr + h_\txtr^2}}
+
e^{-j k \sqrt{ r^2 - 2 h_\txtt h_\txtr + h_\txtr^2}}
\approx
e^{-j k \sqrt{ r^2 + 2 h_\txtt h_\txtr }}
+
e^{-j k \sqrt{ r^2 - 2 h_\txtt h_\txtr }}
=
e^{-j k r \sqrt{ 1 + 2 h_\txtt h_\txtr/r^2 }}
+
e^{-j k r \sqrt{ 1 - 2 h_\txtt h_\txtr/r^2 }}
\approx
e^{-j k r}
\lr{
e^{ -j k h_\txtt h_\txtr/r }
+e^{ j k h_\txtt h_\txtr/r }
}
=
2 e^{-j k r} \cos\lr{ \frac{k h_\txtt h_\txtr}{r} },
\end{dmath}
%
so after reflection the far field electric field has the form
\index{ground reflection}
%
\begin{dmath}\label{eqn:advancedantennaProblemSet2Problem5:420}
\BE =
j \omega A_0
\cos\theta_\txtA \lr{ \sin\theta_\txtA \ycap - \cos\theta_\txtA \zcap }
\inv{r} 2 e^{-j k r} \cos\lr{ \frac{k h_\txtt h_\txtr}{r} }.
\end{dmath}
%
This differs from the line of sight field by a factor of \( 2 \cos\lr{ \ifrac{k h_\txtt h_\txtr}{r} } \).
%
\paragraph{Numerical results}
%
The wavelength is
%
\begin{equation}\label{eqn:advancedantennaProblemSet2Problem5:320}
\lambda = c/\nu = \frac{3 \times 10^8 \si{m/s}}{900 \times 10^6 \si{s^{-1}}} = 0.33 \si{m},
\end{equation}
%
so the cosine argument is
\begin{dmath}\label{eqn:advancedantennaProblemSet2Problem5:340}
k h_\txtt h_\txtr /r
=
\frac{2 \pi \times 50 \times 1.5
}{
0.33 \times 5000.25
}
=
0.28,
\end{dmath}
%
and the cosine adjustment to the field strength is
\begin{dmath}\label{eqn:advancedantennaProblemSet2Problem5:360}
2 \cos 0.28 = 1.92.
\end{dmath}
%
The mobile gain is
%
\begin{equation}\label{eqn:advancedantennaProblemSet2Problem5:380}
G = 10^{2.55 \si{dB}/10} = 1.8,
\end{equation}
%
Noting that \( \textrm{EIRP} = P_\txtt G_\txtt \), the Friis transmission equation, after adjusting for the reflection effects, provides the power at the mobile
\index{IERP}
\index{Friis equation}
%
\begin{dmath}\label{eqn:advancedantennaProblemSet2Problem5:400}
P_\txtr
= \lr{ \frac{\lambda}{4 \pi r} }^2 \lr{ P_\txtt G_\txtt } G_\txtr (1.92)^2
= \lr{ \frac{0.33}{4 \pi \, 5000.25} }^2 \lr{ 30 \times 10^{-3} } \times 1.8 \times (1.92)^2 \si{W}
= 5.6 \times 10^{-12} \,\si{W}.
\end{dmath}
%
The total power received is just 5.6 \si{pW}, assuming no polarization losses (we know the polarization at the mobile, but not for the base station.)
\index{vertical dipole!polarization}
%
In an attempt to avoid calculator errors for this problem I scripted the numerical calculations in \nbref{ps2p5.jl}.  That wasn't entirely successful on submission, since I used 5 \si{m} instead of 1.5 \si{m}!

%\href{https://github.com/peeterjoot/julia/blob/master/ece1229/ps2p5.jl}{https://github.com/peeterjoot/julia/blob/master/ece1229/ps2p5.jl}
%
%\href{http://julialang.org/}{Julia} or Matlab can be used to evaluate it.
\index{Julia}
\index{Matlab}
}
