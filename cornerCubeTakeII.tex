%
% Copyright � 2015 Peeter Joot.  All Rights Reserved.
% Licenced as described in the file LICENSE under the root directory of this GIT repository.
%
%\input{../blogpost.tex}
%\renewcommand{\basename}{cornerCubeTakeII}
%\renewcommand{\dirname}{notes/ece1229/}
%%\newcommand{\dateintitle}{}
%%\newcommand{\keywords}{}
%
%\input{../peeter_prologue_print2.tex}
%
%\usepackage{peeters_layout_exercise}
%%\usepackage{macros_mathematica}
%\usepackage{ece1229}
%\usepackage{siunitx}
%\usepackage{esint} % \oiint
%
%\renewcommand{\QuestionNB}{\alph{Question}.\ }
%\renewcommand{\theQuestion}{\alph{Question}}
%
%\beginArtNoToc
%
%\generatetitle{Corner cube image factor}
%\chapter{Corner cube array factor}
\index{corner cube}
\index{array factor}
%\label{chap:cornerCubeTakeII}
\makeoproblem{Corner cube antenna.}{advancedantenna:problemSet3TakeII:3}{2015 problem set 3, p3}{
\index{corner cube antenna}
Consider the symmetrically placed horizontal dipole antenna of
\cref{fig:homework3:homework3Fig1}{0.2}
, next to a metallic corner cube.
%\imageFigure{../figures/ece1229-antenna/homework3Fig1}{A corner-cube antenna.}{fig:homework3:homework3Fig1}{0.2}
\makesubproblem{}{advancedantenna:problemSet3TakeII:3c}
Calculate the array factor of the antenna in \cref{fig:homework3:homework3Fig1}.
\makesubproblem{}{advancedantenna:problemSet3TakeII:3a}
Estimate the directivity enhancement of the antenna in \cref{fig:homework3:homework3Fig1} compared to the isolated antenna.
\makesubproblem{}{advancedantenna:problemSet3TakeII:3b}
Estimate the radiation resistance of the antenna in \cref{fig:homework3:homework3Fig1} compared to the isolated antenna.
\makesubproblem{}{advancedantenna:problemSet3TakeII:3d}
Plot the array-factor directivity pattern in the x-y plane for \( 0 < \phi \le 2 \pi \).
\makesubproblem{}{advancedantenna:problemSet3TakeII:3e}
By using numerical integration calculate the directivity of the array factor for \( h = (1/8) \lambda, h = (1/4) \lambda \) and \( h = (1/2) \lambda \).
} % makeoproblem
\makeanswer{advancedantenna:problemSet3TakeII:3}{
\makeSubAnswer{}{advancedantenna:problemSet3TakeII:3c}
This problem can be tackled with the image theorem, which requires placement of sources as in \cref{fig:cornerCubeImageSourcePlacement:cornerCubeImageSourcePlacementFig3}.
% FIXME: duplicate.
%\imageFigure{../figures/ece1229-antenna/cornerCubeImageSourcePlacementFig3}{Correct image source placement for the corner cube.}{fig:cornerCubeImageSourcePlacement:cornerCubeImageSourcePlacementFig3}{0.2}
The sources are located one in each quadrant
%
\begin{equation}\label{eqn:cornerCubeTakeII:800}
\begin{aligned}
\Bs_1 &= h \lr{ 1,1,0} \\
\Bs_2 &= h \lr{ -1,1,0} \\
\Bs_3 &= h \lr{ -1,-1,0} \\
\Bs_4 &= h \lr{ 1,-1,0}
\end{aligned}
\end{equation}
%
and the point of measurement at \( \Br = r \rcap = r ( \sin\theta \cos\phi, \sin\theta \sin\phi, \cos\theta ) \).  If \( \Br_m = \Br - \Bs_m \) is the distance from the \( m \)th source to the observation point, then the squared distance is
%
\begin{dmath}\label{eqn:cornerCubeTakeII:840}
r_m
= \Abs{ \Br - \Bs_m }
= \lr{ r^2 + \Bs_m^2 - 2 \Br \cdot \Bs_m }^{1/2}
= r \lr{ 1 + \frac{\Bs_m^2}{r^2} - 2 \frac{\rcap}{r} \cdot \Bs_m }^{1/2}
\approx r \lr{ 1 + \inv{2} \frac{\Bs_m^2}{r^2} - \frac{\rcap}{r} \cdot \Bs_m }
=
r + \inv{2} \frac{\Bs_m^2}{r} - \rcap \cdot \Bs_m
\approx
r - \rcap \cdot \Bs_m.
\end{dmath}
%
Those distances are
\begin{subequations}
\label{eqn:cornerCubeTakeII:860}
\begin{equation}\label{eqn:cornerCubeTakeII:880}
\rcap \cdot \Bs_1 = h \sin\theta \lr{ \cos\phi+ \sin\phi } = \sqrt{2} h \sin\theta \cos\lr{ \phi - \pi/4 }
\end{equation}
\begin{equation}\label{eqn:cornerCubeTakeII:900}
\rcap \cdot \Bs_2 = h \sin\theta \lr{ -\cos\phi+ \sin\phi } = -\sqrt{2} h \sin\theta \cos\lr{ \phi + \pi/4 }
\end{equation}
\begin{equation}\label{eqn:cornerCubeTakeII:920}
\rcap \cdot \Bs_3 = -h \sin\theta \lr{ \cos\phi+ \sin\phi } = -\sqrt{2} h \sin\theta \cos\lr{ \phi - \pi/4 }
\end{equation}
\begin{equation}\label{eqn:cornerCubeTakeII:940}
\rcap \cdot \Bs_4 = h \sin\theta \lr{ \cos\phi- \sin\phi } = \sqrt{2} h \sin\theta \cos\lr{ \phi + \pi/4 }
\end{equation}
\end{subequations}

Suppose the magnetic vector potential has the structure of an infinitesimal dipole
%
\begin{equation}\label{eqn:cornerCubeTakeII:20}
\BA_m = \frac{\mu_0 I_0 }{4 \pi r_m} e^{-j k r_m} \zcap.
\end{equation}
%
In the far field, the direction vectors for all the fields will be approximately
%
\begin{dmath}\label{eqn:cornerCubeTakeII:60}
\zcap - \lr{ \zcap \cdot \rcap} \rcap
=
\cos\theta \rcap - \sin\theta \thetacap - \cos\theta \rcap
=
- \sin\theta \thetacap.
\end{dmath}
%
The far field electric field for each image source is approximately
%
\begin{dmath}\label{eqn:cornerCubeTakeII:80}
\BE_m
= -j \omega \BA_T
=  j \omega \frac{\mu_0 I_0 }{4 \pi r} e^{-j k r_m} \sin\theta \thetacap
=  j \eta k \frac{ I_0 }{4 \pi r} e^{-j k r_m} \sin\theta \thetacap.
\end{dmath}
%
Writing \( s = \sqrt{2} h \) for the distance from the origin to each of the image sources, the
superposition of all the image sources is
%
\begin{dmath}\label{eqn:cornerCubeTakeII:480}
\BE
=  j \eta k \frac{ I_0 }{4 \pi r} e^{-j k r} \sin\theta
\lr{
e^{j k s \sin\theta \cos\lr{ \phi - \pi/4}}
-e^{-j k s \sin\theta \cos\lr{ \phi + \pi/4}}
+e^{-j k s \sin\theta \cos\lr{ \phi - \pi/4}}
-e^{j k s \sin\theta \cos\lr{ \phi + \pi/4}}
 } \thetacap,
\end{dmath}
%
or
%
%\begin{equation}\label{eqn:cornerCubeTakeII:960}
\boxedEquation{eqn:cornerCubeTakeII:960}{
\BE
=  2 j \eta k \frac{ I_0 }{4 \pi r} e^{-j k r} \sin\theta
\lr{
\cos\lr{ k s \sin\theta \cos\lr{ \phi - \pi/4}}
-\cos\lr{ k s \sin\theta \cos\lr{ \phi + \pi/4}}
}.
}
%\end{equation}
%
The array factor can be picked off by inspection
%
%\begin{dmath}\label{eqn:cornerCubeTakeII:500}
\boxedEquation{eqn:cornerCubeTakeII:500}{
\textrm{AF}
= 2 I_0
\lr{
\cos\lr{ k s \sin\theta \cos\lr{ \phi - \pi/4}}
-\cos\lr{ k s \sin\theta \cos\lr{ \phi + \pi/4}}
}.
}
%\end{dmath}
%
\makeSubAnswer{}{advancedantenna:problemSet3TakeII:3a}
The radiation intensity is
%
\begin{equation}\label{eqn:cornerCubeTakeII:540}
U
= \inv{2} \eta \lr{ \frac{k I_0  }{4 \pi} }^2 \sin^2 \theta \Abs{\textrm{AF}}^2
= B_0 \sin^2 \theta
\Abs{\textrm{AF}}^2.
\end{equation}
%
This holds for both isolated antenna with \( \textrm{AF} = 1 \), and the corner cube with \( \Abs{\textrm{AF}}^2 \) given by \cref{eqn:cornerCubeTakeII:500}.

For the isolated antenna, the radiation intensity is maximized at \( \theta = \pi/2 \), so
%
\begin{dmath}\label{eqn:cornerCubeTakeII:560}
D_{0,\textrm{iso}}
= \frac{4 \pi \times 1}{2 \pi \int_0^\pi \sin^3\theta d\theta}
= \frac{2}{4/3}
= \frac{3}{2}.
\end{dmath}
%
For the corner cube the maximization problem is trickier.  As a first approximation, if \( k s \) is assumed to be small, then all the cosines in \cref{eqn:cornerCubeTakeII:500} are close to unity, and the array factor is zero.  The next order in \( k s \) expansion of the cosines is required
%
\begin{dmath}\label{eqn:cornerCubeTakeII:980}
\textrm{AF}
= 2 I_0 \lr{
1 - \frac{(k s \sin\theta)^2}{2} \cos^2( \phi - \pi/4 )
-1 + \frac{(k s \sin\theta)^2}{2} \cos^2( \phi + \pi/4 )
}
=
2 I_0 \frac{(k s \sin\theta)^2}{2} \lr{ -\cos^2( \phi - \pi/4 ) + \cos^2( \phi + \pi/4 ) }
=
- I_0(k s \sin\theta)^2 \sin( 2 \phi ),
\end{dmath}
%
so for small \( k s \)
%
\begin{dmath}\label{eqn:cornerCubeTakeII:1000}
U = B_0
(k s)^4 \sin^6 \theta \sin^2 (2 \phi).
\end{dmath}
%
The radiation intensity is clearly maximized at \( \phi = \pi/4, \theta = \pi/2 \), so
%
\begin{dmath}\label{eqn:cornerCubeTakeII:1020}
U_0 = B_0 (k s)^4
\end{dmath}
%
The radiated power, again for small \( k s \), is
%
\begin{dmath}\label{eqn:cornerCubeTakeII:600}
P_{\textrm{rad}}
=
B_0
\int_0^{\pi/2} d\phi
\int_0^{\pi} d\theta \sin \theta U(\theta, \phi)
\approx
B_0
(k s)^4
\int_0^{\pi/2} d\phi \sin^2(2 \phi)
\int_0^{\pi} d\theta \sin^7 \theta
=
B_0 (k s)^4
\lr{ \frac{\pi}{4} } \times
\lr{ \frac{32}{35} }
= B_0 (k s)^4 4 \pi \frac{2}{35}.
\end{dmath}
%
The approximate directivity of the corner cube is
%
\begin{equation}\label{eqn:cornerCubeTakeII:700}
D_{0,\textrm{ccube}} \approx \frac{4 \pi \times B_0 (ks)^4}{B_0 (ks)^4 4 \pi \frac{2}{35}} = \frac{35}{2} \approx 17.5.
\end{equation}
%
almost 12 times greater than the directivity of the isolated radiator.

The posted solution omits the \( \sin^2 \theta \) contribution from the element factor.  That results in
%
\begin{dmath}\label{eqn:cornerCubeTakeII:601}
P_{\textrm{rad}}
\approx
B_0
(k s)^4
\int_0^{\pi/2} d\phi \sin^2(2 \phi)
\int_0^{\pi} d\theta \sin^5 \theta
=
B_0 (k s)^4
\lr{ \frac{\pi}{4} } \times
\lr{ \frac{16}{15} }
= B_0 (k s)^4 4 \pi \frac{1}{15},
\end{dmath}
%
and
%
\begin{equation}\label{eqn:cornerCubeTakeII:701}
D_{0,\textrm{ccube}} \approx \frac{4 \pi \times B_0 (ks)^4}{B_0 (ks)^4 4 \pi \frac{1}{15}} = 15.
\end{equation}
%
Note that the baseline directivity without the \( \sin^2\theta \)  element factor is unity, since
%
\begin{equation}\label{eqn:cornerCubeTakeII:702}
4 \pi /2 \pi \int_0^\pi \sin\theta d\theta = 1,
\end{equation}
%
so such a relative approximation is still correct for the dipole up to an order of magnitude.
\makeSubAnswer{}{advancedantenna:problemSet3TakeII:3b}
The radiation resistance was defined implicitly by the relation
%
\begin{equation}\label{eqn:cornerCubeTakeII:720}
P_{\textrm{rad}} = \inv{2} \Abs{I_0}^2 R_\txtr,
\end{equation}
%
so the ratio of radiation resistance will just be the ratio of the radiated powers
%
\begin{equation}\label{eqn:cornerCubeTakeII:740}
\frac{R_{\txtr,\textrm{ccube}}}{R_{\txtr,\textrm{iso}}}
=
\frac{P_{\textrm{rad},\textrm{ccube}}}{P_{\textrm{rad},\textrm{iso}}}
=
\frac{8 \pi (ks)^4/35}{8 \pi/3}
=
\frac{ 3 (k s)^4}{35}.
\end{equation}
%
\makeSubAnswer{}{advancedantenna:problemSet3TakeII:3d}
The x-y plane is found at \( \theta = \pi/2 \) where the array factor is
%
\begin{dmath}\label{eqn:cornerCubeTakeII:501}
\textrm{AF}
= 2
\lr{
\cos\lr{ 2 \pi \frac{s}{\lambda} \cos\lr{ \phi - \pi/4}}
-\cos\lr{ 2 \pi \frac{s}{\lambda} \cos\lr{ \phi + \pi/4}}
}.
\end{dmath}
%
This is plotted against both \( \alpha = s/\lambda = \sqrt{2} h/\lambda\), and \( \phi \) in \cref{fig:arrayFactorXY:arrayFactorXYCorrectedFig4}, which shows that there are generally four lobes for any value of \( s \), except for the smallest values where the pattern is near zero.
\mathImageFigure{../figures/ece1229-antenna/arrayFactorXYCorrectedFig4}{Plot of \( \Abs{\textrm{AF}}^2 \) in XY plane with \( \alpha = h/\lambda \).}{fig:arrayFactorXY:arrayFactorXYCorrectedFig4}{0.3}{ps3:ps3Q3plotsCorrected.nb}

This is also plotted in
\cref{fig:arrayFactorXY:arrayFactorXYCorrectedFig5} for a few selected values of \( \alpha \).
%, and in a log scale \cref{fig:ps3p3LogScale25:ps3p3LogScale25Fig1} for these same \( \alpha \) values, plus one additional.
\mathImageTwoFigures{../figures/ece1229-antenna/ps3p3LinearFig1}{../figures/ece1229-antenna/ps3p3LogScale25Fig1}{Polar plot of \( \textrm{AF} \) in XY plane for various values of \( \alpha = s/\lambda \).}{fig:arrayFactorXY:arrayFactorXYCorrectedFig5}{scale=0.2}{ps3:p3.jl}
%\mathImageFigure{../figures/ece1229-antenna/arrayFactorXYCorrectedFig5}{Polar plot of \( \textrm{AF} \) in XY plane for various values of \( \alpha = s/\lambda \).}{fig:arrayFactorXY:arrayFactorXYCorrectedFig5}{0.3}{ps3:ps3Q3plotsCorrected.nb}
%\mathImageFigure{../figures/ece1229-antenna/ps3p3LogScale25Fig1}{Log polar plot of \( \textrm{AF} \) in XY plane for various values of \( \alpha = s/\lambda \).}{fig:ps3p3LogScale25:ps3p3LogScale25Fig1}{0.3}{ps3:p3.jl}
To plot the squared array factor, the physically significant range \( \phi \in [0, \pi/2] \) can be used, because all of the negative sign contributions from quadrant III will be flipped into quadrant I.
\mathImageFigure{../figures/ece1229-antenna/arrayFactorXYSqCorrectedFig5}{Polar plot of \( \Abs{\textrm{AF}}^2 \) for \( \theta = 0.\)}{fig:arrayFactorXYSqCorrected:arrayFactorXYSqCorrectedFig5}{0.3}{ps3:ps3Q3plotsCorrected.nb}

It's more fun to visualize this in 3D as in, and a manipulate control for visualizing \( \Abs{\textrm{AF}}^2 \) is available at
%\href{http://goo.gl/IVaiw2}{http://goo.gl/IVaiw2}
\nbref{cornerCubeArrayFactorSq.cdf}
.   This is plotted in \cref{fig:arrayFactorXYSqIn3D:arrayFactorXYSqIn3DFig6}
for \( \alpha = 0.69 \).
\mathImageFigure{../figures/ece1229-antenna/arrayFactorXYSqIn3DFig6}{Spherical plot of \( \Abs{\textrm{AF}}^2 \) for \( \alpha = 0.69\).}{fig:arrayFactorXYSqIn3D:arrayFactorXYSqIn3DFig6}{0.2}{ps3:ps3Q3plotsCorrected.nb}
\makeSubAnswer{}{advancedantenna:problemSet3TakeII:3e}
The code for the numerical calculations can be found in \nbref{ps3:ps3Q3plotsCorrected.nb}.  The results are
%The listing of \cref{mat:cornerCubeTakeII:20} shows the code used for the numerical calculation.  The results are
%
\begin{equation}\label{eqn:cornerCubeTakeII:820}
\begin{aligned}
D_0[h = \lambda/8] &= 17.1 \\
D_0[h = \lambda/4] &= 15.9 \\
D_0[h = \lambda/2] &= 14.3.
\end{aligned}
\end{equation}
%
If the \( \sin^2\theta \) contribution of the element factor is omitted (as the posted solution does in the directivity approximation), the directivities are all slightly less
%
\begin{equation}\label{eqn:cornerCubeTakeII:821}
\begin{aligned}
D_0[h = \lambda/8] &= 14.6 \\
D_0[h = \lambda/4] &= 13.2 \\
D_0[h = \lambda/2] &= 15.1.
\end{aligned}
\end{equation}
}
%\section{Mathematica Sources}
%
%The Mathematica code associated with these notes is available under
%\href{https://github.com/peeterjoot/mathematica/tree/master/ece1229/ps3}{ece1229/ps3/}
%within the github repository:
%
%git@github.com:peeterjoot/mathematica.git
%
%The notebooks for this problem set are
%
%\input{ps3mathematica.tex}
%
%%The data and figures referenced in these notes were generated with versions not greater than:
%%
%%FIXME:
%%\begin{itemize}
%%\item commit 5641358d1f397dab8a4053f7fb9681b0d532cad6
%%\end{itemize}
%
%\EndNoBibArticle
