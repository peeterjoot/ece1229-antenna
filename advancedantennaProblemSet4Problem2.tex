%
% Copyright � 2015 Peeter Joot.  All Rights Reserved.
% Licenced as described in the file LICENSE under the root directory of this GIT repository.
%
\makeoproblem{Schelkunoff z-axis, zero phase shifts.}{advancedantenna:problemSet4:2}{2015 ps4, p2}{
\index{Schelkunoff!z-axis}
\index{zero phase}
%
Use the Schelkunoff method to design a linear array of isotropic elements placed along
the z-axis such that the zeros of the array factor are located at \( \theta = \ang{0}, \ang{60}, \ang{120} \).
The
inter-element spacing is \( d = \lambda/2 \) and the progressive phase shift is zero degrees.
%
\makesubproblem{}{advancedantenna:problemSet4:2a}
What is the required number of the elements?
\makesubproblem{}{advancedantenna:problemSet4:2b}
Determine the corresponding current excitation coefficients
\makesubproblem{}{advancedantenna:problemSet4:2c}
Find the array factor
\makesubproblem{}{advancedantenna:problemSet4:2d}
Plot the corresponding array factor
} % makeoproblem
%
\makeanswer{advancedantenna:problemSet4:2}{
\makeSubAnswer{}{advancedantenna:problemSet4:2a}
%
With \( d = \lambda/2 \), we write \( z = e^{j \pi \cos\theta} \).  The zeros of the array factor occur at
%
\begin{equation}\label{eqn:advancedantennaProblemSet4Problem2:120}
\begin{aligned}
\pi \cos 0 &= \pi \\
\pi \cos( \pi/3) &= \pi/2 \\
\pi \cos( 2\pi/3) &= -\pi/2,
\end{aligned}
\end{equation}
%
so the array factor is
%
\begin{dmath}\label{eqn:advancedantennaProblemSet4Problem2:140}
\textrm{AF}
=
\lr{ z - e^{j \pi} }
\lr{ z - e^{j \pi/2} }
\lr{ z - e^{-j \pi/2} }
=
\lr{ z + 1 }
\lr{ z - j }
\lr{ z + j }
=
\lr{ z + 1 }
\lr{ z^2 - j^2 }
= z^2 + 1 + z^3 + z.
\end{dmath}
%
Normalized this is
%
\begin{dmath}\label{eqn:advancedantennaProblemSet4Problem2:40}
%\boxedEquation{eqn:advancedantennaProblemSet4Problem2:60}{
\textrm{AF}(z) = \inv{4} \lr{ 1 + z + z^2 + z^3}.
%}
\end{dmath}
%
Four elements are required.
%
\makeSubAnswer{}{advancedantenna:problemSet4:2b}
%
The currents at positions \( \Br_m = m d \zcap, m \in \setlr{0,1,2,3} \) are
%
\begin{equation}\label{eqn:advancedantennaProblemSet4Problem2:80}
\begin{aligned}
I_0 &= \inv{4} \\
I_1 &= \inv{4} \\
I_2 &= \inv{4} \\
I_3 &= \inv{4}.
\end{aligned}
\end{equation}
%
\makeSubAnswer{}{advancedantenna:problemSet4:2c}
%
A phase term may be factored out of the array factor to put it in real form
%
\begin{dmath}\label{eqn:advancedantennaProblemSet4Problem2:100}
\textrm{AF}
=
\frac{z^{3/2}}{4} \lr{ z^{-3/2} + z^{-1/2} + z^{1/2} + z^{3/2} }.
\end{dmath}
%
Substituting \( z = e^{j \pi \cos\theta} \), and discarding the leading \( z^{3/2} \) term, the array factor is
%
\begin{dmath}\label{eqn:advancedantennaProblemSet4Problem2:160}
%\boxedEquation{eqn:advancedantennaProblemSet4Problem2:180}{
\textrm{AF} = \inv{2} \lr{
\cos\lr{ \frac{\pi}{2} \cos\theta }
+
\cos\lr{ \frac{3 \pi}{2} \cos\theta }
}
= \inv{4} \frac{\sin\lr{ 2 \pi \cos\theta }}{\sin\lr{ \frac{\pi}{2} \cos\theta }}
.
%}
\end{dmath}
%
\makeSubAnswer{}{advancedantenna:problemSet4:2d}
%
This is plotted in \cref{fig:ps4p2Plot:ps4p2PlotFig1}, which also clearly shows the zeros at \( \theta = 0, \ang{60}, \ang{120} \) as desired.

%\cref{fig:ps4p2Linear:ps4p2LinearFig1}.
%\imageFigure{../figures/ece1229-antenna/ps4p2LinearFig1}{CAPTION: ps4p2LinearFig1}{fig:ps4p2Linear:ps4p2LinearFig1}{0.3}
%\cref{fig:ps4p2LogScale30:ps4p2LogScale30Fig1}.
%\imageFigure{../figures/ece1229-antenna/ps4p2LogScale30Fig1}{CAPTION: ps4p2LogScale30Fig1}{fig:ps4p2LogScale30:ps4p2LogScale30Fig1}{0.3}
%
\mathImageTwoFigures{../figures/ece1229-antenna/ps4p2LinearFig1}{../figures/ece1229-antenna/ps4p2LogScale30Fig1}{Array factor for specified zeros.}{fig:ps4p2Plot:ps4p2PlotFig1}{scale=0.2}{ps4:p2x.jl}
%\imageFigure{../figures/ece1229-antenna/ps4p2PlotFig1}{Array factor for specified zeros.}{fig:ps4p2Plot:ps4p2PlotFig1}{0.3}
}
