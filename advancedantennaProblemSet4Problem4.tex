%
% Copyright � 2015 Peeter Joot.  All Rights Reserved.
% Licenced as described in the file LICENSE under the root directory of this GIT repository.
%
\makeoproblem{Dolph-Chebyshev.}{advancedantenna:problemSet4:4}{2015 problem set 4, p4}{
\index{Dolph-Chebyshev}

Design a five-element, \( -40 \si{dB} \) sidelobe level Dolph-Chebyshev array of isotropic elements.
The elements are placed along the x-axis with an inter-element spacing \( d = \lambda /2 \).
Determine,

\makesubproblem{}{advancedantenna:problemSet4:4a}
the normalized amplitude coefficients
\makesubproblem{}{advancedantenna:problemSet4:4b}
the array factor
\makesubproblem{}{advancedantenna:problemSet4:4c}
Use numerical integration to calculate the directivity
\makesubproblem{}{advancedantenna:problemSet4:4d}
and the null-to-null beamwidth
\makesubproblem{}{advancedantenna:problemSet4:4e}
Repeat \partref{advancedantenna:problemSet4:4a}-\ref{advancedantenna:problemSet4:4c}
 for a uniform broadside array of the same spacing
\makesubproblem{}{advancedantenna:problemSet4:4f}
Plot the power array-factor patterns for the two arrays on the same plot.
} % makeoproblem

\makeanswer{advancedantenna:problemSet4:4}{
\makeSubAnswer{}{advancedantenna:problemSet4:4a}

The \( 40 \si{dB} \) level is equivalent to
%
\begin{dmath}\label{eqn:advancedantennaProblemSet4Problem4:20}
20 \log_{10} R = 40,
\end{dmath}
%
or
%
\begin{equation}\label{eqn:advancedantennaProblemSet4Problem4:40}
R = 10^{2} = 100.
\end{equation}
%
The Chebyshev scaling factor for a five element array is
%
\begin{equation}\label{eqn:advancedantennaProblemSet4Problem4:60}
x_0 = \cosh\lr{ \inv{4} \cosh^{-1} R } = 2.01.
\end{equation}
%
With \( x = x_0 \cos(u/2) \), the unnormalized array factor is
%
\begin{dmath}\label{eqn:advancedantennaProblemSet4Problem4:160}
\textrm{AF}(u)
= T_4( x )
= T_4( x_0 \cos(u/2) )
=
8 x_0^4 \cos^4(u/2) - 8 x_0^2 \cos^2(u/2) + 1.
\end{dmath}
%
Since
%
\begin{equation}\label{eqn:advancedantennaProblemSet4Problem4:80}
\begin{aligned}
\cos^2(u/2) &= \inv{2} \lr{ \cos(u) + 1 } \\
\cos^4(u/2) &= \inv{8} \lr{ \cos(2 u) + 4 \cos(u) + 3 },
\end{aligned}
\end{equation}
%
the array factor can be expanded in \( \cos( m u ) \), as
%
\begin{dmath}\label{eqn:advancedantennaProblemSet4Problem4:100}
\textrm{AF}(u)
=
x_0^4 \lr{ \cos(2 u) + 4 \cos(u) + 3 }
- 4 x_0^2 \lr{ \cos(u) + 1 }
+ 1
=
x_0^4 \cos(2 u)
+ \lr{ 4 x_0^4 - 4 x_0^2 } \cos(u)
+ 3 x_0^4 - 4 x_0^2 + 1.
\end{dmath}
%
After normalization this is
%
\begin{equation}\label{eqn:advancedantennaProblemSet4Problem4:120}
\begin{aligned}
\textrm{AF}(u) &= \alpha \cos( 2 u ) + \beta \cos(u ) + \gamma \\
\alpha &= \frac{x_0^4}{8 x_0^4 - 8 x_0^2 + 1} \\
\beta &= \frac{ 4 x_0^4 - 4 x_0^2 }{8 x_0^4 - 8 x_0^2 + 1} \\
\gamma &= \frac{ 3 x_0^4 - 4 x_0^2 + 1 }{8 x_0^4 - 8 x_0^2 + 1}
\end{aligned}
\end{equation}
%
The array coefficients are found to have the values
%
\begin{equation}\label{eqn:advancedantennaProblemSet4Problem4:140}
\begin{aligned}
I_{-2} &= \frac{\alpha}{2} = 0.082 \\
I_{-1} &= \frac{\beta}{2} = 0.25 \\
I_{0} &= \gamma  = 0.34 \\
I_{1} &= \frac{\beta}{2} = 0.25 \\
I_{2} &= \frac{\alpha}{2} = 0.082.
\end{aligned}
\end{equation}
%
\makeSubAnswer{}{advancedantenna:problemSet4:4b}

The array factor is defined by \cref{eqn:advancedantennaProblemSet4Problem4:120}, \cref{eqn:advancedantennaProblemSet4Problem4:140}, where \( u = \pi \sin\theta \cos\phi \).

\makeSubAnswer{}{advancedantenna:problemSet4:4c}

The directivity is found to be 3.97 (5.98 \si{dB} ).

\makeSubAnswer{}{advancedantenna:problemSet4:4d}

The zeros of the array factor occur where the argument of
%
\begin{dmath}\label{eqn:advancedantennaProblemSet4Problem4:300}
T_m(x_0 \cos(u/2)) = \cos( m \cos^{-1}\lr{ x_0 \cos(u/2) } ),
\end{dmath}
%
equals \( -\pi/2 + n \pi \), or
%
\begin{dmath}\label{eqn:advancedantennaProblemSet4Problem4:280}
u = 2 \cos^{-1} \lr{ \inv{x_0} \cos \lr{ \frac{\pi}{2 m}( 2 n -1 )} }.
\end{dmath}
%
Compare this to the zeros of the uniform array factor, which was
%
\begin{equation}\label{eqn:advancedantennaProblemSet4Problem4:180}
\textrm{AF}(z)
= \sum_{n = 0}^{N-1} z^n
= \frac{1 - z^N}{1 - z}
= z^{(N-1)/2} \frac{ z^{N/2} - z^{-N/2}}{z^{1/2} - z^{-1/2}}
\end{equation}
%
so with \( z = e^{j u} \), the absolute array factor is
\begin{dmath}\label{eqn:advancedantennaProblemSet4Problem4:200}
\Abs{\textrm{AF}(u)}
=
\inv{N}
\frac
{ \Abs{\sin\lr{N u/2}} }
{ \Abs{\sin\lr{u/2}} }.
\end{dmath}
%
This has zeros where
%
\begin{equation}\label{eqn:advancedantennaProblemSet4Problem4:220}
u = \frac{ 2 n \pi}{N}, \qquad n \ne 0 \in \mathbb{Z}.
\end{equation}
%
These two sets of zeros are plotted on the unit circle in the z-domain in \cref{fig:chebychevAndLinearZdomainZeros:chebychevAndLinearZdomainZerosFig4}.
\mathImageFigure{../figures/ece1229-antenna/chebychevAndLinearZdomainZerosFig4}{Zeros of five element Chebyshev and uniform array elements on z-domain unit circle.}{fig:chebychevAndLinearZdomainZeros:chebychevAndLinearZdomainZerosFig4}{0.3}{ps4:ps4p4Chebychev.nb}

The Chebyshev and uniform array factors are plotted the z-x plane for \( u = k d \sin\theta \cos(0) \) in \si{dB} in \cref{fig:chebychevDbPlotZXplane:chebychevDbPlotZXplaneFig5}.
% and \cref{fig:uniformDbPlotZXplane:uniformDbPlotZXplaneFig6} respectively.
\mathImageTwoFigures{../figures/ece1229-antenna/chebychevDbPlotZXplaneFig5}{../figures/ece1229-antenna/uniformDbPlotZXplaneFig6}{Chebyshev and uniform power array factor in z-x plane (dB).}{fig:chebychevDbPlotZXplane:chebychevDbPlotZXplaneFig5}{scale=0.5}{ps4:ps4p4Chebychev.nb}
%\imageFigure{../figures/ece1229-antenna/chebychevDbPlotZXplaneFig5}{Chebyshev power array factor in z-x plane (dB)}{fig:chebychevDbPlotZXplane:chebychevDbPlotZXplaneFig5}{0.3}
%\imageFigure{../figures/ece1229-antenna/uniformDbPlotZXplaneFig6}{Uniform power array factor in z-x plane (dB)}{fig:uniformDbPlotZXplane:uniformDbPlotZXplaneFig6}{0.3}
For the Chebyshev array the zeros are found to be at \( \setlr{\ang{44}, \ang{61}, \ang{119}, \ang{136}} \), so the null to null beamwidth is \ang{88}.  The 3 \si{dB} beamwidth for the main lobe is found to be \ang{28}.

For the uniform array the zeros are found at \( \setlr{ \ang{24}, \ang{53}, \ang{127}, \ang{156} } \) so that arrays' null to null beamwidth is \ang{48}.
\makeSubAnswer{}{advancedantenna:problemSet4:4e}
The normalized uniform array amplitude coefficients are \( 1/5 \).  The array factor is given by \cref{eqn:advancedantennaProblemSet4Problem4:200}.  The directivity for the linear array is found numerically to be 5.0 (6.99 \si{dB} ).
\makeSubAnswer{}{advancedantenna:problemSet4:4f}
This array configuration has a donut shaped power pattern, as shown in \cref{fig:chebychevBroadside3D:chebychevBroadside3DFig3}.
\mathImageFigure{../figures/ece1229-antenna/chebychevBroadside3DFig3}{5 element Chebyshev array power pattern in 3D.}{fig:chebychevBroadside3D:chebychevBroadside3DFig3}{0.3}{ps4:ps4p4Chebychev.nb}
The two array factor power patterns (normalized) are plotted in \cref{fig:testmultLinear:testmultLinearFig1}.
%\cref{fig:originZoomOfChebychevVsLinearArray:originZoomOfChebychevVsLinearArrayFig1}.
\mathImageTwoFigures{../figures/ece1229-antenna/testmultLinearFig1}{../figures/ece1229-antenna/testmultLogScale50Fig1}{Plots of 5 element Chebyshev and uniform array power patterns for \( u = k d \sin\theta \cos 0 \).}{fig:testmultLinear:testmultLinearFig1}{scale=0.2}{ps4:p4.jl}
%\imageFigure{../figures/ece1229-antenna/chebychevVsLinearArrayBothBroadsideFig2}{Plots of 5 element Chebyshev and uniform array power patterns for \( u = k d \sin\theta \cos 0 \).}{fig:chebychevVsLinearArrayBothBroadside:chebychevVsLinearArrayBothBroadsideFig2}{0.3}
%In both power patterns the side lobes are quite minimal.  These side lobes are not visible in the Chebyshev power pattern, and can be seen in the linear uniform array power pattern by zooming into the origin, as shown in \cref{fig:chebychevVsLinearArrayBothBroadside:chebychevVsLinearArrayBothBroadsideFig2}.
%
%\imageFigure{../figures/ece1229-antenna/originZoomOfChebychevVsLinearArrayFig1}{Zoom into the origin for the 5 element array power patterns}{fig:originZoomOfChebychevVsLinearArray:originZoomOfChebychevVsLinearArrayFig1}{0.3}
}
