%
% Copyright � 2015 Peeter Joot.  All Rights Reserved.
% Licenced as described in the file LICENSE under the root directory of this GIT repository.
%
\makeoproblem{Long thin wire dipoles.}{advancedantenna:problemSet3:2}{2015 ps3, p2}{
\index{dipole!long}
%
\makesubproblem{}{advancedantenna:problemSet3:2a}
On a single diagram, plot the polar patterns for
\( l = 0.5 \lambda, l = 1.0 \lambda, l = 1.25 \lambda \) and \( l = 2.0 \lambda \) long thin wire dipole antennas.
%
\makesubproblem{}{advancedantenna:problemSet3:2b}
Use numerical integration to calculate the maximum directivity for each dipole.
Make a table with your results. Which length corresponds to the highest
directivity?
%
\makesubproblem{}{advancedantenna:problemSet3:2c}
Use numerical integration to calculate the radiation resisistance of the  \( l = 1.25 \lambda \) dipole. Do you expect this dipole to be capacitive or inductive?
} % makeoproblem
%
\makeanswer{advancedantenna:problemSet3:2}{
\makeSubAnswer{}{advancedantenna:problemSet3:2a}
%
Assuming a \( \zcap \) oriented dipole, in the far field, the electric field is
%
\begin{dmath}\label{eqn:advancedantennaProblemSet3Problem2:20}
E_\theta \approx j \eta \frac{I_0 e^{-j k r}}{ 2 \pi r }
\lr{
\frac{\cos\lr{ \frac{k l}{2} \cos\theta} - \cos\lr{\frac{kl}{2}}}{\sin\theta}
}.
\end{dmath}
%
%an average Poynting vector of
%
%\begin{dmath}\label{eqn:advancedantennaProblemSet3Problem2:40}
%\BW_{\textrm{av}} =
%\rcap
%\eta \frac{\Abs{I_0}^2}{8 \pi^2 r^2}
%\lr{
%\frac{\cos\lr{ \frac{k l}{2} \cos\theta} - \cos\lr{\frac{kl}{2}}}{\sin\theta}
%}^2.
%\end{dmath}
%
Writing
\( l = \alpha \lambda \), and noting that
the magnetic field is \( H_\phi \approx E_\theta/\eta \),
the radiation intensity \( U = r^2 W_{\textrm{av}} \) is
%
\begin{dmath}\label{eqn:advancedantennaProblemSet3Problem2:60}
U =
\eta \frac{\Abs{I_0}^2}{8 \pi^2}
\lr{
\frac{\cos\lr{ \pi \alpha \cos\theta} - \cos\lr{ \pi \alpha }}{\sin\theta}
}^2.
\end{dmath}
%
In \cref{fig:longDipole:longDipoleFig1} \( F(\theta) = 8 \pi^2 U /\eta \Abs{I_0}^2 \) is plotted for \( \alpha \in \setlr{0.5, 1, 1.25, 2.0 } \).  For \( \alpha = 1.25 \) some very small side lobes are just barely visible.  For \( \alpha = 2 \) the single lobe directivity is lost, and a significant split of the radiation field along two different directions can be observed.  These individual features can be explored more easily in
%\href{http://goo.gl/OjK4oc}{http://goo.gl/OjK4oc}
\nbref{longDipolesWithLengthControl.cdf}
which provides a Manipulate based interactive control for varying the \( l/\lambda \) ratio.
\mathImageFigure{../figures/ece1229-antenna/longDipoleFig1}{Polar plot of radiation intensities for some electric z-axis oriented dipoles.}{fig:longDipole:longDipoleFig1}{0.3}{ps3:longDipolesSelectedLengthsSavedLabeledPlot.nb}
It is much more satisfactory to view these in a three dimensional plot as in \nbref{ps3:longDipoleInteractiveLength.nb}, and \cref{fig:longDipoleLequals2Lambda:longDipoleLequals2LambdaFig1}, but such a visualization does not work well for overlaid intensity patterns.
\mathImageFigure{../figures/ece1229-antenna/longDipoleLequals2LambdaFig1}{Double wavelength radiation intensity.}{fig:longDipoleLequals2Lambda:longDipoleLequals2LambdaFig1}{0.3}{ps3:longDipoleInteractiveLength.nb}
The side lobes for the \( \alpha = 1.25 \) case do not show up very well in the plot above.  The log polar plot of \cref{fig:ps3p2:ps3p2Fig1} shows this detail better.
\mathImageFigure{../figures/ece1229-antenna/ps3p2Fig1}{Log polar plot of radiation intensities for some electric z-axis oriented dipoles.}{fig:ps3p2:ps3p2Fig1}{0.4}{ps3:p2.jl}
\makeSubAnswer{}{advancedantenna:problemSet3:2b}
The directivity is given by
%
\begin{dmath}\label{eqn:advancedantennaProblemSet3Problem2:80}
D_0 = \frac{4 \pi \evalbar{F(\theta)}{\textrm{max}}}{2 \pi \int_0^\pi F(\theta) \sin \theta d\theta}.
\end{dmath}
%
These values, calculated in \nbref{ps3:directivityLongDipole.nb} using the Mathematica functions NMaximize and NIntegrate, are
\captionedTable{Directivities.}{tab:advancedantennaProblemSet3Problem2:10}{
\begin{tabular}{|l||l|l|l|l|}
\hline
\(\alpha\) & 0.5 & 1 & 1.25 & 2 \\
\hline
\(D_0\) & 1.64092 & 2.411 & 3.28248 & 2.52856 \\
\hline
\end{tabular}
}
The largest directivity for these specific values of \( l \) is found at \( l = 1.25 \lambda \).
\makeSubAnswer{}{advancedantenna:problemSet3:2c}
The radiation resistance is implicitly defined by
%
\begin{equation}\label{eqn:advancedantennaProblemSet3Problem2:100}
P_{\textrm{rad}} = \int U d\Omega = \inv{2} \Abs{I_0}^2 R_\txtr,
\end{equation}
%
or, with \( \eta = 120 \pi \Omega \),
%
\begin{dmath}\label{eqn:advancedantennaProblemSet3Problem2:120}
R_\txtr
= \frac{2}{\Abs{I_0}^2} \int U d\Omega
=
120 \pi \frac{1}{4 \pi^2}
\int_0^{2 \pi } d\phi
\int_0^\pi
\lr{
\frac{\cos\lr{ \pi \alpha \cos\theta} - \cos\lr{ \pi \alpha }}{\sin\theta}
}^2
\sin\theta d\theta
=
60
\int_0^\pi
\frac{ \lr{\cos\lr{ \pi \alpha \cos\theta} - \cos\lr{ \pi \alpha }}^2 }{\sin\theta}
d\theta.
\end{dmath}
%
This numerical integration was computed in \nbref{ps3:longDipolesSelectedLengths.nb} and for the lengths in this problem are
%
\captionedTable{Radiation resistances.}{tab:advancedantennaProblemSet3Problem2:20}{
\begin{tabular}{|l||l|l|l|l|}
\hline
\(\alpha\) & 0.5 & 1 & 1.25 & 2 \\
\hline
\(R_r\) & 73.1296 & 199.088 & 106.537 & 259.634 \\
\hline
\end{tabular}
}

The half-wavelength number calculated matches the value quoted in \citep{balanis2005antenna} \texteqnref{4-93}.

For the reactance, without calculating, I don't know an intuitive way to determine whether it would be positive or negative for any given length.  The graph of \citep{balanis2005antenna} \textfigref{8.17} appears to show that the reactance is roughly positive (inductive) in the \( [0.5,1] \lambda \) interval and negative (capacitive) in the \( [1,1.5] \lambda \) interval.
}
