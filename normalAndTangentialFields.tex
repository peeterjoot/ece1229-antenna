%
% Copyright � 2015 Peeter Joot.  All Rights Reserved.
% Licenced as described in the file LICENSE under the root directory of this GIT repository.
%
\input{../latex/blogpost.tex}
\renewcommand{\basename}{normalAndTangentialFields}
\renewcommand{\dirname}{notes/ece1229/}
%\newcommand{\dateintitle}{}
%\newcommand{\keywords}{}
%
%\input{../latex/peeter_prologue_print2.tex}
%\usepackage{peeters_figures}
%\usepackage{macros_bm}
%\usepackage{txfonts} % oint...
%
%\beginArtNoToc
%
%\generatetitle{Tangential and normal field components}
\section{Tangential and normal field components.}
%\label{chap:normalAndTangentialFields}
%
The integral forms of Maxwell's equations can be used to derive relations for the tangential and normal field components to the sources.  These relations were mentioned in class, but it is useful to go over the derivation.  This isn't all review from first year electromagnetism since we are now using a magnetic source modifications of Maxwell's equations.
%
The derivation below follows that of \citep{balanis1989advanced:maxwell} closely, but I am trying it myself to ensure that I understand the assumptions.

The two infinitesimally thin pillboxes of
\cref{fig:pillboxForTangentialFields:pillboxForTangentialFieldsFig1}
are used in the argument.
\imageTwoFigures{../figures/ece1229-antenna/pillboxForTangentialFieldsFig1}{../figures/ece1229-antenna/pillboxForNormalFieldsFig2}{Pillboxes for tangential and normal field relations}{fig:pillboxForTangentialFields:pillboxForTangentialFieldsFig1}{scale=0.2}
%\imageFigure{../figures/ece1229-antenna/pillboxForNormalFieldsFig2}{CAPTION: }{fig:pillboxForNormalFields:pillboxForNormalFieldsFig2}{0.3}
%\imageFigure{../figures/ece1229-antenna/pillboxForTangentialFieldsFig1}{CAPTION: }{fig:pillboxForTangentialFields:pillboxForTangentialFieldsFig1}{0.3}
%
\index{Maxwell's equations!differential}
Maxwell's equations with both magnetic and electric sources are
%
\begin{subequations}
\begin{equation}\label{eqn:normalAndTangentialFields:20}
\spacegrad \cross \bcE = -\PD{t}{\bcB} -\bcM
\end{equation}
\begin{equation}\label{eqn:normalAndTangentialFields:40}
\spacegrad \cross \bcH = \bcJ + \PD{t}{\bcD}
\end{equation}
\begin{equation}\label{eqn:normalAndTangentialFields:60}
\spacegrad \cdot \bcD = \rho_\txte
\end{equation}
\begin{equation}\label{eqn:normalAndTangentialFields:80}
\spacegrad \cdot \bcB = \rho_\txtm.
\end{equation}
\end{subequations}
%
\index{Stokes' theorem}
\index{divergence theorem}
After application of Stokes' and the divergence theorems Maxwell's equations have the integral form
%
\begin{subequations}
\begin{equation}\label{eqn:normalAndTangentialFields:100}
\ointctrclockwise \bcE \cdot d\Bl = -\int d\BA \cdot \lr{ \PD{t}{\bcB} + \bcM }
\end{equation}
\begin{equation}\label{eqn:normalAndTangentialFields:120}
\ointctrclockwise \bcH \cdot d\Bl = \int d\BA \cdot \lr{ \PD{t}{\bcD} + \bcJ }
\end{equation}
\begin{equation}\label{eqn:normalAndTangentialFields:140}
\int_{\partial V} \bcD \cdot d\BA
=
\int_V \rho_\txte\,dV
\end{equation}
\begin{equation}\label{eqn:normalAndTangentialFields:160}
\int_{\partial V} \bcB \cdot d\BA
=
\int_V \rho_\txtm\,dV.
\end{equation}
\end{subequations}
%
\index{Gauss's law!differential form}
\index{Gauss's law!differential form, magnetic}
\index{Maxwell-Faraday equation!differential form}
\index{Ampere's law!differential form}
\index{Gauss's law!integral form}
\index{Gauss's law!integral form, magnetic}
\index{Maxwell-Faraday equation!integral form}
\index{Ampere's law!integral form}
%
\paragraph{Maxwell-Faraday equation.}
%
\index{electric surface current density}
\index{electric field!tangential}
%
First consider one of the loop integrals, like \cref{eqn:normalAndTangentialFields:100}.  For an infinitesimal loop, that integral is
%
\begin{equation}\label{eqn:normalAndTangentialFields:180}
\ointctrclockwise \bcE \cdot d\Bl
\approx
\mathcal{E}^{(1)}_x \Delta x
+ \mathcal{E}^{(1)} \frac{\Delta y}{2}
+ \mathcal{E}^{(2)} \frac{\Delta y}{2}
-\mathcal{E}^{(2)}_x \Delta x
- \mathcal{E}^{(2)} \frac{\Delta y}{2}
- \mathcal{E}^{(1)} \frac{\Delta y}{2}
\approx
\lr{ \mathcal{E}^{(1)}_x
-\mathcal{E}^{(2)}_x } \Delta x
+ \inv{2} \PD{x}{\mathcal{E}^{(2)}} \Delta x \Delta y
+ \inv{2} \PD{x}{\mathcal{E}^{(1)}} \Delta x \Delta y.
\end{equation}
%
We let \( \Delta y \rightarrow 0 \) which kills off all but the first difference term.

The RHS of \cref{eqn:normalAndTangentialFields:180} is approximately
%
\begin{equation}\label{eqn:normalAndTangentialFields:200}
-\int d\BA \cdot \lr{ \PD{t}{\bcB} + \bcM }
\approx
- \Delta x \Delta y \lr{ \PD{t}{\mathcal{B}_z} + \mathcal{M}_z }.
\end{equation}
%
If the magnetic field contribution is assumed to be small in comparison to the magnetic current (i.e. infinite magnetic conductance), and if a linear magnetic current source of the form is also assumed
%
\begin{equation}\label{eqn:normalAndTangentialFields:220}
\bcM_s = \lim_{\Delta y \rightarrow 0} \lr{\bcM \cdot \zcap} \zcap \Delta y,
\end{equation}
%
then the Maxwell-Faraday equation takes the form
%
\begin{equation}\label{eqn:normalAndTangentialFields:240}
\lr{ \mathcal{E}^{(1)}_x
-\mathcal{E}^{(2)}_x } \Delta x
\approx
- \Delta x \bcM_s \cdot \zcap.
\end{equation}
%
While \( \bcM \) may have components that are not normal to the interface, the surface current need only have a normal component, since only that component contributes to the surface integral.

The coordinate expression of \cref{eqn:normalAndTangentialFields:240} can be written as
%
\begin{equation}\label{eqn:normalAndTangentialFields:260}
\begin{aligned}
- \bcM_s \cdot \zcap
&=
\lr{ \bcE^{(1)} -\bcE^{(2)} } \cdot \lr{ \ycap \cross \zcap }
\\ &=
\lr{ \lr{ \bcE^{(1)} -\bcE^{(2)} } \cross \ycap } \cdot \zcap.
\end{aligned}
\end{equation}
%
This is satisfied when

\boxedEquation{eqn:normalAndTangentialFields:280}{
\lr{ \bcE^{(1)} -\bcE^{(2)} } \cross \ncap = - \bcM_s,
}

where \( \ncap \) is the normal between the interfaces.  I'd failed to understand when reading this derivation initially, how the \( \bcB \) contribution was killed off.  i.e. If the vanishing area in the surface integral kills off the \( \bcB \) contribution, why do we have a \( \bcM \) contribution left.  The key to this is understanding that this magnetic current is considered to be confined very closely to the surface getting larger as \( \Delta y \) gets smaller.

Also note that the units of \( \bcM_s \) are volts/meter like the electric field (not volts/squared-meter like \( \bcM \).)
%
\paragraph{Ampere's law.}
%
\index{electric surface current density}
\index{magnetic field!tangential}
As above, assume a linear electric surface current density of the form
%
\begin{equation}\label{eqn:normalAndTangentialFields:300}
\bcJ_s = \lim_{\Delta y \rightarrow 0} \lr{\bcJ \cdot \ncap} \ncap \Delta y,
\end{equation}
%
in units of amperes/meter (not amperes/meter-squared like \( \bcJ \).)

To apply the arguments above to Ampere's law, only the sign needs to be adjusted

\boxedEquation{eqn:normalAndTangentialFields:290}{
\lr{ \bcH^{(1)} -\bcH^{(2)} } \cross \ncap = \bcJ_s.
}
%
\paragraph{Gauss's law.}
%
Using the cylindrical pillbox surface with radius \( \Delta r \), height \( \Delta y \), and top and bottom surface areas \( \Delta A = \pi \lr{\Delta r}^2 \), the LHS of Gauss's law \cref{eqn:normalAndTangentialFields:140} expands to
%
\begin{equation}\label{eqn:normalAndTangentialFields:320}
\begin{aligned}
\int_{\partial V} \bcD \cdot d\BA
&\approx
\mathcal{D}^{(2)}_y \Delta A
+ \mathcal{D}^{(2)}_\rho 2 \pi \Delta r \frac{\Delta y}{2}
+ \mathcal{D}^{(1)}_\rho 2 \pi \Delta r \frac{\Delta y}{2}
-\mathcal{D}^{(1)}_y \Delta A \\
&\approx
\lr{ \mathcal{D}^{(2)}_y
-\mathcal{D}^{(1)}_y } \Delta A.
\end{aligned}
\end{equation}
%
As with the Stokes integrals above it is assumed that the height is infinitesimal with respect to the radial dimension.  Letting that height \( \Delta y \rightarrow 0 \) kills  off the radially directed contributions of the flux through the sidewalls.

The RHS expands to approximately
%
\begin{equation}\label{eqn:normalAndTangentialFields:340}
\int_V \rho_\txte\,dV
\approx
\Delta A \Delta y \rho_\txte.
\end{equation}
%
Define a highly localized surface current density (coulombs/meter-squared) as
%
\begin{equation}\label{eqn:normalAndTangentialFields:360}
\sigma_\txte = \lim_{\Delta y \rightarrow 0} \Delta y \rho_\txte.
\end{equation}
%
Equating \cref{eqn:normalAndTangentialFields:340} with \cref{eqn:normalAndTangentialFields:320} gives
%
\begin{equation}\label{eqn:normalAndTangentialFields:380}
\lr{ \mathcal{D}^{(2)}_y
-\mathcal{D}^{(1)}_y } \Delta A
=
\Delta A \sigma_\txte,
\end{equation}
%
or

\boxedEquation{eqn:normalAndTangentialFields:400}{
\lr{ \bcD^{(2)} - \bcD^{(1)} } \cdot \ncap = \sigma_\txte.
}
%
\paragraph{Gauss's law for magnetism.}
%
The same argument can be applied to the magnetic flux.  Define a highly localized magnetic surface current density (webers/meter-squared) as
%
\begin{equation}\label{eqn:normalAndTangentialFields:440}
\sigma_\txtm = \lim_{\Delta y \rightarrow 0} \Delta y \rho_\txtm,
\end{equation}
%
yielding the boundary relation

\boxedEquation{eqn:normalAndTangentialFields:420}{
\lr{ \bcB^{(2)} - \bcB^{(1)} } \cdot \ncap = \sigma_\txtm.
}
%
%\EndArticle
