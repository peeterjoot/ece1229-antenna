%
% Copyright � 2015 Peeter Joot.  All Rights Reserved.
% Licenced as described in the file LICENSE under the root directory of this GIT repository.
%
\makeoproblem
%{Superposition of electric and magnetic dipoles.}
{Dipole superposition.}
{advancedantenna:problemSet3:1}{2015 ps3, p1}{
\index{dipole!superposition}
An infinitesimal electric dipole of electric current strength \( I_{\textrm{eo}} \) is oriented along the x-axis. With this there is also an infinitesimal magnetic dipole of magnetic current strength \( I_{\textrm{mo}} \) but oriented along the y-axis.
%
\makesubproblem{}{advancedantenna:problemSet3:1a}
Write down an expression for the total electric field radiated in the far zone.
\makesubproblem{}{advancedantenna:problemSet3:1b}
Assume now that \( I_{\textrm{eo}}/I_{\textrm{mo}} = \eta_\txto = 120 \pi \Omega \).
Simplify the electric field expression found \partref{advancedantenna:problemSet3:1a}.
%
\makesubproblem{}{advancedantenna:problemSet3:1c}
Plot in polar co-ordinates the normalized magnitude of the electric field in the zy plane and for \( 0 < \theta \le 2 \pi \) (far zone).
%
} % makeoproblem
%
\makeanswer{advancedantenna:problemSet3:1}{
\makeSubAnswer{}{advancedantenna:problemSet3:1a}
%
The far field electric field induced by the electric current can be calculated with the transverse projection
%
\begin{equation}\label{eqn:advancedantennaProblemSet3Problem1:20}
\begin{aligned}
\BE_\txte
&= - j \omega \Proj_\T \BA
\\ &= - j \omega \lr{ \BA - (\BA \cdot \kcap) \kcap },
\end{aligned}
\end{equation}
%
where
%
\begin{equation}\label{eqn:advancedantennaProblemSet3Problem1:40}
\BA = \xcap \frac{\mu_0 I_{\textrm{eo}} l}{4 \pi r} e^{-j k r}.
\end{equation}
%
To simplify this, note that the Cartesian to spherical coordinates mapping is
%
\begin{subequations}
\label{eqn:advancedantennaProblemSet3Problem1:60}
\begin{equation}\label{eqn:advancedantennaProblemSet3Problem1:80}
\xcap = \sin\theta \cos\phi \kcap + \cos\theta \cos\phi \thetacap - \sin\phi \phicap
\end{equation}
\begin{equation}\label{eqn:advancedantennaProblemSet3Problem1:100}
\ycap = \sin\theta \sin\phi \kcap + \cos\theta \sin\phi \thetacap + \cos\phi \phicap
\end{equation}
\begin{equation}\label{eqn:advancedantennaProblemSet3Problem1:120}
\zcap = \cos\theta \kcap - \sin\theta \thetacap,
\end{equation}
\end{subequations}
%
so
%
\begin{equation}\label{eqn:advancedantennaProblemSet3Problem1:140}
\BE_\txte
= - j \omega
\lr{ \cos\theta \cos\phi \thetacap - \sin\phi \phicap }
\frac{\mu_0 I_{\textrm{eo}} l}{4 \pi r} e^{-j k r}.
\end{equation}
%
For the magnetic current, first note that the far field magnetic field for an electric current can also be expressed in terms of the magnetic vector potential
%
\begin{equation}\label{eqn:advancedantennaProblemSet3Problem1:160}
\begin{aligned}
\BH
&= \inv{\eta} \kcap \cross \BE
\\ &= -j\frac{\omega}{\eta} \kcap \cross \lr{ \BA - (\BA \cdot \kcap)\kcap }
\\ &= -j\frac{\omega}{\eta} \kcap \cross \BA.
\end{aligned}
\end{equation}
%
Duality provides the far field electric field given an electric vector potential
%
\begin{equation}\label{eqn:advancedantennaProblemSet3Problem1:180}
\BE_\txtm = j \omega \eta \kcap \cross \BF.
\end{equation}
%
For the y-axis oriented magnetic current, the vector potential is
%
\begin{equation}\label{eqn:advancedantennaProblemSet3Problem1:200}
\BF = \ycap \frac{\epsilon_0 I_{\textrm{mo}} l}{4 \pi r} e^{-j k r}.
\end{equation}
%
The electric field will be directed along
%
\begin{equation}\label{eqn:advancedantennaProblemSet3Problem1:220}
\begin{aligned}
\kcap \cross \ycap
&=
\kcap \cross
\lr{
\sin\theta \sin\phi \kcap + \cos\theta \sin\phi \thetacap + \cos\phi \phicap
}
\\ &=
\cos\theta \sin\phi \phicap - \cos\phi \thetacap,
\end{aligned}
\end{equation}
%
so
%
\begin{equation}\label{eqn:advancedantennaProblemSet3Problem1:240}
\BE_\txtm
= j \omega \eta
\lr{
\cos\theta \sin\phi \phicap - \cos\phi \thetacap
}
\frac{\epsilon_0 I_{\textrm{mo}} l}{4 \pi r} e^{-j k r}.
\end{equation}
%
Summing \cref{eqn:advancedantennaProblemSet3Problem1:140}, and \cref{eqn:advancedantennaProblemSet3Problem1:240} gives
%
\begin{equation}\label{eqn:advancedantennaProblemSet3Problem1:260}
\begin{aligned}
\BE
&=
j \omega
\frac{l}{4 \pi r} e^{-j k r} \eta \epsilon_0
\Biglr{
\lr{ -\cos\theta \cos\phi \thetacap + \sin\phi \phicap } \eta I_{\textrm{eo}} \\
&\quad
+
\lr{  \cos\theta \sin\phi \phicap - \cos\phi \thetacap } I_{\textrm{mo}}
}.
\end{aligned}
\end{equation}
%
\makeSubAnswer{}{advancedantenna:problemSet3:1b}
When \( \eta I_{\textrm{eo}} = I_{\textrm{mo}} \), this reduces to
%
\begin{equation}\label{eqn:advancedantennaProblemSet3Problem1:280}
\BE
=
j \omega
\frac{\mu_0 I_{\textrm{eo}} l}{4 \pi r} e^{-j k r}
\lr{ 1 + \cos\theta }
\lr{
-\cos\phi \thetacap
+ \sin\phi \phicap
}.
\end{equation}
%
%Two views of this vector function are plotted in \cref{fig:electricAndMagneticDipoleSuperposition:electricAndMagneticDipoleSuperpositionFig1}
%%, and \cref{fig:electricAndMagneticDipoleSuperposition:electricAndMagneticDipoleSuperpositionFig2}
%showing the electric and magnetic far field vector directions on the surface of the electric field magnitude.  
%See \nbref{electricAndMagneticDipoleSuperpositionStandalone.cdf}
%%\href{http://goo.gl/Sb1Uso}{http://goo.gl/Sb1Uso} 
%for an interactive view of these plots, with \(\theta, \phi\) controls available for the wave vector position.
%
%\mathImageTwoFigures{../figures/ece1229-antenna/electricAndMagneticDipoleSuperpositionFig1}{../figures/ece1229-antenna/electricAndMagneticDipoleSuperpositionFig2}{Electric and magnetic infinitesimal dipole superposition, view from above and below.}{fig:electricAndMagneticDipoleSuperposition:electricAndMagneticDipoleSuperpositionFig1}{scale=0.3}{ps3:electricAndMagneticDipoleSuperposition.nb}
A view of this vector function are plotted in \cref{fig:electricAndMagneticDipoleSuperposition:electricAndMagneticDipoleSuperpositionFig1}
%%, and \cref{fig:electricAndMagneticDipoleSuperposition:electricAndMagneticDipoleSuperpositionFig2}
showing the electric and magnetic far field vector directions on the surface of the electric field magnitude.  
See \nbref{electricAndMagneticDipoleSuperpositionStandalone.cdf}
%\href{http://goo.gl/Sb1Uso}{http://goo.gl/Sb1Uso} 
for an interactive view of these plots, with \(\theta, \phi\) controls available for the wave vector position.
%
\imageFigure{../figures/ece1229-antenna/electricAndMagneticDipoleSuperpositionFig1}{Electric and magnetic infinitesimal dipole superposition, view from above.}{fig:electricAndMagneticDipoleSuperposition:electricAndMagneticDipoleSuperpositionFig1}{0.3}
%\imageFigure{../figures/ece1229-antenna/electricAndMagneticDipoleSuperpositionFig2}{Electric and magnetic infinitesimal dipole superposition, view from below.}{fig:electricAndMagneticDipoleSuperposition:electricAndMagneticDipoleSuperpositionFig2}{0.3}
\makeSubAnswer{}{advancedantenna:problemSet3:1c}
In the zy plane, \( \phi = \pi/2 \), and the electric field has only a \( \phicap \) component
%
\begin{equation}\label{eqn:advancedantennaProblemSet3Problem1:n}
\BE
=
j \omega
\frac{\mu_0 I_{\textrm{eo}} l}{4 \pi r} e^{-j k r}
\lr{ 1 + \cos\theta }
\phicap.
\end{equation}
%
The magnitude of the \( \theta \) variation in the \( [0, 2 \pi] \) interval is plotted in \cref{fig:electricAndMagneticDipoleSuperposition:electricAndMagneticDipoleSuperpositionFig3}.
%
\mathImageFigure{../figures/ece1229-antenna/electricAndMagneticDipoleSuperpositionFig3}{Electric and magnetic infinitesimal dipole superposition, polar plot in ZY plane.}{fig:electricAndMagneticDipoleSuperposition:electricAndMagneticDipoleSuperpositionFig3}{0.3}{ps3:electricAndMagneticDipoleSuperposition.nb}
}
