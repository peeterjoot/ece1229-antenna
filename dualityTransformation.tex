%
% Copyright � 2015 Peeter Joot.  All Rights Reserved.
% Licenced as described in the file LICENSE under the root directory of this GIT repository.
%
%\input{../blogpost.tex}
%\renewcommand{\basename}{dualityTransformation}
%\renewcommand{\dirname}{notes/ece1229/}
%%\newcommand{\dateintitle}{}
%%\newcommand{\keywords}{}
%
%\input{../peeter_prologue_print2.tex}
%\usepackage{macros_bm}
%
%\beginArtNoToc
%
%\generatetitle{Duality transformation}
\section{Duality transformation}
\index{duality transformation}
%\label{chap:dualityTransformation}

In a discussion of Dirac's monopoles, \citep{jackson1975cew} \S 6.12 introduces a duality transformation, forming electric and magnetic fields by forming a rotation that combines a different pair of electric and magnetic fields.  In SI units that transformation becomes
%
\begin{subequations}
\label{eqn:dualityTransformation:20}
\begin{dmath}\label{eqn:dualityTransformation:40}
\begin{bmatrix}
\bcE \\
\eta \bcH
\end{bmatrix}
=
\begin{bmatrix}
\cos\theta & \sin\theta \\
-\sin\theta & \cos\theta
\end{bmatrix}
\begin{bmatrix}
\bcE' \\
\eta \bcH'
\end{bmatrix}
\end{dmath}
\begin{dmath}\label{eqn:dualityTransformation:60}
\begin{bmatrix}
\bcD \\
\bcB/\eta
\end{bmatrix}
=
\begin{bmatrix}
\cos\theta & \sin\theta \\
-\sin\theta & \cos\theta
\end{bmatrix}
\begin{bmatrix}
\bcD' \\
\bcB'/\eta
\end{bmatrix},
\end{dmath}
\end{subequations}
%
where \( \eta = \sqrt{\mu_0/\epsilon_0} \).  It is left as an exercise to the reader to show that application of these to Maxwell's equations
%
\begin{subequations}
\label{eqn:dualityTransformation:80}
\begin{equation}\label{eqn:dualityTransformation:100}
\spacegrad \cdot \bcE = \rho_\txte/\epsilon_0
\end{equation}
\begin{equation}\label{eqn:dualityTransformation:120}
\spacegrad \cdot \bcH = \rho_\txtm/\mu_0
\end{equation}
\begin{equation}\label{eqn:dualityTransformation:140}
-\spacegrad \cross \bcE - \partial_t \bcB =  \bcJ_\txtm
\end{equation}
\begin{equation}\label{eqn:dualityTransformation:160}
\spacegrad \cross \bcH - \partial_t \bcD = \bcJ_\txte,
\end{equation}
\end{subequations}
%
determine a similar relation between the sources.  That transformation of Maxwell's equation is

%\bcE = \cos\theta \bcE' + \sin\theta \eta \bcH'
%\bcH = -\inv{\eta} \sin\theta \bcE' + \cos\theta \bcH'
%\bcD = \cos\theta \bcD' + \sin\theta \bcB/\eta
%\bcB = -\sin\theta \eta \bcD' + \cos\theta \bcB
%
\begin{subequations}
\label{eqn:dualityTransformation:180}
\begin{equation}\label{eqn:dualityTransformation:200}
\spacegrad \cdot \lr{ \cos\theta \bcE' + \sin\theta \eta \bcH' } = \rho_\txte/\epsilon_0
\end{equation}
\begin{equation}\label{eqn:dualityTransformation:220}
\spacegrad \cdot \lr{ -\sin\theta \bcE'/\eta + \cos\theta \bcH' } = \rho_\txtm/\mu_0
\end{equation}
\begin{equation}\label{eqn:dualityTransformation:240}
-\spacegrad \cross \lr{ \cos\theta \bcE' + \sin\theta \eta \bcH' } - \partial_t \lr{ - \sin\theta \eta \bcD' + \cos\theta \bcB' } =  \bcJ_\txtm
\end{equation}
\begin{equation}\label{eqn:dualityTransformation:260}
\spacegrad \cross \lr{ -\sin\theta \bcE'/\eta + \cos\theta \bcH' } - \partial_t \lr{ \cos\theta \bcD' + \sin\theta \bcB'/\eta } = \bcJ_\txte.
\end{equation}
\end{subequations}
%
%or
%
%\begin{subequations}
%\label{eqn:dualityTransformation:280}
%\begin{equation}\label{eqn:dualityTransformation:300}
%\cos\theta \rho_\txte'/\epsilon_0 + \sin\theta \eta \rho_\txtm' = \rho_\txte/\epsilon_0
%\end{equation}
%\begin{equation}\label{eqn:dualityTransformation:320}
%-\sin\theta \rho_\txte'/\epsilon_0\eta + \cos\theta \eta \rho_\txtm' = \rho_\txtm/\mu_0
%\end{equation}
%\begin{equation}\label{eqn:dualityTransformation:340}
%\cos\theta \bcJ_\txtm' - \eta \sin\theta \bcJ_\txte' = \bcJ_\txtm
%\end{equation}
%\begin{equation}\label{eqn:dualityTransformation:360}
%-\sin\theta \bcJ_\txtm'/\eta + \cos\theta \bcJ_\txte' = \bcJ_\txte.
%\end{equation}
%\end{subequations}
%
A bit of rearranging gives
\begin{subequations}
\label{eqn:dualityTransformation:380}
\begin{equation}\label{eqn:dualityTransformation:400}
\begin{bmatrix}
\eta \rho_\txte \\
\rho_\txtm
\end{bmatrix}
=
\begin{bmatrix}
\cos\theta & \sin\theta \\
-\sin\theta & \cos\theta
\end{bmatrix}
\begin{bmatrix}
\eta \rho_\txte' \\
\rho_\txtm'
\end{bmatrix}
\end{equation}
\begin{equation}\label{eqn:dualityTransformation:420}
\begin{bmatrix}
\eta \bcJ_\txte \\
\bcJ_\txtm \\
\end{bmatrix}
=
\begin{bmatrix}
\cos\theta & \sin\theta \\
-\sin\theta & \cos\theta
\end{bmatrix}
\begin{bmatrix}
\eta \bcJ_\txte' \\
\bcJ_\txtm' \\
\end{bmatrix}.
\end{equation}
\end{subequations}
%
For example, with \( \rho_\txtm = \bcJ_\txtm = 0 \), and \( \theta = \pi/2 \), the transformation of sources is
%
\begin{equation}\label{eqn:dualityTransformation:440}
\begin{aligned}
\rho_\txte' &= 0  \\
\bcJ_\txte' &= 0  \\
\rho_\txtm' &= \eta \rho_\txte \\
\bcJ_\txtm' &= \eta \bcJ_\txte,
\end{aligned}
\end{equation}
%
and Maxwell's equations then have only magnetic sources
%
\begin{subequations}
\label{eqn:dualityTransformation:460}
\begin{equation}\label{eqn:dualityTransformation:480}
\spacegrad \cdot \bcE' = 0
\end{equation}
\begin{equation}\label{eqn:dualityTransformation:500}
\spacegrad \cdot \bcH' = \rho_\txtm'/\mu_0
\end{equation}
\begin{equation}\label{eqn:dualityTransformation:520}
-\spacegrad \cross \bcE' - \partial_t \bcB' =  \bcJ_\txtm'
\end{equation}
\begin{equation}\label{eqn:dualityTransformation:540}
\spacegrad \cross \bcH' - \partial_t \bcD' = 0.
\end{equation}
\end{subequations}
%
Of this relation Jackson points out that ``The invariance of the equations of electrodynamics under duality transformations shows that it is a matter of convention to speak of a particle possessing an electric charge, but not magnetic charge.''  This is an interesting comment, and worth some additional thought.

%\EndArticle
