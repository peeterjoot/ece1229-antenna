%
% Copyright � 2015 Peeter Joot.  All Rights Reserved.
% Licenced as described in the file LICENSE under the root directory of this GIT repository.
%
%\input{../blogpost.tex}
%\renewcommand{\basename}{imageTheorem}
%\renewcommand{\dirname}{notes/ece1229/}
%%\newcommand{\dateintitle}{}
%%\newcommand{\keywords}{}
%
%\input{../peeter_prologue_print2.tex}
%
%\beginArtNoToc
%
%\generatetitle{Image theorem}
\section{Image theorem}
\index{image theorem}
%\label{chap:imageTheorem}

In the last problem set we examined the array factor for a corner cube configuration, shown in \cref{fig:homework3:homework3Fig1}.

\imageFigure{../figures/ece1229-antenna/homework3Fig1}{A corner-cube antenna.}{fig:homework3:homework3Fig1}{0.2}

\paragraph{Motivation}

This is a horizontal dipole antenna placed next to a metallic corner.  The radiation at points in the interior of the cube have contributions due to the line of sight field from the antenna as well as reflections.  We looked at an approximation of ground reflections using the \underline{Image Theorem}, modeling the ground as a perfectly conducting surface.  I completely misunderstood that theorem and how it should be applied.  As presented it seemed like a simple way to figure out the reflection characteristics.  This confused me since it did not seem consistent with Fresnel reflection theory.  I did try to reconcile to the two, but that reconciliation only appeared to work for certain dipole orientations, and that orientation dependence remained an open question.
%
It turns out that the idea of the Image Theorem is to find a source configuration that contains the specified source, but contains enough other sources that the tangential component of the electric field superposition is zero on the conducting surface, as required by Maxwell's equations.  This allows the boundary to be completely removed from the problem.
%
Thinking of the corner cube configuration as a reflection problem, I positioned sources as in \cref{fig:incorrectImagePlacementForCornerCube:incorrectImagePlacementForCornerCubeFig2}.

\imageFigure{../figures/ece1229-antenna/incorrectImagePlacementForCornerCubeFig2}{Incorrect Image Theorem source placement for corner cube.}{fig:incorrectImagePlacementForCornerCube:incorrectImagePlacementForCornerCubeFig2}{0.2}

Because of the horizontal orientation of the dipole, I argued that the reflection coefficient should be -1.  The reflection point is a bit messy to calculate, and it turns out to zeroth order in \( h/r \) the \( \sin\theta \) magnitude scaling of the reflected (far-field) field is present for both reflected rays.  I though that this was probably because the observation point lays at the same altitude for both the line of sight ray and the reflected ray.

Attempting this problem as a reflection problem makes it much more difficult than it needs to be.  It turns out that the correct image source placement for this problem is that of \cref{fig:cornerCubeImageSourcePlacement:cornerCubeImageSourcePlacementFig3}.

\imageFigure{../figures/ece1229-antenna/cornerCubeImageSourcePlacementFig3}{Correct image source placement for the corner cube.}{fig:cornerCubeImageSourcePlacement:cornerCubeImageSourcePlacementFig3}{0.2}

This wasn't at all obvious to me.  The key is understanding that the goal of the image source placement isn't to figure out how the reflection will occur, but to manufacture a source configuration for which the tangential component of the electric field is zero on the conducting surface.

\paragraph{Image placement for infinite conducting plane.}
\index{infinite conducting plane}
%
Before thinking about the corner cube configuration, consider a horizontal dipole next to an infinite conducting plane.  This, and the correct image source placement is illustrated in \cref{fig:reflectionOfImagePoints:reflectionOfImagePointsFig1}.

\imageFigure{../figures/ece1229-antenna/reflectionOfImagePointsFig1}{Image source placement for horizontal dipole.}{fig:reflectionOfImagePoints:reflectionOfImagePointsFig1}{0.2}

I'll now verify that this is the correct image source.  This is basically a calculation that the tangential components of the electric fields from both sources sum to zero.

Let,
%
\begin{equation}\label{eqn:imageTheorem:20}
r = \Abs{\Bs - \Br_0},
\end{equation}
%
so that the magnetic vector potential for the first quadrant dipole has the form
%
\begin{equation}\label{eqn:imageTheorem:40}
\BA = \frac{A_0}{4 \pi r} e^{-j k r} \zcap.
\end{equation}
%
With
%
\begin{equation}\label{eqn:imageTheorem:60}
\begin{aligned}
\kcap &= \frac{\Bs - \Br_0}{s} \\
\tilde{\BE} &= \zcap - \lr{\zcap \cdot \kcap} \kcap,
\end{aligned}
\end{equation}
%
the far-field electric field at the point \( \Bs \) on the plane is
%
\begin{equation}\label{eqn:imageTheorem:80}
\BE = -j \omega \frac{A_0}{4 \pi r} e^{-j k r} \tilde{\BE}.
\end{equation}
%
If the normal to the plane is \( \ncap \) the tangential component of this field is the projection of \( \BE \) on the direction
%
\begin{equation}\label{eqn:imageTheorem:100}
\pcap = \frac{\kcap \cross \ncap}{\Abs{\kcap \cross \ncap}}.
\end{equation}
%
That tangential component is directed along
%
\begin{equation}\label{eqn:imageTheorem:120}
\lr{\tilde{\BE} \cdot \pcap } \pcap
=
\lr{\lr{\zcap - \lr{\zcap \cdot \kcap} \kcap} \cdot \lr{\kcap \cross \ncap}} \frac{\kcap \cross \ncap}{\Abs{\kcap \cross \ncap}^2}.
\end{equation}
%
Because the triple product \( \kcap \cdot \lr{\kcap \cross \ncap} = 0 \), the tangential component of the electric field, provided \( \kcap \cdot \ncap \ne 0 \), is
%
\begin{equation}\label{eqn:imageTheorem:140}
\BE_\parallel
=
-j \omega \frac{A_0}{4 \pi r} e^{-j k r} \zcap \cdot \lr{\kcap \cross \ncap} \frac{\kcap \cross \ncap}{ 1 - \lr{ \ncap \cdot \kcap }^2 }.
\end{equation}
%
Now the wave vector direction for the second quadrant ray on the plane is required.  Both \( \kcap' \) and \( \Bs' \) are reflections across the plane.  Any such reflection has the value
%
\begin{dmath}\label{eqn:imageTheorem:160}
\Bx'
= \lr{ \Bx \wedge \ncap} \ncap - \lr{ \Bx \cdot \ncap } \ncap
= - \lr{ \ncap \wedge \Bx + \ncap \cdot \Bx } \ncap
= - \ncap \Bx \ncap.
\end{dmath}
%
This multivector product nicely encapsulates the reflection operation.  Consider a reflection against the y-z plane with normal \( \Be_1 \) to verify that this works
%
\begin{dmath}\label{eqn:imageTheorem:180}
-\Be_1 \Bx \Be_1
=
-\Be_1 \lr{ x \Be_1 + y \Be_2 + z \Be_3 } \Be_1
=
-\lr{ x - y \Be_2 \Be_1 + z \Be_3 \Be_1 } \Be_1
=
-\lr{ x \Be_1 - y \Be_2 + z \Be_3 }
=
- x \Be_1 + y \Be_2 + z \Be_3.
\end{dmath}
%
This has the x component flipped in sign and the rest left untouched as desired for a reflection in the y-z plane.

The second quadrant field will have \( \kcap' \cross \ncap \) terms in place of all the \( \kcap \cross \ncap \) terms of \cref{eqn:imageTheorem:140}.  We want to know how the two compare.  This calculation is simply done using the dual form of the cross product temporarily
%
\begin{dmath}\label{eqn:imageTheorem:200}
\kcap' \cross \ncap
=
-I \lr{ \kcap' \wedge \ncap}
=
-I \gpgradetwo{\kcap' \ncap}
=
-I \gpgradetwo{ {-\ncap \kcap \ncap} \ncap}
=
I \gpgradetwo{ \ncap \kcap }
=
I \ncap \wedge \kcap
=
-\ncap \cross \kcap
=
\kcap \cross \ncap.
\end{dmath}
%
So, provided the image source in the second quadrant is oppositely oriented (sign inversion), the tangential components of the two will sum to zero on that surface.

Thinking back to the corner cube, it is clear that an image source opposite to the source across from one of the walls will result in a zero tangential electric field along this boundary as is the case here (say the y-z plane).  A second pair of sources opposite from each other anywhere else also about the y-z plane will not change that zero tangential electric field on this surface, but if the signs of the sources is alternated as in \cref{fig:cornerCubeImageSourcePlacement:cornerCubeImageSourcePlacementFig3} it will also result in zero tangential electric field on the z-x plane, which has the desired boundary value effects for both surfaces of the corner cube.

Once the image sources are placed, the problem can be tackled with the boundary removed.
%
%\EndNoBibArticle
