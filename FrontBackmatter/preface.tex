%
% Copyright � 2015 Peeter Joot.  All Rights Reserved.
% Licenced as described in the file LICENSE under the root directory of this GIT repository.
%

%
%\chapter{Preface}
% this suppresses an explicit chapter number for the preface.
\chapter*{Preface}%\normalsize
  \thispagestyle{empty}
  \addcontentsline{toc}{chapter}{Preface}

This book contains course notes from the Spring 2015 session of the University of Toronto Advanced Antenna Theory course (ECE1229H), taught by Professor G. V. Eleftheriades.

\paragraph{Official course description:}

``This course deals with the analysis and design of a range of antennas. Topics addressed include: definitions of antenna parameters; vector potentials; solutions to the inhomogeneous wave equation; principles of duality and reciprocity; wire antennas; antenna arrays; phased arrays; synthesis techniques for discrete and continuous line sources; integral equations and solutions using the method of moments; field equivalence principle; aperture antennas; antenna measurement techniques; diffraction; horn antennas; reflector antennas; microstrip antennas; reflectarrays; electrically small antennas; and broadband antennas.''

Synopsis

\begin{enumerate}
\item
Fundamental Antenna Parameters (patterns, directivity, effective aperture, input impedance, Friis transmission equation, radar range equations, RCS)
\item
Review of Maxwell's Equations
\item
Radiation from arbitrary current distributions
\item
Wire and Mobile Communication Antennas: Dipoles, loops, ground effects
\item
Reciprocity; Equivalence of transmit and receive radiation patterns
\item
Phased Arrays
\item
Self Impedance: Integral equations and method of moments (MoM)
\item
Mutual Impedance : Induced EMF method
\item
Aperture Antennas I : Equivalent current method, rectangular apertures, horn antennas
\item
Apertures Antennas II : Plane-wave expansion, slots
\item
Printed and IC Antennas : Microstrip patch antennas, miniaturized antennas
\item
Metamaterial Antennas
\item
Broadband Antennas : Self complementarity, spirals, log periodic, Yagi Uda
\end{enumerate}

References include
\begin{itemize}
	\item (Main Text) \citep{balanis2005antenna} C.A. Balanis, ``Antenna Theory,'' Wiley, 3rd Edition
   \item (Recommended) \citep{stutzman1998antenna} W.L. Stutzman and G. A. Thiele, ``Antenna Theory and Design'' 2nd Edition, Wiley.
   \item (Recommended) \citep{eleftheriades2005negative} G.V. Eleftheriades and K.G. Balmain (Edt.) ``Negative-Refraction Metamaterials'', Wiley and IEEE Press.
\end{itemize}
\paragraph{This document contains:}
\begin{itemize}
\item Personal notes exploring details that were not clear to me from the lectures, or from the texts associated with the lecture material.
%\ifthenelse{\boolean{redacted}}%
%{%
\item Assigned problems.  Like anything else take these as is.  I have attempted to either correct errors or mark them as such.%
%}%
%{\relax}
%
%\item Some worked problems attempted as course prep, for fun, or for test preparation, or post test reflection.
%\item Links to Matlab function implementations associated with the problem sets.
\end{itemize}
This set of notes is significantly different from my notes for many other classes.  With the class taught on slides (and some of those slides mirroring the text closely), I did not take live notes in class.
These notes fill in details that I felt deserved clarification, contain problem sets solutions, as well as a number of loosely related musings on Geometric Algebra equivalents to some of the generalized concepts of electromagnetic theory encountered in this class (i.e. magnetic sources).
% blurb from blog posts.  merge these with fixedup grammar.
% Unlike most of the other classes I have taken, I am not attempting to take comprehensive notes for this class. The class is taught on slides which go by faster than I can easily take notes for (and some of which match the textbook closely).  In class I have annotated my copy of textbook with little details instead.  This set of notes contains musings of details that were unclear, or in some cases, details that were provided in class, but are not in the text (and too long to pencil into my book), as well as some notes Geometric Algebra formalism for Maxwell's equations with magnetic sources (something I've encountered for the first time in any real detail in this class).

My thanks go to Professor Eleftheriades for teaching this course.

Peeter Joot  \quad peeterjoot@pm.me
