%
% Copyright © 2015 Peeter Joot.  All Rights Reserved.
% Licenced as described in the file LICENSE under the root directory of this GIT repository.
%
\section{Vertical dipole reflection coefficient.}
\index{vertical dipole}
\index{reflection coefficient}
%
In class a ground reflection scenario was covered for a horizontal dipole.  Reading the text I was surprised to see what looked like the same sort of treatment \S 4.7.2, but ending up with a quite different result.  It turns out the difference is because the text was treating the vertical dipole configuration, whereas Prof. Eleftheriades was treating a horizontal dipole configuration, which have different reflection coefficients.  These differing reflection coefficients are due to differences in the polarization of the field.

To understand these differences in reflection coefficients, consider first the field due to a vertical dipole as sketched in \cref{fig:verticalDipoleConfiguration:verticalDipoleConfigurationFig1}, with a wave vector directed from the transmission point downwards in the z-y plane.
\index{wave vector}
%
\imageFigure{../figures/ece1229-antenna/verticalDipoleConfigurationFig1}{Vertical dipole configuration.}{fig:verticalDipoleConfiguration:verticalDipoleConfigurationFig1}{0.2}
%
The wave vector has direction
%
\begin{equation}\label{eqn:chapter4Notes:560}
\kcap = \zcap e^{\zcap \xcap \theta} = \zcap \cos\theta + \ycap \sin\theta.
\end{equation}
%
Suppose that the (magnetic) vector potential is that of an infinitesimal dipole
%
\begin{equation}\label{eqn:chapter4Notes:580}
\BA = \zcap \frac{\mu_0 I_0 l}{4 \pi r} e^{-j k r}
.
%= \frac{A_r}{4 \pi r} e^{-j k r}
\end{equation}
%
The electric field, in the far field, can be computed by computing the normal projection to the wave vector direction
%
\begin{equation}\label{eqn:chapter4Notes:600}
\begin{aligned}
\BE &= -j \omega \lr{\BA \wedge \kcap} \cdot \kcap
\\ &= -j \omega \frac{\mu_0 I_0 l}{4 \pi r} \lr{\zcap \wedge \lr{\zcap \cos\theta + \ycap \sin\theta} } \lr{\zcap \cos\theta + \ycap \sin\theta}
\\ &= -j \omega \frac{\mu_0 I_0 l}{4 \pi r} \lr{ \zcap \ycap \sin\theta } \lr{\zcap \cos\theta + \ycap \sin\theta}
\\ &= -j \omega \frac{\mu_0 I_0 l}{4 \pi r} \sin\theta \lr{-\ycap \cos\theta + \zcap \sin\theta}
\\ &= j \omega \frac{\mu_0 I_0 l}{4 \pi r} \sin\theta \ycap e^{\zcap \ycap \theta}.
\end{aligned}
\end{equation}
%
This is directed in the z-y plane rotated an additional \( \pi/2 \) past \( \kcap \).  The magnetic field must then be directed into the page, along the x axis.  This is sketched in \cref{fig:verticalDipoleConfiguration:verticalDipoleConfigurationFig2}.
%
\imageFigure{../figures/ece1229-antenna/verticalDipoleConfigurationFig2}{Electric and magnetic field directions.}{fig:verticalDipoleConfiguration:verticalDipoleConfigurationFig2}{0.3}
%
Referring to \citep{hecht1998hecht} (\texteqnref{4.40}) for the coefficient of reflection component
%
\begin{equation}\label{eqn:chapter4Notes:620}
R
=
\frac{
n_t \cos\theta_i - n_i \cos\theta_t
}
{
n_i \cos\theta_i + n_t \cos\theta_t
}.
\end{equation}
%
This is the Fresnel equation for the case when
that corresponds to
%continuity of the components of \( \BE \) and \( \BB \) that lie in the plane of reflection (where
\( \BE \) lies in the plane of incidence, and the magnetic field is completely parallel to the plane of reflection).  For the no transmission case, allowing \( v_t \rightarrow 0 \), the index of refraction is \( n_t = c/v_t \rightarrow \infty \), and the reflection coefficient is \( 1 \) as claimed in \S 4.7.2 of \citep{balanis2005antenna}.  Because of the symmetry of this dipole configuration, the azimuthal angle that the wave vector is directed along does not matter.
\index{plane of incidence}
\index{plane of reflection}
\index{index of refraction}
%
\section{Horizontal dipole reflection coefficient.}
\index{horizontal dipole}
\index{reflection coefficient}
%
In the class notes, a horizontal dipole coming out of the page is indicated.  With the page representing the z-y plane, this is a magnetic vector potential directed along the x-axis direction
%
\begin{equation}\label{eqn:chapter4Notes:640}
\BA = \xcap \frac{\mu_0 I_0 l}{4 \pi r} e^{-j k r}.
%= \frac{A_r}{4 \pi r} e^{-j k r}
\end{equation}
%
For a wave vector directed in the z-y plane as in \cref{eqn:chapter4Notes:560}, the electric far field is directed along
%
\begin{equation}\label{eqn:chapter4Notes:660}
\begin{aligned}
\lr{ \xcap \wedge \kcap } \cdot \kcap
&=
\xcap - \lr{ \xcap \cdot \kcap } \kcap
\\ &=
\xcap - \lr{ \cancel{\xcap \cdot \lr{
\zcap \cos\theta + \ycap \sin\theta
}} } \kcap
\\ &= \xcap.
\end{aligned}
\end{equation}
%
The electric far field lies completely in the plane of reflection.  From \citep{hecht1998hecht} (\texteqnref{4.34}), the Fresnel reflection coefficients is
\index{Fresnel equations}
%
\begin{equation}\label{eqn:chapter4Notes:680}
R =
\frac{
n_i \cos\theta_i - n_t \cos\theta_t
}
{
n_i \cos\theta_i + n_t \cos\theta_t
},
\end{equation}
%
which approaches \( -1 \) when \( n_t \rightarrow \infty \).  This is consistent with the image theorem summation that Prof. Eleftheriades used in class.
%
\paragraph{Azimuthal angle dependency of the reflection coefficient.}
%
Now consider a horizontal dipole directed along the y-axis.  For the same wave vector direction as above, the electric far field is now directed along
%
\begin{equation}\label{eqn:chapter4Notes:700}
\begin{aligned}
\lr{ \ycap \wedge \kcap } \cdot \kcap
&=
\ycap - \lr{ \ycap \cdot \kcap } \kcap
\\ &=
\ycap - \lr{ \ycap \cdot \lr{
\zcap \cos\theta + \ycap \sin\theta
} } \kcap
\\ &=
\ycap - \kcap \sin\theta
\\ &=
\ycap - \sin\theta \lr{
\zcap \cos\theta + \ycap \sin\theta
}
\\ &=
\ycap \cos^2 \theta - \sin\theta \cos\theta \zcap
\\ &= \cos\theta \lr{ \ycap \cos\theta - \sin\theta \zcap }
\\ &= \cos\theta \ycap e^{ \zcap \ycap \theta }.
\end{aligned}
\end{equation}
%
That is
%
\begin{equation}\label{eqn:chapter4Notes:720}
\BE =
-j \omega \frac{\mu_0 I_0 l}{4 \pi r} e^{-j k r}
\cos\theta \ycap e^{ \zcap \ycap \theta }.
\end{equation}
%
This far field electric field lies in the plane of incidence (a direction of \( \thetacap \) rotated by \( \pi/2 \)), not in the plane of reflection.  The corresponding magnetic field should be directed along the plane of reflection, which is easily confirmed by calculation
%
\begin{equation}\label{eqn:chapter4Notes:740}
\begin{aligned}
\kcap \cross
\lr{ \ycap \cos\theta - \sin\theta \zcap }
&=
\lr{ \zcap \cos\theta + \ycap \sin\theta } \cross
\lr{ \ycap \cos\theta - \sin\theta \zcap }
\\ &=
-\xcap \cos^2 \theta - \xcap \sin^2\theta
\\ &= -\xcap.
\end{aligned}
\end{equation}
%
The far field magnetic field is seen to be
%
\begin{equation}\label{eqn:chapter4Notes:721}
\BH =
j \omega \frac{I_0 l}{4 \pi r} e^{-j k r}
\cos\theta \xcap,
\end{equation}
%
so a reflection coefficient of \( 1 \) is required to calculate the power loss after a single ground reflection signal bounce for this relative orientation of antenna to the target.
\index{ground reflection}
%
I fail to see how the horizontal dipole treatment in \S 4.7.5 can use a single reflection coefficient without taking into account the azimuthal dependency of that reflection coefficient.

Reflecting on this (no pun intended), made me realize that the no transmission case has some interesting aspects.  One of these is that radiation momentum must be transferred to the reflecting surface in some fashion since the direction of the incident radiation changes.  Perhaps this is why the use of Image theory seems to be careful to state that the reflecting plane is a perfect electrical conductor.  Study of reflection off of conducting surfaces is clearly in order to understand how this differs from normal reflection in transmitting media.
%
%\EndArticle
