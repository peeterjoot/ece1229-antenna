%
% Copyright � 2015 Peeter Joot.  All Rights Reserved.
% Licenced as described in the file LICENSE under the root directory of this GIT repository.
%
%\input{../blogpost.tex}
%\renewcommand{\basename}{reciprocityTheoremGA}
%\renewcommand{\dirname}{notes/ece1229/}
%%\newcommand{\dateintitle}{}
%%\newcommand{\keywords}{}
%
%\input{../peeter_prologue_print2.tex}
%
%\beginArtNoToc
%
%\generatetitle{Reciprocity theorem in Geometric Algebra}
\index{Geometric Algebra}
%\chapter{Reciprocity theorem}
\index{reciprocity theorem}
\index{tensor}
%\label{chap:reciprocityTheoremGA}
The reciprocity theorem involves a Poynting like \textAndIndex{antisymmetric} difference of the following form
%
\begin{equation}\label{eqn:reciprocityTheoremGA:20}
\BE^{(a)} \cross \BH^{(b)} - \BE^{(b)} \cross \BH^{(a)}.
\end{equation}
%
This smells like something that can probably be related to a combined electromagnetic field multivectors \index{multivector} in some sort of structured fashion.  Guessing that this is related to the antisymmetric sum of two electromagnetic field multivectors turns out to be correct.  Let
%
\begin{subequations}
\label{eqn:reciprocityTheoremGA:40}
\begin{dmath}\label{eqn:reciprocityTheoremGA:60}
F^{(a)} = \BE^{(a)} + c \BB^{(a)} I
\end{dmath}
\begin{dmath}\label{eqn:reciprocityTheoremGA:80}
F^{(b)} = \BE^{(b)} + c \BB^{(b)} I.
\end{dmath}
\end{subequations}
%
Now form the antisymmetric sum
%
\begin{dmath}\label{eqn:reciprocityTheoremGA:100}
\inv{2} \lr{ F^{(a)} F^{(b)} - F^{(b)} F^{(a)} }
=
\inv{2} \lr{\BE^{(a)} + c \BB^{(a)} I }
\lr{\BE^{(b)} + c \BB^{(b)} I }
-
\inv{2} \lr{\BE^{(b)} + c \BB^{(b)} I }
\lr{\BE^{(a)} + c \BB^{(a)} I }
=
\inv{2} \lr{ \BE^{(a)} \BE^{(b)} -\BE^{(b)} \BE^{(a)} }
+ \frac{I c}{2} \lr{ \BE^{(a)} \BB^{(b)} - \BB^{(b)} \BE^{(a)} }
+ \frac{I c}{2} \lr{ \BB^{(a)} \BE^{(b)} - \BE^{(b)} \BB^{(a)} }
+ \frac{c^2}{2} \lr{ \BB^{(b)} \BB^{(a)} - \BB^{(a)} \BB^{(b)} }
=
\BE^{(a)} \wedge \BE^{(b)} + c^2 \lr{ \BB^{(b)} \wedge \BB^{(a)} }
+ I c \lr{
   \BE^{(a)} \wedge \BB^{(b)}
+
   \BB^{(a)} \wedge \BE^{(b)}
}
=
I \BE^{(a)} \cross \BE^{(b)} + c^2 I \lr{ \BB^{(b)} \cross \BB^{(a)} }
-
c \lr{
   \BE^{(a)} \cross \BB^{(b)}
+
   \BB^{(a)} \cross \BE^{(b)}
}.
\end{dmath}
%
This has two components, the first is a bivector (pseudoscalar times vector) that includes all the non-mixed products, and the second is a vector that includes all the mixed terms.  We can therefore write the antisymmetric difference of the reciprocity theorem by extracting just the grade \index{grade selection} one terms of the antisymmetric sum of the combined electromagnetic field
%
\begin{equation}\label{eqn:reciprocityTheoremGA:120}
\BE^{(a)} \cross \BH^{(b)} - \BE^{(b)} \cross \BH^{(a)}
=
-\frac{1}{2 c \mu_0} \gpgradeone{ \lr{ F^{(a)} F^{(b)} - F^{(b)} F^{(a)} } }.
\end{equation}
%
Observing that the antisymmetrization used in the reciprocity theorem is only one portion of the larger electromagnetic field antisymmetrization, introduces two new questions
%
\begin{enumerate}
\item How would the reciprocity theorem be derived directly in terms of \( F^{(a)} F^{(b)} - F^{(b)} F^{(a)} \)?
\item What is the significance of the other portion of this antisymmetrization \( \BE^{(a)} \cross \BE^{(b)} - c^2 \mu_0^2 \lr{ \BH^{(a)} \cross \BH^{(b)} } \) ?
\end{enumerate}
%
%\EndNoBibArticle
