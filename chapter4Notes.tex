%
% Copyright � 2015 Peeter Joot.  All Rights Reserved.
% Licenced as described in the file LICENSE under the root directory of this GIT repository.
%
%%%\input{../blogpost.tex}
%%%\renewcommand{\basename}{chapter4Notes}
%%%\renewcommand{\dirname}{notes/ece1229/}
%%%\newcommand{\keywords}{ECE1229H}
%%%\input{../peeter_prologue_print2.tex}
%%%
%%%\usepackage{ece1229}
%%%
%%%\beginArtNoToc
%%%%\generatetitle{ECE1229H Advanced Antenna Theory.  Lecture 4: Linear wire antennas.  Taught by Prof.\ G.V. Eleftheriades}
%%%\generatetitle{Linear wire antennas}
\index{linear antenna}
%%%%\chapter{Linear wire antennas}
\label{chap:chapter4Notes}
%
%\section{Disclaimer}
%
%Peeter's lecture notes from class.  These may be incoherent and rough.
%%%
%%%These are notes for the UofT course ECE1229, Advanced Antenna Theory, taught by Prof. Eleftheriades, covering \chaptext 4 \citep{balanis2005antenna} content.
%%%
%%%Unlike most of the other classes I have taken, I am not attempting to take comprehensive notes for this class.  The class is taught on slides that match the textbook so closely, there is little value to me taking notes that just replicate the text.  Instead, I am annotating my copy of textbook with little details instead.  My usual notes collection for the class will contain musings of details that were unclear, or in some cases, details that were provided in class, but are not in the text (and too long to pencil into my book.)
%%%
\section{Magnetic Vector Potential.}
\index{magnetic vector potential}
%
The symbol 
\( \BA \) has been referred to as the \underline{Magnetic Vector Potential} in 
class and in the problem set.
My recollection was that we called this the \underline{Vector Potential}\index{vector potential}.
Prefixing this with magnetic seemed counter intuitive to me since it is generated by electric sources (charges and currents).
This terminology can be justified due to the fact that \( \BA \) generates the magnetic field by its curl.
\index{magnetic field}
\index{curl}
\index{magnetic vector potential}
Some mention of this can be found in \citep{wiki:magneticPotential}, which also points out that the \underline{Electric Potential} refers to the scalar \( \phi \).
\index{electric potential}
\index{electric vector potential}
Prof. Eleftheriades points out that \underline{Electric Vector Potential} refers to the vector potential \( \BF \) generated by magnetic sources (because in that case the electric field is generated by the curl of \( \BF \).)
%
\section{Plots of infinitesimal dipole radial dependence.}
\index{infinitesimal dipole}
%
In \S 4.2 of \citep{balanis2005antenna} are some discussions of the \( kr < 1 \), \( kr = 1 \), and \( kr > 1 \) radial dependence of the fields and power of a solution to an infinitesimal dipole system.  Here are some plots of those \( k r \) dependence, along with the \( k r = 1 \) contour as a reference.  All the \( \theta \) dependence and any scaling is left out.

The CDF notebook
\nbref{visualizeDipoleFields.cdf}
is available to interactively plot these, rotate the plots and change the ranges of what is plotted.

Plots of the real and imaginary parts of 
\begin{equation}\label{eqn:chapter4Notes:69}
\begin{aligned}
H_\phi &= \frac{j k}{r} e^{-j k r} \lr{ 1-\frac{j}{k r} } \\
E_r &= \inv{r^2} \lr{1-\frac{j}{k r}} e^{-j k r} \\
E_\theta &= \frac{ j k }{r} \lr{1 -\frac{j}{k r} -\frac{1}{k^2 r^2} } e^{-j k r},
\end{aligned}
\end{equation}
can be found in
\cref{fig:infinitesimalDipoleHphiReal:infinitesimalDipoleHphiRealFig3},
\cref{fig:infinitesimalDipoleErReal:infinitesimalDipoleErRealFig1}, and
\cref{fig:infinitesimalDipoleEthetaReal:infinitesimalDipoleEthetaRealFig5} respectively.
\mathImageTwoFigures{../figures/ece1229-antenna/infinitesimalDipoleHphiRealFig3}{../figures/ece1229-antenna/infinitesimalDipoleHphiImagFig4}{Radial dependence of \( \Real H_\phi \) and \( \Imag H_\phi \).}{fig:infinitesimalDipoleHphiReal:infinitesimalDipoleHphiRealFig3}{scale=0.3}{selectedInfinitesimalDipolePlots.nb}
\mathImageTwoFigures{../figures/ece1229-antenna/infinitesimalDipoleErRealFig1}{../figures/ece1229-antenna/infinitesimalDipoleErImagFig2}{Radial dependence of \( \Real E_r\) and \( \Imag E_r\).}{fig:infinitesimalDipoleErReal:infinitesimalDipoleErRealFig1}{scale=0.3}{selectedInfinitesimalDipolePlots.nb}
\mathImageTwoFigures{../figures/ece1229-antenna/infinitesimalDipoleEthetaRealFig5}{../figures/ece1229-antenna/infinitesimalDipoleEthetaImagFig6}{Radial dependence of \( \Real E_\theta\) and \( \Imag E_\theta\).}{fig:infinitesimalDipoleEthetaReal:infinitesimalDipoleEthetaRealFig5}{scale=0.3}{selectedInfinitesimalDipolePlots.nb}
%\imageFigure{../figures/ece1229-antenna/infinitesimalDipoleHphiRealFig3}{Radial dependence of \( \Real H_\phi \).}{fig:infinitesimalDipoleHphiReal:infinitesimalDipoleHphiRealFig3}{0.3}
%\imageFigure{../figures/ece1229-antenna/infinitesimalDipoleHphiImagFig4}{Radial dependence of \( \Imag H_\phi \).}{fig:infinitesimalDipoleHphiImag:infinitesimalDipoleHphiImagFig4}{0.3}
% and
%\cref{fig:infinitesimalDipoleErImag:infinitesimalDipoleErImagFig2}.
%\imageFigure{../figures/ece1229-antenna/infinitesimalDipoleErRealFig1}{Radial dependence of \( \Real E_r\).}{fig:infinitesimalDipoleErReal:infinitesimalDipoleErRealFig1}{0.3}
%\imageFigure{../figures/ece1229-antenna/infinitesimalDipoleErImagFig2}{Radial dependence of \( \Imag E_r\).}{fig:infinitesimalDipoleErImag:infinitesimalDipoleErImagFig2}{0.3}
%Finally, a plot of the real and imaginary parts of 
% and
%\cref{fig:infinitesimalDipoleEthetaImag:infinitesimalDipoleEthetaImagFig6}.
%
%
%\imageFigure{../figures/ece1229-antenna/infinitesimalDipoleEthetaRealFig5}{Radial dependence of \( \Real E_\theta\).}{fig:infinitesimalDipoleEthetaReal:infinitesimalDipoleEthetaRealFig5}{0.3}
%\imageFigure{../figures/ece1229-antenna/infinitesimalDipoleEthetaImagFig6}{Radial dependence of \( \Imag E_\theta\).}{fig:infinitesimalDipoleEthetaImag:infinitesimalDipoleEthetaImagFig6}{0.3}
%
\section{Electric Far field for a spherical potential.}
\index{spherical potential}
%
It is interesting to look at the far electric field associated with an arbitrary spherical magnetic vector potential, assuming all of the radial dependence is in the spherical envelope.  That is
\index{far field}
%
\begin{equation}\label{eqn:chapter4Notes:20}
\BA = \frac{e^{-j k r}}{r} \lr{
 \rcap a_r\lr{ \theta, \phi }
+\thetacap a_\theta\lr{ \theta, \phi }
+\phicap a_\phi\lr{ \theta, \phi }
}.
\end{equation}
%
The electric field is
%
\begin{equation}\label{eqn:chapter4Notes:40}
\BE = - j \omega \BA - j \frac{1}{\omega \mu_0 \epsilon_0 } \spacegrad \lr{\spacegrad \cdot \BA }.
\end{equation}
%
\index{divergence!spherical coordinates}
\index{gradient!spherical coordinates}
The divergence and gradient in spherical coordinates are
%
\begin{subequations}
\label{eqn:chapter4Notes:60}
\begin{equation}\label{eqn:chapter4Notes:80}
\spacegrad \cdot \BA
=
\inv{r^2} \PD{r}{} \lr{ r^2 A_r }
+ \inv{r \sin\theta } \PD{\theta}{} \lr{A_\theta \sin\theta}
+ \inv{r \sin\theta } \PD{\phi}{A_\phi}
\end{equation}
\begin{equation}\label{eqn:chapter4Notes:100}
\spacegrad \psi
=
\rcap \PD{r}{\psi}
+\frac{\thetacap}{r} \PD{\theta}{\psi}
+ \frac{\phicap}{r \sin\theta} \PD{\phi}{\psi}.
\end{equation}
\end{subequations}
%
For the assumed potential, the divergence is
%
\begin{equation}\label{eqn:chapter4Notes:120}
\begin{aligned}
&\spacegrad \cdot \BA \\
&=
\frac{a_r}{r^2} \PD{r}{} \lr{ r^2 \frac{e^{-j k r}}{r} }
+ \inv{r \sin\theta } \frac{e^{-j k r}}{r} \PD{\theta}{} \lr{\sin\theta a_\theta}
+ \inv{r \sin\theta } \frac{e^{-j k r}}{r} \PD{\phi}{a_\phi}
\\ &=
a_r
e^{-j k r}
\lr{
\inv{r^2}
-j k \inv{r}
}
+ \inv{r^2 \sin\theta } e^{-j k r} \PD{\theta}{} \lr{\sin\theta a_\theta}
+ \inv{r^2 \sin\theta } e^{-j k r} \PD{\phi}{a_\phi} \\
&\approx
-j k \frac{a_r}{r}
e^{-j k r}.
\end{aligned}
\end{equation}
%
The last approximation dropped all the \( 1/r^2 \) terms that will be small compared to \( 1/r \) contribution that dominates when \( r \rightarrow \infty \), the far field.

The gradient can now be computed
%
\begin{equation}\label{eqn:chapter4Notes:140}
\begin{aligned}
&\spacegrad \lr{\spacegrad \cdot \BA } \approx
-j k
\spacegrad
\lr{
\frac{a_r}{r}
e^{-j k r}
} \\
&=
-j k \lr{
\rcap \PD{r}{}
+\frac{\thetacap}{r} \PD{\theta}{}
+ \frac{\phicap}{r \sin\theta} \PD{\phi}{}
}
\frac{a_r}{r}
e^{-j k r}
\\ &=
-j k \lr{
\rcap a_r \PD{r}{} \lr{
\frac{1}{r}
e^{-j k r}
}
+\frac{\thetacap}{r^2}
e^{-j k r}
\PD{\theta}{a_r}
+
e^{-j k r}
\frac{\phicap}{r^2 \sin\theta}
\PD{\phi}{a_r}
}
\\ &=
-j k \lr{
-\rcap \frac{a_r}{r^2} \lr{
1
+ j k r
}
+\frac{\thetacap}{r^2}
\PD{\theta}{a_r}
+
\frac{\phicap}{r^2 \sin\theta}
\PD{\phi}{a_r}
}
e^{-j k r}
\approx
- k^2 \rcap \frac{a_r}{r}
e^{-j k r}.
\end{aligned}
\end{equation}
%
Again, a far field approximation has been used to kill all the \( 1/r^2 \) terms.
%
The far field approximation of the electric field is now possible
%
\begin{equation}\label{eqn:chapter4Notes:160}
\begin{aligned}
\BE
&= - j \omega \BA - j \frac{1}{\omega \mu_0 \epsilon_0 } \spacegrad \lr{\spacegrad \cdot \BA } \\
&=
- j \omega
\frac{e^{-j k r}}{r} \lr{
 \rcap a_r\lr{ \theta, \phi }
+\thetacap a_\theta\lr{ \theta, \phi }
+\phicap a_\phi\lr{ \theta, \phi }
}
 + j \frac{1}{\omega \mu_0 \epsilon_0 }
 k^2 \rcap \frac{a_r}{r}
e^{-j k r} \\
&=
- j \omega
\frac{e^{-j k r}}{r} \lr{
 \cancel{\rcap a_r\lr{ \theta, \phi }}
+\thetacap a_\theta\lr{ \theta, \phi }
+\phicap a_\phi\lr{ \theta, \phi }
}
 + \cancel{j \frac{c^2}{\omega }
 \lr{\frac{\omega}{c}}^2 \rcap \frac{a_r}{r}
e^{-j k r}
} \\
&=
- j \omega
\frac{e^{-j k r}}{r} \lr{
\thetacap a_\theta\lr{ \theta, \phi }
+\phicap a_\phi\lr{ \theta, \phi }
}.
\end{aligned}
\end{equation}
%
Observe the perfect, somewhat miraculous seeming, cancellation of all the radial components of the field.  If \( \BA_\txtT \) is the non-radial projection of \( \BA \), the electric far field is just
%
\boxedEquation{eqn:chapter4Notes:180}{
%\begin{equation}\label{eqn:chapter4Notes:180}
\BE_{\textrm{ff}} = -j \omega \BA_\txtT.
%\end{equation}
}
%
\section{Magnetic Far field for a spherical potential.}
%
Application of the same assumed representation for the magnetic field gives
\begin{equation}\label{eqn:chapter4Notes:220}
\begin{aligned}
\BB
&=
\spacegrad \cross \BA
\\ &=
\frac{\rcap}{r \sin\theta} \partial_\theta \lr{A_\phi \sin\theta}
+ \frac{\thetacap}{r} \lr{ \inv{\sin\theta} \partial_\phi A_r - \partial_r \lr{r A_\phi}} \\
&\quad+ \frac{\phicap}{r} \lr{ \partial_r\lr{r A_\theta} - \partial_\theta A_r}
\\ &=
\frac{\rcap}{r \sin\theta} \partial_\theta \lr{
\frac{e^{-j k r}}{r} a_\phi
\sin\theta}
+ \frac{\thetacap}{r} \lr{ \inv{\sin\theta} \partial_\phi \lr{
\frac{e^{-j k r}}{r} a_r
} - \partial_r \lr{r
\frac{e^{-j k r}}{r} a_\phi
}
} \\
&\quad+ \frac{\phicap}{r} \lr{ \partial_r\lr{r
\frac{e^{-j k r}}{r} a_\theta
} - \partial_\theta
\lr{
\frac{e^{-j k r}}{r} a_r
}
}
\\ &=
\frac{\rcap}{r \sin\theta}
\frac{e^{-j k r}}{r}
\partial_\theta \lr{
a_\phi
\sin\theta}
+ \frac{\thetacap}{r} \lr{ \inv{\sin\theta}
\frac{e^{-j k r}}{r}
\partial_\phi
a_r
- \partial_r \lr{
e^{-j k r}
}
a_\phi
} \\
&\quad
+ \frac{\phicap}{r} \lr{
\partial_r
\lr{
e^{-j k r}
}
a_\theta
-
\frac{e^{-j k r}}{r}
\partial_\theta
a_r
} \\
&\approx
j k \lr{ \thetacap a_\phi
-
\phicap a_\theta
}
\frac{e^{-j k r}}{r}
\\ &=
-j k \rcap \cross \lr{
\thetacap a_\theta
+\phicap a_\phi
}
\frac{e^{-j k r}}{r}
\\ &=
\inv{c} \BE_{\textrm{ff}}.
\end{aligned}
\end{equation}
%
The approximation above drops the \( 1/r^2 \) terms.  Since
%
\begin{equation}\label{eqn:chapter4Notes:240}
\inv{\mu_0 c} = \inv{\mu_0} \sqrt{\mu_0\epsilon_0} = \sqrt{\frac{\epsilon_0}{\mu_0}} = \inv{\eta},
\end{equation}
%
the magnetic far field can be expressed in terms of the electric far field as
\index{far field!magnetic}
\boxedEquation{eqn:chapter4Notes:260}{
%\begin{equation}\label{eqn:chapter4Notes:260}
\BH = \inv{\eta} \rcap \cross \BE.
%\end{equation}
}
%
\section{Plane wave relations between electric and magnetic fields.}
\index{plane wave}
%
I recalled an identity of the form \cref{eqn:chapter4Notes:260} in \citep{jackson1975cew}, but didn't think that it required a far field approximation.
The reason for this was because the Jackson identity assumed a plane wave representation of the field, something that the far field assumptions also locally require.

Assuming a plane wave representation for both fields
%
\begin{subequations}
\label{eqn:chapter4Notes:280}
\begin{equation}\label{eqn:chapter4Notes:300}
\bcE(\Bx, t) = \BE e^{j \lr{\omega t - \Bk \cdot \Bx}}
\end{equation}
\begin{equation}\label{eqn:chapter4Notes:320}
\bcB(\Bx, t) = \BB e^{j \lr{\omega t - \Bk \cdot \Bx}}.
\end{equation}
\end{subequations}
%
The cross product relation between the fields follows from the Maxwell-Faraday law of induction
%
\begin{equation}\label{eqn:chapter4Notes:340}
0 = \spacegrad \cross \bcE + \PD{t}{\bcB},
\end{equation}
%
or
%
\begin{equation}\label{eqn:chapter4Notes:360}
\begin{aligned}
0
&=
\Be_r \cross \BE \partial_r e^{j\lr{ \omega t - \Bk \cdot \Bx}}
+
j \omega \BB e^{j \lr{\omega t - \Bk \cdot \Bx}}
\\ &=
-j \Be_r k_r \cross \BE e^{j \lr{\omega t - \Bk \cdot \Bx}}
+
j \omega \BB e^{j \lr{\omega t - \Bk \cdot \Bx}}
\\ &=
\lr{ - \Bk \cross \BE + \omega \BB } j
e^{j \lr{\omega t - \Bk \cdot \Bx}},
\end{aligned}
\end{equation}
%
or
%
\begin{equation}\label{eqn:chapter4Notes:380}
\begin{aligned}
\BH
&= \frac{ k}{k c \mu_0 } \kcap \cross \BE
\\ &= \inv{ \eta } \kcap \cross \BE,
\end{aligned}
\end{equation}
%
which also finds \cref{eqn:chapter4Notes:260}, but with much less work and less mess.
%
\section{Transverse only nature of the far-field fields.}
\index{transverse nature}
%
Also observe that its possible to tell that the far field fields have only transverse components using the same argument that they are locally plane waves at that distance.  The plane waves must satisfy the zero divergence Maxwell's equations
%
%
\begin{subequations}
\label{eqn:chapter4Notes:400}
\begin{equation}\label{eqn:chapter4Notes:420}
\spacegrad \cdot \bcE = 0
\end{equation}
\begin{equation}\label{eqn:chapter4Notes:440}
\spacegrad \cdot \bcB = 0,
\end{equation}
\end{subequations}
%
so by the same logic
%
\begin{subequations}
\label{eqn:chapter4Notes:460}
\begin{equation}\label{eqn:chapter4Notes:480}
\Bk \cdot \BE = 0
\end{equation}
\begin{equation}\label{eqn:chapter4Notes:500}
\Bk \cdot \BB = 0.
\end{equation}
\end{subequations}
%
In the far field the electric field must equal its transverse projection
\index{electric field!transverse projection}
%
\begin{equation}\label{eqn:chapter4Notes:520}
\BE = \Proj_\T \lr{-j \omega \BA
- j \frac{1}{\omega \mu_0 \epsilon_0 } \spacegrad \lr{\spacegrad \cdot \BA } }.
\end{equation}
%
Since by \cref{eqn:chapter4Notes:140} the scalar potential term has only a radial component, that leaves
\index{scalar potential}
%
\begin{equation}\label{eqn:chapter4Notes:540}
\BE = -j \omega \Proj_\T \BA,
\end{equation}
%
which provides \cref{eqn:chapter4Notes:180} with slightly less work.
%
