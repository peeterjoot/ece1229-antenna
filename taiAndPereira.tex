%
% Copyright � 2015 Peeter Joot.  All Rights Reserved.
% Licenced as described in the file LICENSE under the root directory of this GIT repository.
%
%\input{../blogpost.tex}
%\renewcommand{\basename}{taiAndPereira}
%\renewcommand{\dirname}{notes/ece1229/}
%%\newcommand{\dateintitle}{}
%%\newcommand{\keywords}{}
%
%\input{../peeter_prologue_print2.tex}
%
%\usepackage{ece1229}
%\usepackage{peeters_layout_exercise}
%
%\beginArtNoToc
%
%\generatetitle{On Tai and Pereira's half power beamwidth approximations}
%\chapter{On Tai and Pereira's half power beamwidth approximations}
%\label{chap:taiAndPereira}
%
\makeproblem{Directivities for a short horizontal electrical dipole.}{problem:taiAndPereira:1}{
\index{half power beamwidth}
\index{electric dipole}
In \citep{tai1976approximate} a field for which directivities can be calculated exactly was used in comparisons of some directivity approximations
\index{directivity}
\index{electrical dipole!horizontal}
%
\begin{dmath}\label{eqn:taiAndPereira:140}
\BE = E_0 \lr{ \cos\theta \cos\phi \thetacap - \sin\phi \phicap }.
\end{dmath}
%
(Observe that an inverse radial dependence in \(E_0\) must be implied here for this to be a valid far-field representation of the field.)

Show that Tai \& Pereira's formula gives \( D_1 = 3 \), and \( D_2 = 1 \) respectively for this field.

Calculate the exact directivity for this field.
%
} % problem
%
\makeanswer{problem:taiAndPereira:1}{
%
The field components are
%
\begin{subequations}
\begin{dmath}\label{eqn:taiAndPereira:180}
E_\theta = E_0 \cos\theta \cos\phi
\end{dmath}
\begin{dmath}\label{eqn:taiAndPereira:200}
E_\phi = -E_0 \sin\phi
\end{dmath}
\end{subequations}
%
Using \cref{eqn:taiAndPereira:10} from the paper, the directivities are
%
\begin{dmath}\label{eqn:taiAndPereira:220}
D_1 = \frac{2}{\int_0^\pi \cos^2 \theta \sin\theta d\theta}
= \frac{2}{\evalrange{-\inv{3}\cos^3\theta}{0}{\pi}}
= 3,
\end{dmath}
%
and
%
\begin{dmath}\label{eqn:taiAndPereira:240}
D_2
= \frac{2}{\int_0^\pi \sin\theta d\theta}
= \frac{2}{\evalrange{-\cos\theta}{0}{\pi}}
= 1.
\end{dmath}
%
To find the exact directivity, first the Poynting vector is required.  That is
\index{Poynting vector}
%
\begin{dmath}\label{eqn:taiAndPereira:260}
\BP
= \frac{
 \Abs{E_0}^2
}{2 c \mu_0}
\lr{ \cos\theta \cos\phi \thetacap - \sin\phi \phicap }
\cross
\lr{ \rcap \cross \lr{ \cos\theta \cos\phi \thetacap - \sin\phi \phicap } }
= \frac{
 \Abs{E_0}^2
}{ 2 c \mu_0}
\lr{ \cos\theta \cos\phi \thetacap - \sin\phi \phicap }
\cross
\lr{ \cos\theta \cos\phi \phicap + \sin\phi \thetacap }
= \frac{
 \Abs{E_0}^2 \rcap
}{2 c \mu_0}
\lr{ \cos^2\theta \cos^2\phi + \sin^2\phi },
\end{dmath}
%
so the radiation intensity is
%
\begin{dmath}\label{eqn:taiAndPereira:280}
U(\theta, \phi) \propto \cos^2\theta \cos^2\phi + \sin^2\phi.
\end{dmath}
%
The \( \thetacap \), and \( \phicap \) contributions to this intensity, and the total intensity are all plotted in
\cref{fig:TaiAndPereiraSampleFieldThetaCapComponentThree}
%\cref{fig:TaiAndPereiraSampleFieldThetaCapComponent:TaiAndPereiraSampleFieldThetaCapComponentFig1},
%\cref{fig:TaiAndPereiraSampleFieldPhiCapComponent:TaiAndPereiraSampleFieldPhiCapComponentFig2}, and
%\cref{fig:TaiAndPereiraSampleFieldAllComponents:TaiAndPereiraSampleFieldAllComponentsFig3} respectively.

FIXME: did I save these under the right paths?  Recall thetacap and phicap reversed.
%FIXME: use mathImageFigure
%
\mathImageThreeFiguresOneLine{../figures/ece1229-antenna/TaiAndPereiraSampleFieldThetaCapComponentFig1}{../figures/ece1229-antenna/TaiAndPereiraSampleFieldPhiCapComponentFig2}{../figures/ece1229-antenna/TaiAndPereiraSampleFieldAllComponentsFig3}{Radiation intensity.}{fig:TaiAndPereiraSampleFieldThetaCapComponentThree}{scale=0.3}{sphericalPlot3d.nb}
%
%\imageFigure{../figures/ece1229-antenna/TaiAndPereiraSampleFieldThetaCapComponentFig1}
%{The \(\thetacap\) contribution to the radiation intensity.}
%{fig:TaiAndPereiraSampleFieldThetaCapComponent:TaiAndPereiraSampleFieldThetaCapComponentFig1}{0.3}
%
%\imageFigure{../figures/ece1229-antenna/TaiAndPereiraSampleFieldPhiCapComponentFig2}
%{The \(\phicap\) contribution to the radiation intensity.}
%{fig:TaiAndPereiraSampleFieldPhiCapComponent:TaiAndPereiraSampleFieldPhiCapComponentFig2}{0.3}
%
%\imageFigure{../figures/ece1229-antenna/TaiAndPereiraSampleFieldAllComponentsFig3}
%{Radiation intensity.}
%{fig:TaiAndPereiraSampleFieldAllComponents:TaiAndPereiraSampleFieldAllComponentsFig3}{0.3}

Given this the total radiated power is
%
\begin{dmath}\label{eqn:taiAndPereira:300}
P_\trad = \int_0^{2 \pi} \int_0^\pi
\lr{ \cos^2\theta \cos^2\phi + \sin^2\phi } \sin\theta d\theta d\phi
= \frac{8 \pi}{3}.
\end{dmath}
%
Observe that the radiation intensity \( U \) can also be decomposed into two components, one for each component of the original \( \BE \) phasor.
%
\begin{subequations}
\begin{dmath}\label{eqn:taiAndPereira:320}
U_\theta = \cos^2 \theta \cos^2 \phi
\end{dmath}
\begin{dmath}\label{eqn:taiAndPereira:340}
U_\phi = \sin^2 \phi
\end{dmath}
\end{subequations}
%
This decomposition allows for expression of the partial directivities in these respective (orthogonal) directions
%
\begin{subequations}
\begin{equation}\label{eqn:taiAndPereira:360}
D_\theta = \frac{4 \pi U_\theta}{P_\trad} = \frac{3}{2} \cos^2 \theta \cos^2 \phi
\end{equation}
\begin{equation}\label{eqn:taiAndPereira:380}
D_\phi = \frac{4 \pi U_\phi}{P_\trad} = \frac{3}{2} \sin^2 \phi
\end{equation}
\end{subequations}
%
The maximum of each of these partial directivities is both \( 3/2 \), giving a maximum directivity of
%
\begin{equation}\label{eqn:taiAndPereira:400}
D_0 =
\evalbar{D_\theta}{\tmax}
+\evalbar{D_\phi}{\tmax} = 3,
\end{equation}
%
the exact value from the paper.
} % answer
%
%
\makeproblem{E and H plane directivities.}{problem:taiAndPereira:2}{
\index{directivity!E-plane}
\index{directivity!H-plane}
%
In \citep{tai1976approximate} directivities associated with the half power beamwidths are given as
%
\begin{subequations}
\label{eqn:taiAndPereira:10}
\begin{dmath}\label{eqn:taiAndPereira:20}
D_1 =  \frac{\Abs{E_\theta}^2_\tmax}{\inv{2} \int_0^\pi \Abs{E_\theta(\theta, 0)}^2 \sin\theta d\theta}
\end{dmath}
\begin{dmath}\label{eqn:taiAndPereira:40}
D_2 =  \frac{\Abs{E_\phi}^2_\tmax}{\inv{2} \int_0^\pi \Abs{E_\phi(\theta, \pi/2)}^2 \sin\theta d\theta},
\end{dmath}
\end{subequations}
%
whereas \citep{balanis2005antenna} lists these as
%
\begin{subequations}
\begin{dmath}\label{eqn:taiAndPereira:60}
\inv{D_1} =  \inv{2 \ln 2} \int_0^{\Theta_{1 r}/2} \sin\theta d\theta
\end{dmath}
\begin{dmath}\label{eqn:taiAndPereira:80}
\inv{D_2} =  \inv{2 \ln 2} \int_0^{\Theta_{2 r}/2} \sin\theta d\theta.
\end{dmath}
\end{subequations}
%
Reconcile these pairs of relations.
%
} % problem
%
\makeanswer{problem:taiAndPereira:2}{
%
TODO.
} % answer
%
%
%\makeproblem{Arithmetic mean formula.}{problem:taiAndPereira:3}{
%
%\Cref{eqn:taiAndPereira:10} and the associated arithmetic mean formula
%
%\begin{dmath}\label{eqn:taiAndPereira:160}
%\inv{D_0} = \inv{2}\lr{\inv{D_1} + \inv{D_2}},
%\end{dmath}
%
%should follow from the far field approximation formula for \( U \).  Derive that result.
%
%} % problem
%
%\makeanswer{problem:taiAndPereira:3}{
%TODO.
%} % answer
%
%\makeproblem{For narrow beams.}{problem:taiAndPereira:4}{
%It is claimed that for narrow beams better approximations are
%\begin{subequations}
%\begin{dmath}\label{eqn:taiAndPereira:100}
%D_1 \approx \frac{16 \ln 2}{\Theta_{1 r}^2}
%\end{dmath}
%\begin{dmath}\label{eqn:taiAndPereira:120}
%D_1 \approx \frac{16 \ln 2}{\Theta_{2 r}^2},
%\end{dmath}
%\end{subequations}
%
%where these are derived by considering ``the asymptotic expression for the directivity of an antenna with a rotationally symmetrical power pattern of the form \( U(\theta) = \cos^m \theta, \, \theta \in [0, \pi/2] \), with a large value of \( m \)''.
%
%Derive these results.  What does it mean to have a rotationally symmetric power pattern in two different directions?
%
%} % problem
%
%\makeanswer{problem:taiAndPereira:4}{
%TODO.
%} % answer
%
%\EndArticle
