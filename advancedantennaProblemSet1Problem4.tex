%
% Copyright � 2015 Peeter Joot.  All Rights Reserved.
% Licenced as described in the file LICENSE under the root directory of this GIT repository.
%
\makeoproblem{Polarization power loss.}{advancedantenna:problemSet1:4}{2015 ps1, p4}{
\index{polarization!power loss}
Transmitting and receiving antennas operating at 1 \si{GHz} have gains of 20 and 15 \si{dB}
respectively and are separated by a distance of 1 \si{km}. Find the power delivered to a
matched load when the input power is 150 \si{W} and when
\makesubproblem{}{advancedantenna:problemSet1:4a}
both antennas are polarization matched.
\makesubproblem{}{advancedantenna:problemSet1:4b}
One antenna is linearly polarized and the other is circularly polarized.
\index{gain}
\index{matched load}
\index{polarization power loss}
} % makeoproblem
%
\makeanswer{advancedantenna:problemSet1:4}{
\makeSubAnswer{}{advancedantenna:problemSet1:4a}
%
Answering this requires an application of the Friis transmission equation.  First note that the gains in non-dB units are
\index{Friis transmission equation}
%
\begin{subequations}
\begin{equation}\label{eqn:advancedantennaProblemSet1Problem4:20}
G_1 = 10^{20/10},
\end{equation}
\begin{equation}\label{eqn:advancedantennaProblemSet1Problem4:40}
G_2 = 10^{15/10}.
\end{equation}
\end{subequations}
%
The wavelength is
\index{wavelength}
%
\begin{equation}\label{eqn:advancedantennaProblemSet1Problem4:60}
\begin{aligned}
\lambda &= \frac{c}{\nu} \\
&= \frac{3 \times 10^8 \,\si{m/s}}{10^9 \,\si{s^{-1}}} \\
&= 0.3 \, \si{m}.
\end{aligned}
\end{equation}
%
From the Friis equation, the receiving antenna has power
%
\begin{equation}\label{eqn:advancedantennaProblemSet1Problem4:80}
\begin{aligned}
P_\txtr
&= P_\txtt \lr{ \frac{\lambda}{4 \pi R}}^2 G_1 G_2
\\ &= 150 \, \si{W} \lr{ \frac{0.3 \,\si{m}}{4 \pi (10^3 \,\si{m})} }^2 10^{3.5}
\\ &= 0.27 \, \si{m W}.
\end{aligned}
\end{equation}
%
\makeSubAnswer{}{advancedantenna:problemSet1:4b}
%
Suppose that the linear polarization vector is
\index{linear polarization}
%
\begin{equation}\label{eqn:advancedantennaProblemSet1Problem4:100}
\rhocap_1 = \xcap,
\end{equation}
%
and the circular polarization vector is
\index{circular polarization}
%
\begin{equation}\label{eqn:advancedantennaProblemSet1Problem4:120}
\rhocap_2 = \inv{\sqrt{2}} \lr{ \xcap + j \ycap}.
\end{equation}
%
The polarization factor is
%
\begin{equation}\label{eqn:advancedantennaProblemSet1Problem4:140}
\Abs{\rhocap_1 \cdot \rhocap_2}^2 = \inv{2},
\end{equation}
%
so the power found in \cref{eqn:advancedantennaProblemSet1Problem4:80} must be reduced by 50 \% when there is a linear vs. circular polarization mismatch.
%
Rotating one of these polarization vectors, say the linear polarization vector, does not change the result.  For example, let
%
\begin{equation}\label{eqn:advancedantennaProblemSet1Problem4:160}
\rhocap_1 = \inv{\sqrt{a^2 + b^2}} \lr{ a \xcap + b \ycap }.
\end{equation}
%
The polarization factor is now
%
\begin{equation}\label{eqn:advancedantennaProblemSet1Problem4:180}
\begin{aligned}
\Abs{\rhocap_1 \cdot \rhocap_2 }^2
&=
\Abs{ \frac{ a}{\sqrt{2(a^2 + b^2)}} + \frac{ j b}{\sqrt{2(a^2 + b^2)}} }^2
\\ &= \inv{2 (a^2 + b^2)} \lr{ a^2 + b^2 }
\\ &= \inv{2},
\end{aligned}
\end{equation}
%
yielding the same factor of two reduction in power.
}
