%
% Copyright © 2015 Peeter Joot.  All Rights Reserved.
% Licenced as described in the file LICENSE under the root directory of this GIT repository.
%
%
\makeproblem{Chebyscheff Recurrence relation.}{problem:tschebyscheff:460}{
\index{Chebyscheff!recurrence relation}
Prove \cref{eqn:chebyscheff:100}.
} % problem
%
\makeanswer{problem:tschebyscheff:460}{
To show this, let
%
\begin{equation}\label{eqn:chebyscheff:460}
x = \cos\theta,
\end{equation}
%
\begin{equation}\label{eqn:chebyscheff:580}
2 x T_{m-1} - T_{m-2}
=
2 \cos\theta \cos((m-1) \theta) - \cos((m-2)\theta).
\end{equation}
%
Recall the cosine addition formulas
%
\begin{equation}\label{eqn:chebyscheff:540}
\begin{aligned}
\cos( a + b )
&=
\Real e^{j(a + b)}
\\ &=
\Real e^{ja} e^{jb}
\\ &=
\Real
\lr{ \cos a + j \sin a }
\lr{ \cos b + j \sin b }
\\ &=
\cos a \cos b - \sin a \sin b.
\end{aligned}
\end{equation}
%
Applying this gives
%
\begin{equation}\label{eqn:chebyscheff:600}
\begin{aligned}
2 &x T_{m-1} - T_{m-2} \\
&=
2 \cos\theta \Biglr{ \cos(m\theta)\cos\theta +\sin(m\theta) \sin\theta }
- \Biglr{
\cos(m\theta)\cos(2\theta) + \sin(m\theta) \sin(2\theta)
}
\\ &=
2 \cos\theta \Biglr{ \cos(m\theta)\cos\theta +\sin(m\theta)\sin\theta) }
- \Biglr{
\cos(m\theta)(\cos^2 \theta - \sin^2 \theta) + 2 \sin(m\theta) \sin\theta \cos\theta
}
\\ &=
\cos(m\theta) \lr{ \cos^2\theta + \sin^2\theta }
\\ &=T_m(x). \qedmarker
\end{aligned}
\end{equation}
%
} % answer
%
\makeproblem{Chebyscheff first order LDE relation.}{problem:tschebyscheff:2}{
\index{Chebyscheff!first order LDE}
Prove \cref{eqn:chebyscheff:420}.
} % problem
%
\makeanswer{problem:tschebyscheff:2}{
%
To show this, again, let
%
\begin{equation}\label{eqn:chebyscheff:470}
x = \cos\theta.
\end{equation}
%
Observe that
%
\begin{equation}\label{eqn:chebyscheff:480}
1 = -\sin\theta \frac{d\theta}{dx},
\end{equation}
%
so
%
\begin{equation}\label{eqn:chebyscheff:500}
\begin{aligned}
\frac{d}{dx}
&= \frac{d\theta}{dx} \frac{d}{d\theta}
\\ &= -\frac{1}{\sin\theta} \frac{d}{d\theta}.
\end{aligned}
\end{equation}
%
Plugging this in gives
%
\begin{equation}\label{eqn:chebyscheff:520}
\begin{aligned}
\lr{ 1 - x^2} &\frac{d}{dx} T_m(x) + m x T_m(x) - m T_{m-1}(x) \\
&=
\sin^2\theta
\lr{
-\frac{1}{\sin\theta} \frac{d}{d\theta}} \\
&\qquad \Biglr{
\cos( m \theta ) + m \cos\theta \cos( m \theta ) - m \cos( (m-1)\theta ) 
} \\
&=
-\sin\theta (-m \sin(m \theta)) + m \cos\theta \cos( m \theta ) - m \cos( (m-1)\theta ).
\end{aligned}
\end{equation}
%
Applying the cosine addition formula \cref{eqn:chebyscheff:540} gives
%
\begin{equation}\label{eqn:chebyscheff:560}
\begin{aligned}
&m \lr{ \sin\theta \sin(m \theta) + \cos\theta \cos( m \theta ) }  \\
&\quad - m
\lr{
\cos( m \theta) \cos\theta + \sin( m \theta ) \sin\theta
}
=0. \qedmarker
\end{aligned}
\end{equation}
%
} % answer
%
\makeproblem{Chebyscheff second order LDE relation.}{problem:tschebyscheff:4}{
\index{Chebyscheff!second order LDE}
Prove \cref{eqn:chebyscheff:400}.
} % problem
%
\makeanswer{problem:tschebyscheff:4}{
%
This follows the same way.  The first derivative was
%
\begin{equation}\label{eqn:chebyscheff:640}
\begin{aligned}
\frac{d T_m(x)}{dx}
&=
-\inv{\sin\theta}
\frac{d}{d\theta} \cos(m\theta)
\\ &=
-\inv{\sin\theta} (-m) \sin(m\theta)
\\ &=
m \inv{\sin\theta} \sin(m\theta),
\end{aligned}
\end{equation}
%
so the second derivative is
%
\begin{equation}\label{eqn:chebyscheff:620}
\begin{aligned}
\frac{d^2 T_m(x)}{dx^2}
&=
-m \inv{\sin\theta} \frac{d}{d\theta} \inv{\sin\theta} \sin(m\theta)
\\ &=
-m \inv{\sin\theta}
\lr{
-\frac{\cos\theta}{\sin^2\theta} \sin(m\theta) + \inv{\sin\theta} m \cos(m\theta)
}.
\end{aligned}
\end{equation}
%
Putting all the pieces together gives
%
\begin{equation}\label{eqn:chebyscheff:660}
\begin{aligned}
\lr{ 1 - x^2 } &\frac{d^2 T_m(x)}{dx^2} - x \frac{dT_m(x)}{dx} + m^2 T_{m}(x) \\
&=
m
\lr{
\frac{\cos\theta}{\sin\theta} \sin(m\theta) - m \cos(m\theta)
} \\
&\quad - \cos\theta m \inv{\sin\theta} \sin(m\theta)
+ m^2 \cos(m \theta) \\
&=
0. \qedmarker
\end{aligned}
\end{equation}
%
} % answer
%
\makeproblem{Chebyscheff orthogonality relation.}{problem:tschebyscheff:3}{
\index{Chebyscheff!orthogonality}
Prove \cref{eqn:chebyscheff:440}.
} % problem
%
\makeanswer{problem:tschebyscheff:3}{
%
First consider the 0,0 inner product, making an \( x = \cos\theta \), so that \( dx = -\sin\theta d\theta \), and
%
\begin{equation}\label{eqn:chebyscheff:680}
\begin{aligned}
\innerprod{T_0}{T_0}
&=
\int_{-1}^1 \inv{\lr{1-x^2}^{1/2}} dx
\\ &=
\int_{-\pi}^0 \lr{-\inv{\sin\theta}} -\sin\theta d\theta
\\ &=
0 - (-\pi)
\\ &= \pi.
\end{aligned}
\end{equation}
%
Note that since the \( [-\pi, 0] \) interval was chosen, the negative root of \( \sin^2\theta = 1 - x^2 \) was chosen, since \( \sin\theta \) is negative in that interval.

The m,m inner product with \( m \ne 0 \) is
%
\begin{equation}\label{eqn:chebyscheff:700}
\begin{aligned}
\innerprod{T_m}{T_m}
&=
\int_{-1}^1 \inv{\lr{1-x^2}^{1/2}} \lr{ T_m(x)}^2 dx
\\ &=
\int_{-\pi}^0 \lr{-\inv{\sin\theta}} \cos^2(m\theta) -\sin\theta d\theta
\\ &=
\int_{-\pi}^0 \cos^2(m\theta) d\theta
\\ &=
\inv{2} \int_{-\pi}^0 \lr{ \cos(2 m\theta) + 1 } d\theta
\\ &= \frac{\pi}{2}.
\end{aligned}
\end{equation}
%
% cos 2 x = 2 cos^2 x -1
% cos 2 x + 1 = 2 cos^2 x

So far so good.  For \( m \ne n \) the inner product is
%
\begin{equation}\label{eqn:chebyscheff:720}
\begin{aligned}
\innerprod{T_m}{T_m}
&=
\int_{-\pi}^0 \cos(m\theta) \cos(n\theta) d\theta
\\ &=
\inv{4} \int_{-\pi}^0
\lr{
e^{j m \theta}
+ e^{-j m \theta}
}
\lr{
e^{j n \theta}
+ e^{-j n \theta}
}
d\theta
\\ &=
\inv{4} \int_{-\pi}^0
\lr{
e^{j (m + n) \theta}
+e^{-j (m + n) \theta}
+e^{j (m - n) \theta}
+e^{j (-m + n) \theta}
}
d\theta
\\ &=
\inv{2} \int_{-\pi}^0
\lr{
\cos( (m + n)\theta )
+\cos( (m - n)\theta )
}
d\theta
\\ &=
\inv{2}
\evalrange{
\lr{
\frac{\sin( (m + n)\theta )}{ m + n }
+\frac{\sin( (m - n)\theta )}{ m - n}
}
}{-\pi}{0}
\\ &=0. \qedmarker
\end{aligned}
\end{equation}
%
} % answer
