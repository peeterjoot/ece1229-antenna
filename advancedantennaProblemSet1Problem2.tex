%
% Copyright � 2015 Peeter Joot.  All Rights Reserved.
% Licenced as described in the file LICENSE under the root directory of this GIT repository.
%
\makeoproblem{Directivity and free space loss.}{advancedantenna:problemSet1:2}{2015 problem set 1, p2}{
\index{directivity}
\index{free space loss}
A satellite dish has a diameter \( d = 1.5 \si{m} \) and an aperture efficiency of 70\%. Calculate the
directivity of the dish at 12 \si{GHz}. If the distance from a geostationary satellite is 37,000
\si{km} calculate the corresponding free-space loss in \si{dB}.
\index{aperture efficiency}
\index{directivity}
\index{free-space loss}
} % makeoproblem

\makeanswer{advancedantenna:problemSet1:2}{

Ignoring any concavity in the dish (which is probably parabolic, with physical area somewhere between \(\pi r^2\), and \( 2 \pi^2\)), the maximum effective area is
\index{effective area}
%
\begin{dmath}\label{eqn:advancedantennaProblemSet1Problem2:20}
A_{\tem}
= \epsilon_{\tem} A_\txtp
= 0.7 \pi 0.75^2
= 0.39 \pi
= 1.24 \, (\si{m^2})
\end{dmath}
%
The maximum directivity is

% \nu \lambda = c
\begin{dmath}\label{eqn:advancedantennaProblemSet1Problem2:40}
D_0
= \frac{4 \pi A_\tem}{\lambda^2}
= \frac{4 \pi A_\tem \nu^2}{c^2}
= 4 \pi \times 1.24 \,\si{m^2} \times \lr{12 \times 10^{9} \, \si{s^{-1}} }^2/\lr{3 \times 10^8 \, \si{m/s}}^2
= 2.5 \times 10^4.
\end{dmath}
%
The free-space loss factor at \( R = 37 \times 10^6 \, \si{m} \) is
%
\begin{dmath}\label{eqn:advancedantennaProblemSet1Problem2:60}
\lr{\frac{\lambda}{4 \pi R}}^2
=
\lr{\frac{c}{4 \pi R \nu}}^2
=
\lr{
\frac
{3 \times 10^8 \,\si{m/s}}
{
4 \pi
\lr{37 \times 10^6 \, \si{m}}
\lr{ 12 \times 10^9 \, \si{s^{-1}}}
}
}^2
= 2.9 \times 10^{-21}
= -205 \, \si{dB}.
\end{dmath}
}

