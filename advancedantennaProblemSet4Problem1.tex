%
% Copyright � 2015 Peeter Joot.  All Rights Reserved.
% Licenced as described in the file LICENSE under the root directory of this GIT repository.
%
\makeoproblem{Schelkunoff z-axis array, binary array.}{advancedantenna:problemSet4:1}{2015 ps4, p1}{
\index{Schelkunoff array}
\index{binary array}
%
A three-element array is placed along the z-axis. Assume that the spacing between the
elements is \( d = \lambda/2 \) and the relative amplitude excitations are \( I_1 = I_3 = 1 \) and \( I_2 = 2 \)

Use the Schelkunoff method to
%
\makesubproblem{}{advancedantenna:problemSet4:1a}
Determine the angles of the nulls when the corresponding progressive phase shifts \( a d \) are \( 0, \pi/2, \pi, 3 \pi/2 \).  Do this for each case.
%
\makesubproblem{}{advancedantenna:problemSet4:1b}
For each case plot the corresponding array factor
} % makeoproblem
%
\makeanswer{advancedantenna:problemSet4:1}{
\makeSubAnswer{}{advancedantenna:problemSet4:1a}
%
With the array elements placed at \( \Br_m = m d \zcap, m \in [0,2] \), the array factor is
%
\begin{dmath}\label{eqn:advancedantennaProblemSet4Problem1:20}
\textrm{AF} =
1 \times \lr{ e^{j \lr{ k d \cos\theta + a d} } }^0
+ 2 \times \lr{ e^{j \lr{ k d \cos\theta + a d} } }^1
+ 1 \times \lr{ e^{j \lr{ k d \cos\theta + a d} } }^2.
\end{dmath}
%
With \( z = e^{j \lr{ k d \cos\theta + a d} } \), this is
%
\begin{equation}\label{eqn:advancedantennaProblemSet4Problem1:40}
\textrm{AF} = 1 + 2 z + z^2 = \lr{ 1 + z }^2.
\end{equation}
%
This is a binary array with nulls located at \( z = -1 \).  The angles where that is the case are
%
\begin{equation}\label{eqn:advancedantennaProblemSet4Problem1:60}
k d \cos\theta + a d = (2 N + 1) \pi,
\end{equation}
%
which is, for the separation of this problem,
%
\begin{equation}\label{eqn:advancedantennaProblemSet4Problem1:80}
\frac{2 \pi}{\lambda} \frac{\lambda}{2} \cos\theta + a d = ( 2 N + 1 )\pi,
\end{equation}
%
or
%
\begin{equation}\label{eqn:advancedantennaProblemSet4Problem1:100}
\theta = \cos^{-1} \lr{
2 N + 1 - \frac{a d}{\pi}
}.
\end{equation}
%
\begin{enumerate}
\item Case I: \( ad = 0 \).

Here
%
\begin{equation}\label{eqn:advancedantennaProblemSet4Problem1:120}
\theta = \cos^{-1} \lr{
2 N + 1
},
\end{equation}
%
which has solutions at \( N = 0, -1 \) of
%
%\begin{equation}\label{eqn:advancedantennaProblemSet4Problem1:140}
\boxedEquation{eqn:advancedantennaProblemSet4Problem1:140}{
\begin{aligned}
\theta &= \cos^{-1} 1 = 0 \\
\theta &= \cos^{-1} (-1) = \pi = \ang{180}.
\end{aligned}
}
%\end{equation}
%
\item Case II: \( ad = \pi/2 \).
Here
%
\begin{equation}\label{eqn:advancedantennaProblemSet4Problem1:160}
\theta = \cos^{-1} \lr{
2 N + 1 - \inv{2}
},
\end{equation}
%
which has solutions at \( N = 0 \) of
%
%\begin{equation}\label{eqn:advancedantennaProblemSet4Problem1:180}
\boxedEquation{eqn:advancedantennaProblemSet4Problem1:180}{
\theta = \cos^{-1} (1/2) = \pi/3 = \ang{60}.
}
%\end{equation}
%
\item Case III: \( ad = \pi \).
Here
%
\begin{equation}\label{eqn:advancedantennaProblemSet4Problem1:200}
\theta = \cos^{-1} \lr{
2 N
},
\end{equation}
%
which has solutions at \( N = 0 \) of
%
%\begin{equation}\label{eqn:advancedantennaProblemSet4Problem1:220}
\boxedEquation{eqn:advancedantennaProblemSet4Problem1:220}{
\theta = \cos^{-1} 0 = \pi/2 = \ang{90}.
}
%\end{equation}
%
\item Case IV: \( ad = 3 \pi/2 \).
Here
%
\begin{equation}\label{eqn:advancedantennaProblemSet4Problem1:240}
\theta = \cos^{-1} \lr{
2 N - \frac{1}{2}
},
\end{equation}
%
which has solutions at \( N = 0 \) of
%
%\begin{equation}\label{eqn:advancedantennaProblemSet4Problem1:260}
\boxedEquation{eqn:advancedantennaProblemSet4Problem1:260}{
\theta = \cos^{-1} (-1/2) = 2 \pi/3 = \ang{120}.
}
%\end{equation}
%
\end{enumerate}
%
\makeSubAnswer{}{advancedantenna:problemSet4:1b}
%
These are plotted in \cref{fig:ps4p1PlotAdFourPlots}.

%\begin{enumerate}
%\item Case I: \( ad = 0 \).
%%\cref{fig:ps4p1PlotAdEquals0Degrees:ps4p1PlotAdEquals0DegreesFig1}.
%\item Case II: \( ad = \pi/2 \).
%%\cref{fig:ps4p1PlotAdEquals90Degrees:ps4p1PlotAdEquals90DegreesFig2}.
%\item Case III: \( ad = \pi \).
%%\cref{fig:ps4p1PlotAdEquals180Degrees:ps4p1PlotAdEquals180DegreesFig3}.
%\item Case IV: \( ad = 3 \pi/2 \).
%%\cref{fig:ps4p1PlotAdEquals270Degrees:ps4p1PlotAdEquals270DegreesFig4}.
%\end{enumerate}

\mathImageFourFiguresTwoLines{../figures/ece1229-antenna/ps4p1PlotAdEquals0DegreesFig1}{../figures/ece1229-antenna/ps4p1PlotAdEquals90DegreesFig2}{../figures/ece1229-antenna/ps4p1PlotAdEquals180DegreesFig3}{../figures/ece1229-antenna/ps4p1PlotAdEquals270DegreesFig4}{Plot \( \Abs{\textrm{AF}} \) for \( ad = 0, \pi/2, \pi, 3 \pi/2 \).}{fig:ps4p1PlotAdFourPlots}{scale=0.7}{ps4:p1plots.m}

%\onefigure{scale=0.3}{ps4p1PlotAdEquals0DegreesFig1}
%\onefigure{scale=0.3}{ps4p1PlotAdEquals90DegreesFig2}
%\onefigure{scale=0.3}{ps4p1PlotAdEquals180DegreesFig3}
%\onefigure{scale=0.3}{ps4p1PlotAdEquals270DegreesFig4}

%\mathImageFigure{../figures/ece1229-antenna/ps4p1PlotAdEquals0DegreesFig1}{Plot \( \Abs{\textrm{AF}} \) for \( ad = 0 \).}{fig:ps4p1PlotAdEquals0Degrees:ps4p1PlotAdEquals0DegreesFig1}{0.3}{ps4:p1plots.m}
%\mathImageFigure{../figures/ece1229-antenna/ps4p1PlotAdEquals90DegreesFig2}{Plot \( \Abs{\textrm{AF}} \) for \( ad = \pi/2\).}{fig:ps4p1PlotAdEquals90Degrees:ps4p1PlotAdEquals90DegreesFig2}{0.3}{ps4:p1plots.m}
%\mathImageFigure{../figures/ece1229-antenna/ps4p1PlotAdEquals180DegreesFig3}{Plot \( \Abs{\textrm{AF}} \) for \( ad = \pi \).}{fig:ps4p1PlotAdEquals180Degrees:ps4p1PlotAdEquals180DegreesFig3}{0.3}{ps4:p1plots.m}
%\mathImageFigure{../figures/ece1229-antenna/ps4p1PlotAdEquals270DegreesFig4}{Plot \( \Abs{\textrm{AF}} \) for \( ad = 3 \pi/2 \).}{fig:ps4p1PlotAdEquals270Degrees:ps4p1PlotAdEquals270DegreesFig4}{0.3}{ps4:p1plots.m}
}
